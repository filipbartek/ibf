\documentclass{article}
\usepackage[utf8]{inputenc}
\usepackage[T1]{fontenc}
\usepackage{amsmath}
\usepackage{amsfonts}
\usepackage{amssymb}

\bibliographystyle{plain}

\usepackage{hyperref}
% http://en.wikibooks.org/wiki/LaTeX/Glossary#Jump_start
% Place \usepackage{glossaries} and \makeglossaries in your preamble (after \usepackage{hyperref} if present).

% Glossary
\usepackage{glossaries}
\newacronym{dnf}{DNF}{disjunctive normal form}
\makeglossaries

\author{Filip Bártek}
\title{Minimal \acrshort{dnf} representation of false point interval Boolean functions}

\begin{document}
\maketitle

\section{Introduction}
In this text,
I will show the construction of minimal \acrshort{dnf} representation
(minimal in the number of clauses)
of false point interval Boolean functions,
i.e.~the pointwise complements of interval Boolean functions.
Doing so, I will simplify the method shown in \cite{Dubovsky2012}
and extend the results shown in \cite{Schieber2005154}.

\bibliography{fibf}

\printglossaries

\end{document}
