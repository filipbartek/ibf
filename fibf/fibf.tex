\documentclass{article}
\usepackage[utf8]{inputenc}
\usepackage[T1]{fontenc}
\usepackage{amsmath}
\usepackage{amsfonts}
\usepackage{amssymb}

% Theorem environments
\usepackage{amsthm}
\theoremstyle{definition}
\newtheorem{definition}{Definition}[section]

% Macros
\newcommand{\interval}[2]{[#1, #2]}
\newcommand{\finterval}[2]{\overline{\interval{#1}{#2}}}
\newcommand{\bit}[2]{#1^{[#2]}}
\newcommand{\bits}[3]{#1^{\interval{#2}{#3}}}

\bibliographystyle{plain}

\usepackage{hyperref}
% http://en.wikibooks.org/wiki/LaTeX/Glossary#Jump_start
% Place \usepackage{glossaries} and \makeglossaries in your preamble (after \usepackage{hyperref} if present).

% Glossary
\usepackage{glossaries}
\newacronym{dnf}{DNF}{disjunctive normal form}
\makeglossaries

\author{Filip Bártek}
\title{Minimal \acrshort{dnf} representation of false point interval Boolean functions}

\begin{document}
\maketitle

\section{Introduction}
In this text,
I will show the construction of minimal \acrshort{dnf} representation
(minimal in the number of clauses)
of false point interval Boolean functions,
i.e.~the pointwise complements of interval Boolean functions.
Doing so, I will simplify the method shown in \cite{Dubovsky2012}
and extend the results shown in \cite{Schieber2005154}.

\section{Definitions}
\begin{definition}[Interval Boolean function]
Let $a \leq b$ be two $n$-bit numbers ($0 \leq a \leq b \leq 2^n - 1$).
Then $f^n_{\interval{a}{b}}: 2^n \rightarrow 2$ is a function defined as follows:
% http://en.wikibooks.org/wiki/LaTeX/Advanced_Mathematics#Using_aligned_braces_for_piecewise_functions
% http://en.wikipedia.org/wiki/Piecewise
\[f^n_{\interval{a}{b}} (x) = \left\{
  \begin{array}{lr}
    1 & $if $ x \in \interval{a}{b}\\
    0 & $otherwise$
  \end{array}
\right.
\]
\end{definition}

\begin{definition}[False point interval Boolean function]
Let $a \leq b$ be two $n$-bit numbers ($0 \leq a \leq b \leq 2^n - 1$).
Then $f^n_{\finterval{a}{b}}: 2^n \rightarrow 2$ is a function defined as follows:
\[f^n_{\finterval{a}{b}} (x) = \left\{
  \begin{array}{lr}
    1 & $if $ x \notin \interval{a}{b}\\
    0 & $otherwise$
  \end{array}
\right.
\]
\end{definition}

\section{Spanning false point interval Boolean functions}
Let $0 \leq a \leq b \leq 2^n - 1$.
We need to find a set of ternary vectors of minimal cardinality that spans exactly $\finterval{a}{b}$.

We'll consider two cases separately:

\begin{enumerate}
\item $\bit{a}{1} = \bit{b}{1}$
\item $\bit{a}{1} \neq \bit{b}{1}$
\end{enumerate}

We'll deal with the first case by solving the instance $\finterval{\bits{a}{2}{n}}{\bits{b}{2}{n}}$
and adding an extra ternary vector.

We'll transform the second case to an instance of spanning a true point interval Boolean function,
which can be handled using the method introduced in \cite{Schieber2005154}.

\subsection{$\bit{a}{1} = \bit{b}{1}$}
% TODO: Finish.

\subsection{$\bit{a}{1} \neq \bit{b}{1}$}
% TODO: Finish.

\bibliography{fibf}

\printglossaries

\end{document}
