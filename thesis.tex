%%% The main file.


% Build process:

% pdflatex thesis
% bibtex thesis
% makeglossaries thesis
% pdflatex thesis
% pdflatex thesis

% User command for Texmaker:
% pdflatex % | bibtex % | makeglossaries % | pdflatex % | pdflatex %


% Based on the template "diplomky.git" by Martin Mares
% http://www.ucw.cz/gitweb/?p=diplomky.git


% Style guide (directive):
% http://www.mff.cuni.cz/vnitro/dekan/2015/smer01.htm


%% Settings for single-side (simplex) printing
\documentclass[12pt,a4paper]{report}

\usepackage[utf8]{inputenc}

\usepackage[english]{babel}
\usepackage{ibf}
\usepackage{thesis}

\bibliographystyle{plain}

\begin{document}

%%% Title page of the thesis and other mandatory pages

% Somewhat relaxed hyphenation
\lefthyphenmin=2
\righthyphenmin=2

%%% Title page of the thesis

\pagestyle{empty}
\hypersetup{pageanchor=false}
\begin{center}

\large

Charles University in Prague

\medskip

Faculty of Mathematics and Physics

\vfill

{\bf\Large \MakeUppercase{\ThesisType{} thesis}}

\vfill

\centerline{\mbox{\includegraphics[width=60mm]{img/logo.pdf}}}

\vfill
\vspace{5mm}

{\LARGE\ThesisAuthor}

\vspace{15mm}

{\LARGE\bfseries\ThesisTitle}

\vfill

\Department

\vfill

\begin{tabular}{rl}

Supervisor of the \ThesisType{} thesis: & \Supervisor \\
\noalign{\vspace{2mm}}
Study programme: & \StudyProgramme \\
\noalign{\vspace{2mm}}
Specialization: & \StudyBranch \\
\end{tabular}
\todomaybe[inline]{Mares: \quot{Study branch};
\texttt{dp\_uprava\_en.pdf}: \quot{Specialization}}

\vfill

% Zde doplňte rok
Prague \YearSubmitted

\end{center}

\newpage

%%% Here should be a bound sheet included -- a signed copy of the "bachelor
%%% thesis assignment". This assignment is NOT a part of the electronic
%%% version of the thesis. DO NOT SCAN.

%%% Dedication

\openright

\noindent
\Dedication

\newpage

%%% A page with a solemn declaration to the bachelor thesis

\openright
\hypersetup{pageanchor=true}
\pagestyle{plain}
\pagenumbering{roman}
\vglue 0pt plus 1fill

\noindent
I declare that I carried out this \ThesisType{} thesis independently, and only with the cited
sources, literature and other professional sources.

\medskip\noindent
I understand that my work relates to the rights and obligations under the Act No.~121/2000 Sb.,
the Copyright Act, as amended, in particular the fact that the Charles
University in Prague has the right to conclude a license agreement on the use of this
work as a school work pursuant to Section 60 subsection 1 of the Copyright Act.

\vspace{10mm}

\hbox{\hbox to 0.5\hsize{%
In ........ date ............	% FIXME!
\hss}\hbox to 0.5\hsize{%
signature
\hss}}

\vspace{20mm}
\newpage

%%% Mandatory information page of the thesis

\openright

\vbox to 0.5\vsize{
\setlength\parindent{0mm}
\setlength\parskip{5mm}

\todonote[inline]{Czech data is required by \texttt{dp\_uprava\_en.pdf}.}
\begin{otherlanguage}{czech}
Název práce:
\NazevPrace

Autor:
\AutorPrace

\TypPracoviste:
\Katedra

Vedoucí diplomové práce:
\Vedouci, \KatedraVedouciho

Abstrakt:
\Abstrakt

Klíčová slova:
\KlicovaSlova
\end{otherlanguage}

\vss}\nobreak\vbox to 0.49\vsize{
\setlength\parindent{0mm}
\setlength\parskip{5mm}

Title:
\ThesisTitle

Author:
\ThesisAuthor

\DeptType:
\Department

Supervisor:
\Supervisor, \SupervisorsDepartment

Abstract:
\Abstract
\todomaybe[inline]{Remove mathematical expressions from abstract.}

Keywords:
\Keywords

\vss}

\newpage


\openright
\pagestyle{plain}
\pagenumbering{arabic}
\setcounter{page}{1}


%%% A page with automatically generated content of the bachelor thesis. For
%%% a mathematical thesis, it is permissible to have a list of tables and abbreviations,
%%% if any, at the beginning of the thesis instead of at its end.

\tableofcontents

% Introduction
%   Definitions
%     k-interval function
% 1-interval
%   Definitions
%     Spanning set
%     Disjoint spanning set
%   Zaks algorithm (without proof of optimality)
%     Prefix and suffix
%     General case
% 2-interval (?)
%    2-break - transform to 1-interval
%      break < step
%    3-break - difficult - show example from Dubovsky
% k-interval
%   2k-approximation algorithm
%     2k-approximates in disjoint case as well
%   Natural generalization of 2-approximation algorithm
%     for 2-interval functions doesn't perform much better
%     than 2k (?)
%     Relate to Dubovsky's 2-approx. alg. for 2-int. fns.
% Conclusion

%%% Each chapter is kept in a separate file
\chapter*{Introduction}
\addcontentsline{toc}{chapter}{Introduction}

An $n$-ary Boolean function takes an $n$-tuple of Boolean values as input
and outputs a single Boolean value.
A Boolean function can be represented in various ways.
A common way to represent Boolean functions
is the truth table,
which explicitly lists the output value for each possible input $n$-tuple \citep[Definition 1.2]{Crama2011}.
For example,
the simple Boolean function $f^2_{\wedge}$ that realizes the binary logical operator of conjunction
can be represented by the following table:
\begin{center}
\begin{tabular}{cc|c}
$x_1$ & $x_2$ & $f^2_{\wedge}(x_1, x_2)$ \\
\hline
$0$ & $0$ & $0$ \\
$0$ & $1$ & $0$ \\
$1$ & $0$ & $0$ \\
$1$ & $1$ & $1$ \\
\end{tabular}
\end{center}

The upper index of $f^2_{\wedge}$
signifies the arity of the function.
In this case it is $2$,
meaning it is a function of two Boolean variables,
or equivalently of pairs of Boolean values,
or binary vectors of length $2$.

Other representations of Boolean functions
include
Boolean formulas \citep[Definition 1.4]{Crama2011},
circuits
\citep[Definition 3.1]{Wegener1987}
and binary decision diagrams \citep[Section 1.12.3]{Crama2011}.
Another possible representation is sets of intervals,
which is the one we study in this thesis.
We will relate it to the \acrfull{dnf}
Boolean formula representation.
More specifically,
we will look for efficient ways to transform an interval representation
to an equivalent \acrshort{dnf} representation.

The \acrshort{dnf} representation of Boolean functions
has been applied in constraint-based software
and hardware
testing systems
\citep{DeMillo1991,Lewin1995}.
% Section IV.A, p.~906
% Section 4, p.~47
In these systems,
\acrshort{dnf} formulas act as characteristic functions
of constrained variable domains.
It can be especially useful to represent the constrained
domains with disjoint spanning sets,
since these allow fast uniform polling
of admissible values
\citep{Schieber2005154}.

Since the constrained variables are typically numeric
and comparison is one of the basic constraints,
the variable domains are often constrained
to small sets of intervals.
\todo{Cite some source, e.g.~Handbook of Constraint Programming, or simplify and cite Schieber.}
Therefore it can be interesting to look
for compact \acrshort{dnf}
representations of intervals.

The notion of
interval Boolean functions was introduced
by \citet{Schieber2005154}.
A pair of $n$-bit integers $a, b$
defines the $1$-interval function
$\intervalfn{f}{n}{a}{b}$:
$$
\apply{\intervalfn{f}{n}{a}{b}}{x} = 1
\iff a \leq x \leq b
$$

Note that such function
is the characteristic function of the interval
$\interval{a}{b}$.

\citeauthor{Schieber2005154} showed an efficient method
to find a minimum \acrfull{dnf} representation
of any $1$-interval Boolean function
given by a pair of endpoints \citep{Schieber2005154}.

Since the problem of Boolean \acrshort{dnf} minimization is in general
$\Sigmatwop$-complete \citep{Umans1998},
it is an interesting question how difficult it is
to minimize \acrshort{dnf} representation of
a $k$-interval function,
that is a function defined by a set of $k$ intervals.
\citeauthor{Dubovsky2012} investigated the problem
in the case of $2$-interval functions
\citep{Dubovsky2012}.
Another useful measure
is to consider the number of switches of the function,
which correspond to the inner interval endpoints
(we will give a precise definition
in \cref{sec:lswitch}).

In this thesis we review the existing results regarding
minimization of \acrshort{dnf} representations of
single- and multi-interval functions,
and then present four new results:

\begin{itemize}
\item a simplified algorithm
for minimizing \acrshort{dnf} representations
of $2$-switch
$2$-interval functions (\cref{sec:2int2switch}),

\item
an argument that shows
that the technique used for proving the optimality
of \citeauthor{Schieber2005154}'s
$1$-interval minimization algorithm
\citep{Schieber2005154}
and \citeauthor{Dubovsky2012}'s
$2$-switch $2$-interval minimization algorithm
\citep{Dubovsky2012}
can not be used in general for
functions with $3$ or more switches
(\cref{sec:3switch}),

\item
a simple approximation algorithm that,
given the endpoints of intervals
of a $k$-interval function $f$,
finds a \acrshort{dnf} representation of $f$
that is at worst
$2k$ times larger
than an optimal \acrshort{dnf} representation
(\cref{chap:2kapprox}), and

\item
a more sophisticated approximation algorithm
for $k$-interval functions
which performs better than the simple $2k$ approximation algorithm,
although we show that the relative improvement
\todonote{Multiplicative improvement; ratio.}
diminishes for large $k$
(\cref{chap:betterapprox}).
\end{itemize}

\chapter{Single interval functions}

In this chapter,
I will introduce an efficient algorithm for optimal spanning
of single interval Boolean functions,
originally shown by Schieber et al.\cite{Schieber2005154}

In the chapters that follow,
I will use this algorithm as a procedure
in order to span multi-interval functions.

In the remainder of this chapter,
$a$ and $b$ ($a \leq b$) will denote the endpoints
of the spanned interval
and $n$ the number of input bits.
Thus,
we'll be spanning the function $\intervalfn{f}{n}{a}{b}$.

The algorithm is recursive.

\section{Trivial cases}

First we'll deal with the trivial interval functions.

\subsection{\texorpdfstring{$a = b$}{a = b}}

We span the interval $\interval{a}{a}$
with the single ternary vector $a$.

From now on,
let $a < b$.

\subsection{
\texorpdfstring{$a = \rep{0}{n}$}{a = 0...0}
and
\texorpdfstring{$b = \rep{1}{n}$}{b = 1...1}
}

We span the interval
$\interval{\rep{0}{n}}{\rep{1}{n}}$
with the single ternary vector $\rep{\phi}{n}$.

From now on,
let $a > \rep{0}{n}$ or $b < \rep{1}{n}$.

\section{Prefix and suffix case}
\label{sec:prefixsuffix}

Since we have dealt with the trivial cases,
we are now left with the situation
$\rep{0}{n} \leq a < b \leq \rep{1}{n}$
and
$a > \rep{0}{n}$ or $b < \rep{1}{n}$.

Let's consider the prefix case,
that is $\interval{\rep{0}{n}}{b}$ ($a = \rep{0}{n}$),
and the suffix case,
that is $\interval{a}{\rep{1}{n}}$ ($b = \rep{1}{n}$).

\begin{definition}[Complement]
The complement of a binary symbol $x$ ($x \in \booldom$),
denoted $\compl{x}$,
is $1 - x$.

We get the complement of a binary vector
by flipping all of its bits,
that is $\compl{x} = (\compl{x_1}, \ldots,
\compl{x_n})$.

The complement of the $\phi$ symbol is $\phi$
($\compl{\phi} = \phi$).

To obtain the complement of a ternary vector,
we flip all its fixed bits.
\end{definition}

Note that prefix and suffix cases are complementary --
we may transform a suffix instance
$\interval{a}{\rep{1}{n}}$
to a prefix instance $\interval{\rep{0}{n}}{\compl{a}}$.
Flipping the polarity of the resulting ternary vectors
yields a solution of the initial suffix instance
$\interval{a}{\rep{1}{n}}$.
This means it suffices to solve the prefix case,
which is what we'll do.

Let $a = \rep{0}{n}$ and $b < \rep{1}{n}$.
Let $c$ be the $n$-bit number $b + 1$.
Since $b < \rep{1}{n}$,
we don't need more than $n$ bits to encode $c$.

The algorithm produces one ternary vector
for each $1$-bit in $c$.
If $o$ is a position of a $1$-bit in $c$
($\bit{c}{o} = 1$),
then the corresponding ternary vector
is $\bits{c}{1}{o - 1} 0 \rep{\phi}{n - o}$.
Thus we get the following spanning set:

\begin{equation*}
\mathcal{T} =
\{\bits{c}{1}{o - 1} 0 \rep{\phi}{n - o} | \bit{c}{o} = 1\}
\end{equation*}

\begin{theorem}
$\mathcal{T}$ spans exactly the interval
$\interval{\rep{0}{n}}{b}$ (feasibility).
\end{theorem}

\begin{proof}
To see that every number spanned by $\mathcal{T}$
is in $\interval{\rep{0}{n}}{b}$,
note that all the numbers spanned by $\mathcal{T}$
are smaller than $c$,
since their \acrlong{msb} different from $c$
must be a $0$-bit
and they must differ from $c$.

On the other hand,
every number $x$ smaller than $c$ must differ from $c$,
and the leftmost different bit must be $0$ in $x$
and $1$ in $c$.
Let $o$ be its position.
Then
$\bits{c}{1}{o - 1} 0 \rep{\phi}{n - o} \in \mathcal{T}$
spans $x$.
\end{proof}

\begin{theorem}
$\mathcal{T}$ is minimal in size (optimality).
\end{theorem}

\begin{proof}
We'll construct a set $V$ of $|\mathcal{T}|$ true points
no pair of which can be spanned by a single ternary vector.
Dubovský\cite{Dubovsky2013} calls such sets ,,orthogonal''.
Doing so, we'll show a lower bound $|\mathcal{T}|$
on the size of feasible solutions,
proving optimality of $\mathcal{T}$.

Similarly to the spanning vectors,
the orthogonal true points correspond to $1$-bits of $c$:

\begin{equation}
V =
\{\bits{c}{1}{o - 1} 0 \bits{c}{o + 1}{n} |
\bit{c}{o} = 1\}
\end{equation}

Clearly $|V| = |\mathcal{T}|$.

It's easy to see that any ternary vector that spans two
% TODO: Not so easy to see for a fresh reader.
% Go into more detail.
different points in $V$ must also span the false point $c$,
so it can't be a part of the solution.
Also note that all points in $V$ are smaller than $c$,
so they are true points.

If there was a feasible spanning set
of size smaller than $|V|$,
at least one of its vectors would need to span at least
two points in $V$.
As we have shown, such ternary vector would necessarily
also span the false point $c$,
leading to contradiction with the set's feasibility.
\end{proof}

\section{General case}

Having solved the trivial and prefix and suffix cases,
we are left with the situation
$\rep{0}{n} < a < b < \rep{1}{n}$.

If $a$ and $b$ have the same \acrshort{msb},
we recursively span
$\interval{\bits{a}{2}{n}}{\bits{b}{2}{n}}$
and prepend the \acrshort{msb}
to the solution.

Let $\bit{a}{1} \neq \bit{b}{1}$.
Since $a \leq b$,
necessarily $\bit{a}{1} = 0$
and $\bit{b}{1} = 1$.
This restriction leaves us with
four possible combinations of pairs
of \acrshort{msb}s:

\begin{enumerate}
\item $\bits{a}{1}{2} = 01$, $\bits{b}{1}{2} = 10$
\item $\bits{a}{1}{2} = 00$, $\bits{b}{1}{2} = 10$
\item $\bits{a}{1}{2} = 01$, $\bits{b}{1}{2} = 11$
\item $\bits{a}{1}{2} = 00$, $\bits{b}{1}{2} = 11$
\end{enumerate}

Following Schieber et al.,\cite{Schieber2005154}
we'll deal with each of these cases separately.
We won't prove feasibility nor optimality in this text.
Please refer to the original article for the proofs.

\subsection{\texorpdfstring
{$\bits{a}{1}{2} = 01$ and $\bits{b}{1}{2} = 10$}
{a[1,2] = 01 and b[1,2] = 10}
}

In this case,
we'll span the two sub-intervals
$\interval{a}{0 \rep{1}{n-1}}$
and
$\interval{1 \rep{0}{n-1}}{b}$
separately.
Note that after leaving out the initial shared bit,
they are prefix and suffix interval, respectively,
so the algorithm from section \ref{sec:prefixsuffix}
is used to span each of the sub-intervals.

\subsection{\texorpdfstring
{$\bits{a}{1}{2} = 00$ and $\bits{b}{1}{2} = 10$}
{a[1,2] = 00 and b[1,2] = 10}
}
\label{sec:0010}

In this case,
we divide the interval in three sub-intervals:

\begin{itemize}
\item $\interval{a}{00 \rep{1}{n-2}}$
\item $\interval{01 \rep{0}{n-2}}{01 \rep{1}{n-2}}$
\item $\interval{10 \rep{0}{n-2}}{b}$
\end{itemize}

The subinterval
$\interval{01 \rep{0}{n-2}}{01 \rep{1}{n-2}}$
is spanned by the single ternary vector
$01 \rep{\phi}{n-2}$.

The subintervals
$\interval{a}{00 \rep{1}{n-2}}$
and
$\interval{10 \rep{0}{n-2}}{b}$
are spanned together as follows:

\begin{itemize}
\item Recursively solve the $(n-1)$-bit instance
$\interval{0 \bits{a}{3}{n}}{1 \bits{b}{3}{n}}
= \interval{\bit{a}{1} \bits{a}{3}{n}}{\bit{b}{1} \bits{b}{3}{n}}$
\item Insert a $0$-bit in the second position
of the vectors from the resulting spanning set
\end{itemize}

\subsection{\texorpdfstring
{$\bits{a}{1}{2} = 01$ and $\bits{b}{1}{2} = 11$}
{a[1,2] = 01 and b[1,2] = 11}
}

This case is complementary to case \ref{sec:0010}.
As in suffix case,
note that flipping the bits in the endpoints
transforms this case to case \ref{sec:0010}
and flipping the fixed bits in the resulting spanning set
preserves correctness.

\subsection{\texorpdfstring
{$\bits{a}{1}{2} = 00$ and $\bits{b}{1}{2} = 11$}
{a[1,2] = 00 and b[1,2] = 11}
}

% TODO: Solve the general case
% (without proving optimality
% and perhaps also feasibility).
\chapter{\texorpdfstring{$2k$}{2k}-approximation algorithm
for minimizing \texorpdfstring{\acrshort{dnf}}{DNF} representation
of \texorpdfstring{$k$}{k}-interval Boolean functions}

\section{Introduction}
In this chapter,
an algorithm will be shown that computes
a small \acrshort{dnf} representation
of a Boolean function given as a set of $k$ intervals.
The input intervals are represented by pairs of endpoints
($n$-bit numbers).
An approximation ratio of $2k$ will be proved.

\section{Definitions}

\begin{definition}[$k$-interval Boolean function]
\label{def:kibf}
Let $a_1, b_1, \ldots, a_k, b_k$ be $n$-bit numbers
such that $0 \leq a_1$,
$a_1 \leq b_1$,
$b_1 \leq a_2 - 2$,
$\ldots$,
$b_k \leq 2^n - 1$.
Then $f^n_{\interval{a_1}{b_1}, \ldots, \interval{a_k}{b_k}}: \booldom{}^n \rightarrow \booldom{}$ is a function defined as follows:
\[f^n_{\interval{a_1}{b_1}, \ldots, \interval{a_k}{b_k}} (x) = \left\{
  \begin{array}{lr}
    1 & $if $ x \in \interval{a_i}{b_i}$ for some $i\\
    0 & $otherwise$
  \end{array}
\right.
\]
\end{definition}
Note that the adjacent intervals
are required to be separated by at least one false point.

\section{Algorithm}

\subsection{Description}
\paragraph{Input}
Numbers $a_1, b_1, \ldots, a_k, b_k$
that satisfy the inequalities in definition \ref{def:kibf}

\paragraph{Output}
A set of ternary vectors

\paragraph{Procedure}
The algorithm goes through
all the intervals $\interval{a_i}{b_i}$.
For each $i$, the longest common prefix of $a_i$ and $b_i$
is computed. Let $j$ be its length.
Note that $\bit{a}{j+1} = 0$ and $\bit{b}{j+1} = 1$.
Now let $a'' = \bits{a_i}{j+2}{n}$
and $b'' = \bits{b_i}{j+2}{n}$.
Optimally span the suffix interval
$\interval{a''}{1^{n-j-1}}$
and the prefix interval
$\interval{0^{n-j-1}}{b''}$
using the (linear time) algorithm
introduced in \cite{Schieber2005154}.
Prepend $\bits{a_i}{1}{j+1}$
and $\bits{b_i}{1}{j+1}$
to the respective ternary vectors
and add them to the output spanning set.

% TODO: Treat (explicitly) the degenerate cases:
% a_1 = 0
% b_k = 2^n - 1
% j = n
% j = n - 1

\subsection{Correctness}
\begin{theorem}
The algorithm spans exactly
$f^n_{\interval{a_1}{b_1}, \ldots, \interval{a_k}{b_k}}$.
\end{theorem}

\begin{proof}
This is easy to see from the fact that the subintervals
form a partition of the multi-interval
(i.e.~the set of all true points)
% TODO: Use a better term if available
and that each of them is spanned exactly
by the suffix or prefix procedure.
\end{proof}

\subsection{Approximation ratio}
\begin{theorem}
Let $\mathcal{T}_{opt}$ be an optimal spanning set of
$f^n_{\interval{a_1}{b_1}, \ldots, \interval{a_k}{b_k}}$
and let $\mathcal{T}_{approx}$ be the spanning set returned
by the algorithm.
We claim that:
\begin{equation}
|\mathcal{T}_{approx}| \leq 2k |\mathcal{T}_{opt}|
\end{equation}
\end{theorem}

\begin{proof}
Let $\mathcal{T}_x$ be the largest ($n$-bit) spanning set
of a "suffix" or "prefix" subinterval
% TODO: Clarify: the interval is prefix or suffix
% on n - j - 1 bits, but not in general on n bits.
added in the algorithm.
Without loss of generality,
let the respective subinterval be "prefix"
$\interval{\bits{b_i}{1}{j+1}}{b_i}$.
From \cite[p.~36]{Dubovsky2012} we know that
there is an orthogonal set
of $\interval{\bits{b_i}{1}{j+1}}{b_i}$
of size $|\mathcal{T}_x|$,
and moreover that its orthogonality only depends
on the false point $b+1$.
Note, however, that $b+1$ is also a false point in
$f^n_{\interval{a_1}{b_1}, \ldots, \interval{a_k}{b_k}}$.
Thus we obtain an orthogonal set of size $|\mathcal{T}_x|$
for the $k$-interval function,
limiting the size of its optimal spanning set
$|\mathcal{T}_{opt}| \geq |\mathcal{T}_x|$.

Since $\mathcal{T}_x$ is the largest
of the $2k$ partial sets used to span the function
in the approximation algorithm,
we know that
$|\mathcal{T}_{approx}| \leq 2k |\mathcal{T}_x|$.

Joining the inequalities together we conclude:
$|\mathcal{T}_{approx}| \leq 2k |\mathcal{T}_x| \leq
2k |\mathcal{T}_{opt}|$.
\end{proof}

\chapter*{Conclusion}
\addcontentsline{toc}{chapter}{Conclusion}

We have shown that
with respect to finding
a minimum \acrshort{dnf} representation,
$2$-switch $2$-interval functions
can be reduced to $1$-interval functions.

Then we showed that for every $l \geq 3$,
there is an $l$-switch function that is not coverable,
extending \citeauthor{Dubovsky2012}'s result
about $3$-switch functions.
This suggests that possible future algorithms
that optimally span multi-interval functions
are going to require a different approach
for proving optimality.

To remedy the difficulty with spanning multi-interval
functions,
we introduced a $2k$-approximation algorithm
for minimizing \acrshort{dnf} representations
of $k$-interval functions.
The algorithm naturally extends
\citeauthor{Schieber2005154}'s suffix-prefix
approximation algorithm
for disjoint spanning sets of $1$-interval functions
\citep[section 6]{Schieber2005154}.
\todo{Cite in section 2kapprox as well.}
Our algorithm produces disjoint spanning sets
and has the approximation ratio of $2k$
for the problem of finding
a minimum disjoint representation as well.

Some interesting questions are still left to be answered
regarding multi-interval Boolean functions.
Since general Boolean minimization is
$\Sigma_2^p$-hard \citep{Umans1998},
there must
\todo{Really? Probably yes; note that we can translate a $k$-interval function in a spanning set of size $2kn$.}
be some $k$ such that minimization
of $k$-interval functions is $\Sigma_2^p$-hard,
and a $k'$ for which minimization is NP-hard.
\citeauthor{Schieber2005154} showed that such $k' > 1$,
which is the only bound shown so far.
There is a substantial difference
between $1$- and $2$-interval functions
since the former are coverable in general,
while the latter are not.
This prevents us from using a similar approach
to minimizing $2$-interval functions.
I suspect this difference could correspond to
a difference in computational complexity of the problem.
\todo{Is it a good idea to "suspect" in a thesis?}
\todomaybe{Suspect more explicitly, e.g. $3$-switch functions are NP-hard.}

The $2k$-approximation algorithm presented in this thesis
is constructed so that we can use the orthogonal sets
for proving the approximation ratio.
There may be a better approximation algorithm
for $k$-interval functions
\todo{We may have shown a better one in \texttt{betterapprox}.}
and searching for it could provide an interesting insight
in proving lower bounds for minimum \acrshort{dnf}
representation size of the functions in question.


%%% Bibliography
%%% Seznam použité literatury je zpracován podle platných standardů. Povinnou citační
%%% normou pro bakalářskou práci je ISO 690. Jména časopisů lze uvádět zkráceně, ale jen
%%% v kodifikované podobě. Všechny použité zdroje a prameny musí být řádně citovány.

\addcontentsline{toc}{chapter}{\bibname}
\bibliography{ibf}


%%% Figures used in the thesis (consider if this is needed)
%\listoffigures

%%% Tables used in the thesis (consider if this is needed)
%\listoftables

%%% Abbreviations used in the thesis, if any, including their explanation
%\chapwithtoc{List of Abbreviations}
\printglossaries
% TODO: Make glossary appear in table of contents

%%% Attachments to the bachelor thesis, if any (various additions such
%%% as programme extracts, diagrams, etc.). Each attachment must be referred to
%%% at least once from one's own text of the thesis. Attachments are numbered.
%\chapwithtoc{Attachments}

\openright
\end{document}
