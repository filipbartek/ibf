% https://d3s.mff.cuni.cz/~ceres/adm/jak-obhajit-diplomovou-praci.php

% $Header: /Users/joseph/Documents/LaTeX/beamer/solutions/generic-talks/generic-ornate-15min-45min.en.tex,v 90e850259b8b 2007/01/28 20:48:30 tantau $

\documentclass{beamer}

% This file is a solution template for:

% - Giving a talk on some subject.
% - The talk is between 15min and 45min long.
% - Style is ornate.


% Copyright 2004 by Till Tantau <tantau@users.sourceforge.net>.
%
% In principle, this file can be redistributed and/or modified under
% the terms of the GNU Public License, version 2.
%
% However, this file is supposed to be a template to be modified
% for your own needs. For this reason, if you use this file as a
% template and not specifically distribute it as part of a another
% package/program, I grant the extra permission to freely copy and
% modify this file as you see fit and even to delete this copyright
% notice.

\usepackage[utf8]{inputenc}

\mode<presentation>
{
% https://www.hartwork.org/beamer-theme-matrix/

  %\usetheme{Warsaw}
  \usetheme{Boadilla}

  %\setbeamercovered{transparent}
  % or whatever (possibly just delete it)
}

\usepackage[english,czech]{babel}

%\usepackage{times} % Change font

\usepackage{natbib}
\bibliographystyle{plainnat}

\usepackage{appendixnumberbeamer}

% Local packages
\usepackage{defense}
\usepackage{ibf-commands}
\usepackage{glossary-cs}
\usepackage{beamer-cs}
\usepackage{ibf-todonotes}
\usepackage{ibf-fig}

\theoremstyle{remark}
\newtheorem{observation}{Pozorování}
\newtheorem{algorithm}{Algoritmus}
\newtheorem{conclusion}{Závěr}

%\AtBeginSection{}
\AtBeginSubsection{} % Neukazuj Outline frame na začátku subsekce

% http://tex.stackexchange.com/a/136995
% Prevent "Sekce *" from showing up in \sectionpage.
\defbeamertemplate{section page}{mine}[1][]{%
  \begin{centering}
    {\usebeamerfont{section name}\usebeamercolor[fg]{section name}#1}
    \vskip1em\par
    \begin{beamercolorbox}[sep=12pt,center]{part title}
      \usebeamerfont{section title}\insertsection\par
    \end{beamercolorbox}
  \end{centering}
}
\setbeamertemplate{section page}[mine]

\newcommand{\downmapsto}{\downarrow}

\begin{document}

\frame{\titlepage}

%\frame{\tableofcontents}

% - Exactly two or three sections (other than the summary).
% - At *most* three subsections per section.
% - Talk about 30s to 2min per frame. So there should be between about
%   15 and 30 frames, all told.

\begin{frame}{Motivace}

\todonote[inline]{Některé systémy pro testování hardwaru a softwaru používají pro generování testovacích vzorů splňující podmínky na celočíselných proměnných.}

\begin{itemize}
\item Generování testovacích vzorů (hardware, software)
\nocite{Lewin1995,DeMillo1991}
\item Omezující podmínky na celočíselných proměnných
\begin{itemize}
\item Lineární (nerovnosti) $\rightarrow$ intervaly
\item Nelineární $\rightarrow$ \acrshort{dnf}
\end{itemize}
\end{itemize}

\end{frame}

\section{Definice}

\subsection{Intervalové funkce}

\begin{frame}{$1$-intervalové funkce}
\todonote[inline]{Když nahlížíme na vstupní binární vektory booleovské funkce jako na celá čísla, ...}
$$
\text{$n$-bitové celé číslo} \sim \text{binární vektor délky $n$}
$$
\todonote[inline]{Na binární (neboli booleovské) vektory nahlížíme jako na binární reprezentace celých čísel, což nám dává především (zajímavé) uspořádání na binárních vektorech ($\leq$).}

\begin{definition}[$1$-intervalová funkce \citep{Schieber2005154}]
$f_{\interval{a}{b}}(x) = 1 \iff a \leq x \leq b$
\end{definition}

\begin{example}[$1$-intervalová funkce]
$$f_{\interval{001}{101}}$$
\todonote[inline]{Ternární funkce definovaná intervalem $001$, $101$}

\begin{figure}[h]
\centering
\begin{tikzpicture}[
    point/.style = {draw, circle, inner sep=2pt},
    truepoint/.style = {point, fill = black},
    trueinterval/.style = {line width = 2pt}
]

\pgfmathsetmacro{\nn}{3}

\pgfmathsetlengthmacro{\labeldist}{8pt}
\pgfmathsetlengthmacro{\spandist}{12pt}

\drawaxis{\nn}{\labeldist}
\drawinterval{001}{101}

\end{tikzpicture}
\end{figure}
\end{example}
\end{frame}

\begin{frame}{$k$-intervalové funkce}
\todonote[inline]{Přirozeným zobecněním na množiny intervalů dostaneme víceintervalové funkce.}

\begin{definition}[$k$-intervalová funkce]
$f_{\interval{a_1}{b_1}, \ldots, \interval{a_k}{b_k}}(x) = 1 \iff a_i \leq x \leq b_i$ pro některé $i$
\end{definition}

\begin{example}[$2$-intervalová funkce]
$$f_{\interval{000}{001}, \interval{100}{110}}$$

\begin{figure}[h]
\centering
\begin{tikzpicture}[
    point/.style = {draw, circle, inner sep=2pt},
    truepoint/.style = {point, fill = black},
    trueinterval/.style = {line width = 2pt}
]

\pgfmathsetmacro{\nn}{3}

\pgfmathsetlengthmacro{\labeldist}{8pt}
\pgfmathsetlengthmacro{\spandist}{12pt}

\drawaxis{\nn}{\labeldist}{\textwidth}

\drawinterval{000}{001}{}
\drawinterval{100}{110}{}

\end{tikzpicture}
\end{figure}
\end{example}
\end{frame}

\begin{frame}{$l$-zlomové funkce}
\begin{definition}[$l$-zlomová funkce \citep{Husek2014}]
Funkce $f$ je $l$-zlomová,
pokud existuje právě $l$ vektorů $x$ takových,
že $f(x) \neq f(x+1)$.
\end{definition}

\begin{example}[$1$-zlomová funkce]
\begin{figure}[h]
\centering
\begin{tikzpicture}[
    point/.style = {draw, circle, inner sep=2pt},
    truepoint/.style = {point, fill = black},
    trueinterval/.style = {line width = 2pt},
    switchpoint/.style = {shape=coordinate},
    anchornode/.style = {shape=coordinate}
]

\pgfmathsetmacro{\nn}{3}

\pgfmathsetlengthmacro{\labeldist}{8pt}
\pgfmathsetlengthmacro{\spandist}{12pt}
\pgfmathsetlengthmacro{\signaldiff}{16pt}

\drawaxis{\nn}{-16pt -\labeldist}{\textwidth}

\drawinterval{000}{100}{}

\draw [draw opacity = 0] (100) -- (101) node [midway, switchpoint, draw opacity = 1] (100-101) {};

\begin{scope}[nodes = {above=\spandist * 1}]
\node [anchornode, yshift=\signaldiff] at (000) (000-1) {};
\node [anchornode, yshift=\signaldiff] at (100-101) (100-101-1) {};
\node [anchornode] at (100-101) (100-101-0) {};
\node [anchornode] at (111) (111-0) {};
\end{scope}

\draw (000-1) -- (100-101-1) -- (100-101-0) -- (111-0);

\end{tikzpicture}
\end{figure}
\end{example}

\begin{example}[$3$-zlomová funkce]
%$$f_{\interval{000}{001}, \interval{100}{110}}$$
\begin{figure}[h]
\centering
\begin{tikzpicture}[
    point/.style = {draw, circle, inner sep=2pt},
    truepoint/.style = {point, fill = black},
    trueinterval/.style = {line width = 2pt},
    %switchpoint/.style = {point, rectangle, fill = none, inner sep = 4pt}
    switchpoint/.style = {shape=coordinate},
    anchornode/.style = {shape=coordinate}
]

\pgfmathsetmacro{\nn}{3}

\pgfmathsetlengthmacro{\labeldist}{8pt}
\pgfmathsetlengthmacro{\spandist}{12pt}
\pgfmathsetlengthmacro{\signaldiff}{16pt}

\drawaxis{\nn}{-16pt -\labeldist}{\textwidth}

\drawinterval{000}{001}{}
\drawinterval{100}{110}{}

\draw [draw opacity = 0] (001) -- (010) node [midway, switchpoint, draw opacity = 1] (001-010) {};
\draw [draw opacity = 0] (011) -- (100) node [midway, switchpoint, draw opacity = 1] (011-100) {};
\draw [draw opacity = 0] (110) -- (111) node [midway, switchpoint, draw opacity = 1] (110-111) {};

\begin{scope}[nodes = {above=\spandist * 1}]
\node [anchornode, yshift=\signaldiff] at (000) (000-1) {};

\node [anchornode, yshift=\signaldiff] at (001-010) (001-010-1) {};
\node [anchornode] at (001-010) (001-010-0) {};

\node [anchornode] at (011-100) (011-100-0) {};
\node [anchornode, yshift=\signaldiff] at (011-100) (011-100-1) {};

\node [anchornode, yshift=\signaldiff] at (110-111) (110-111-1) {};
\node [anchornode] at (110-111) (110-111-0) {};

\node [anchornode] at (111) (111-0) {};
\end{scope}

\draw (000-1) -- (001-010-1) -- (001-010-0) -- (011-100-0) -- (011-100-1) -- (110-111-1) -- (110-111-0) -- (111-0);

\end{tikzpicture}
\end{figure}
\end{example}

\todonote[inline]{Jde dobře vidět, že počet zlomů je těsně svázaný s počtem intervalů.}
\todonote[inline]{$k$-intervalová funkce má $2k$, $2k-1$ nebo $2k-2$ zlomů.}
\end{frame}

\subsection{\Acrlong{dnf} a pokrytí}

\begin{frame}{\Acrlong{dnf} a ternární vektory}
\todonote[inline]{Je známým faktem, že každou booleovskou funkci lze reprezentovat formulí v disjunktivně normální formě.}

$$
\text{term} \sim \text{\definiendum{ternární vektor}, tj.~vektor nad abecedou $\curly{0, 1, \phi}$}
$$

\todonote[inline]{DNF term nad proměnnými $x_1, \ldots, x_n$ odpovídá ternárnímu vektoru délky $n$.}

\begin{example}[Term a ternární vektor]
$$
x_1 \nott{x_4} \sim 1 \phi \phi 0
$$
\end{example}

\todonote[inline]{Symbol na $i$-té pozici určuje přítomnost a polaritu proměnné $x_i$ v termu. $\phi$ figuruje jako žolík -- neomezuje hodnotu odpovídající proměnné.}
\todonote[inline]{Namísto \acrshort{dnf} formulí budeme uvažovat množiny ternárních vektorů. Říkáme jim pokrytí.}

\todonote[inline]{Ternární vektor je \emph{maska} binárních vektorů.}

\todonote[inline]{Odteď budeme hovořit o ternárních vektorech, ale budeme mít na paměti, že odpovídají DNF termům.}
\end{frame}

\begin{frame}{Pokrytí}
\begin{definition}[Pokrytí]
\definiendum{Pokrytí} funkce $f$ je množina ternárních vektorů ekvivalentní \acrshort{dnf} reprezentaci $f$.
\end{definition}

\begin{example}[Pokrytí]
$$
f_{\interval{001}{110}} \sim \curly{001, 01 \phi, 10 \phi, 110}
$$

\begin{figure}[h]
\centering
\begin{tikzpicture}[
    point/.style = {draw, circle, inner sep=2pt},
    truepoint/.style = {point, fill = black},
    trueinterval/.style = {line width = 2pt},
    spanpoint/.style = {point, fill = black, inner sep=1pt},
    spaninterval/.style = {line width = 1pt}
]

\pgfmathsetmacro{\nn}{3}

\pgfmathsetlengthmacro{\labeldist}{8pt}
\pgfmathsetlengthmacro{\spandist}{12pt}

\drawaxis{\nn}{\labeldist}{\textwidth}

\drawinterval{001}{110}{}

\begin{scope}[nodes = {below=\spandist}]
    \drawinterval{001}{001}{$001$};
    \drawinterval{010}{011}{$01 \phi$};
    \drawinterval{100}{101}{$10 \phi$};
    \drawinterval{110}{110}{$110$};
\end{scope}

\end{tikzpicture}
\end{figure}
\end{example}
\end{frame}

\subsection{Booleovská minimalizace}

\begin{frame}{Minimalizace pokrytí $k$-intervalové funkce}
\todonote[inline]{Minimální pokrytí odpovídá DNF formuli minimální v počtu termů. Funkce může mít více než jedno minimální pokrytí.}

\begin{problem} %[Minimalizace pokrytí $k$-intervalové funkce]
\begin{description}
\item[Vstup]
$n$-bitová čísla
$a_1, b_1, \ldots, a_k, b_k$

\item[Výstup]
Minimální pokrytí $k$-intervalové funkce $f_{\interval{a_1}{b_1}, \ldots, \interval{a_k}{b_k}}$
%\citet[sekce~3.3]{Crama2011}
\end{description}
\end{problem}

\begin{example}[Minimální pokrytí $2$-intervalové funkce]
$$
f_{\interval{001}{110}} \sim \curly{01 \phi, \phi 01, 1 \phi 0}
$$

\begin{figure}[h]
\centering
\begin{tikzpicture}[
    point/.style = {draw, circle, inner sep=2pt},
    truepoint/.style = {point, fill = black},
    trueinterval/.style = {line width = 2pt},
    spanpoint/.style = {point, fill = black, inner sep=1pt},
    spaninterval/.style = {line width = 1pt}
]

\pgfmathsetmacro{\nn}{3}

\pgfmathsetlengthmacro{\labeldist}{8pt}
\pgfmathsetlengthmacro{\spandist}{12pt}

\drawaxis{\nn}{\labeldist}{\textwidth}

\node [right=0.5, yshift=4pt] at (111) (A) {};

\drawinterval{001}{110}{}

% 01 \phi
\begin{scope}[nodes = {below=\spandist}]
    \drawinterval{010}{011}{};
    \node at (A) {$01 \phi$};
\end{scope}

% \phi 01
\begin{scope}[nodes = {below=\spandist * 2}]
    \drawinterval{001}{001}{};
    \drawinterval{101}{101}{};
    \node at (A) {$\phi 01$};
\end{scope}

% 1 \phi 0
\begin{scope}[nodes = {below=\spandist * 3}]
    \drawinterval{100}{100}{};
    \drawinterval{110}{110}{};
    \node at (A) {$1 \phi 0$};
\end{scope}

\end{tikzpicture}
\end{figure}
\end{example}
\end{frame}

\section{Výsledky}

\subsection{Teoretická východiska}

\begin{frame}{Teoretická východiska}

\todonote[inline]{Výsledky, ze kterých jsem vycházel}

\begin{itemize}
\item \citet{Schieber2005154}
\begin{itemize}
\item
Optimalizační algoritmus pro $1$-intervalové funkce
%\begin{itemize}
%\item Jednoduchý algoritmus pro prefixové a suffixové (tj.~$1$-zlomové) funkce
%\end{itemize}
\end{itemize}

\item \citet{Dubovsky2012}
\begin{itemize}
\item
Optimalizační algoritmus pro $2$-zlomové $2$-intervalové funkce

\item
$2$-aproximační algoritmus pro $2$-intervalové funkce

\item $3$-zlomová a $4$-zlomová funkce,
které nejsou coverable
\end{itemize}
\end{itemize}

\end{frame}

\subsection{Nové výsledky}

\begin{frame}{Nové výsledky v~diplomové práci}
\begin{itemize}
\item Zjednodušený optimalizační algoritmus pro $2$-zlomové $2$-intervalové funkce
\pause
\item Technika použitá v~důkazech optimality pokrytí $2$-zlomových funkcí není obecně použitelná pro $3$- a vícezlomové funkce
\pause
\item $2k$-aproximační algoritmus pro $k$-intervalové funkce pro každé $k \geq 0$
\pause
\item Přirozené vylepšení aproximačního algoritmu a důkaz, že není o moc lepší
\end{itemize}
\end{frame}

\begin{frame}{Zjednodušený algoritmus pro $2$-zlomové funkce}

$$
\text{$2$-zlomová $2$-intervalová funkce}
\mapsto \text{$1$-intervalová funkce}
$$

\todonote[inline]{Nebudu zacházet do detailů, aby zbylo víc prostoru na další (zajímavější) výsledky.}

\todomaybe[inline]{Vynechej příklad.}
\todomaybe[inline]{Zjednoduš příklad -- roztáhni $n$.}
\begin{example}[Transformace $2$-zlomové funkce na $1$-intervalovou]
\begin{figure}[h]
\centering
\begin{tikzpicture}[
    point/.style = {draw, circle, inner sep=2pt},
    truepoint/.style = {point, fill = black},
    trueinterval/.style = {line width = 2pt},
    spanpoint/.style = {point, fill = black, inner sep=1pt},
    spaninterval/.style = {line width = 1pt}
]

\pgfmathsetmacro{\nn}{3}

\pgfmathsetlengthmacro{\labeldist}{8pt}
\pgfmathsetlengthmacro{\spandist}{12pt}

\renewcommand{\tprefix}{A}
\drawaxis{\nn}{\labeldist}{\textwidth}

\drawinterval{A000}{A001}{}
\drawinterval{A101}{A111}{}

\renewcommand{\tprefix}{B}
\drawaxisoff{\nn}{-16pt - \labeldist}{\textwidth}{\spandist * 4}
\drawinterval{B001}{B101}{}
\drawinterval{B011}{B011}{}
\drawinterval{B100}{B100}{}

\draw (A001) -- (B101);
\draw (A000) -- (B100);

\draw (A101) -- (B001);
\draw (A111) -- (B011);

\renewcommand{\tprefix}{}

\end{tikzpicture}
\end{figure}
\end{example}

\end{frame}

\begin{frame}{Non-coverable $l$-zlomová funkce pro každé $l \geq 3$}

\todonote[inline]{Dosavadní důkazy optimality jsou založené na \emph{coverability} funkcí v~řešených třídách ($2$ a méně zlomů) a mimochodem ukazují, že všechny funkce v~těchto třídách jsou coverable.}

\begin{tabular}{ll}
\citet{Dubovsky2012}: & $3$-zlomová funkce  $f_{\interval{0}{4}, \interval{9}{14}}$ není coverable. \\
\citet{Bartek2015}: & $l$-zlomová non-coverable funkce pro každé $l \geq 3$
\end{tabular}

\begin{conclusion}
Optimalitu algoritmu pro třídu všech $k$-intervalových funkcí pro $k \geq 2$ nelze dokázat konstrukcí ortogonální množiny.
\end{conclusion}

\end{frame}

\begin{frame}{Jednoduchý $2k$-aproximační algoritmus}

\todonote[inline]{Abychom měli aspoň nějakou představu o velikosti minimálního pokrytí pro obecné $k$-intervalové funkce, navrhl jsem jednoduchý aproximační algoritmus.}
\todonote[inline]{Řešíme obecný problém pokrytí $k$-intervalové funkce.}

\begin{algorithm}[Suffix-prefix dekompozice]
Každý interval rozdělíme na dva podintervaly (suffixový a prefixový).
Každý z~těchto podintervalů pokryjeme zvlášť optimálně.
\end{algorithm}

Aproximační poměr: $2k$

\begin{example}[Suffix-prefix dekompozice]
\begin{figure}
\centering
\begin{tikzpicture}[
    point/.style = {draw, circle, inner sep=2pt},
    truepoint/.style = {point, fill = black},
    trueinterval/.style = {line width = 2pt},
    ortho/.style = {point, rectangle, fill = none, inner sep = 6pt}
]

\pgfmathsetmacro{\nn}{4}

\pgfmathsetlengthmacro{\labeldist}{8pt}
\pgfmathsetlengthmacro{\spandist}{12pt}

\drawaxis{\nn}{\labeldist}{\textwidth}

\node [above=\labeldist] at (0000) {$\textcolor{gray}{0}\underline{0}00$};
\node [above=\labeldist] at (0100) {$\textcolor{gray}{0}\underline{1}00$};
\node [above=\labeldist] at (1001) {$\textcolor{gray}{1}\underline{0}01$};
\node [above=\labeldist] at (1110) {$\textcolor{gray}{1}\underline{1}10$};

%\node [ortho] at (1001) {};
%\node [ortho] at (1010) {};

\drawinterval{0000}{0100}{}
\drawinterval{1001}{1110}{}

\drawaxisoff{\nn}{\labeldist}{\textwidth}{\spandist * 3}

\drawinterval{0000}{0011}{};
\drawinterval{0100}{0100}{};

\drawinterval{1001}{1011}{};
\drawinterval{1100}{1110}{};

\node [above=\labeldist, xshift=4pt] at (0000) {$\textcolor{gray}{00}00$};
\node [above=\labeldist, xshift=-4pt] at (0011) {$\textcolor{gray}{00}11$};
\node [above=\labeldist] at (0100) {$\textcolor{gray}{0100}$};

\node [above=\labeldist, xshift=4pt] at (1001) {$\textcolor{gray}{10}01$};
\node [above=\labeldist, xshift=-4pt] at (1011) {$\textcolor{gray}{10}11$};
\node [above=\labeldist, xshift=4pt] at (1100) {$\textcolor{gray}{11}00$};
\node [above=\labeldist, xshift=-4pt] at (1110) {$\textcolor{gray}{11}10$};

\begin{scope}[nodes = {below=\spandist * 2}]
    \drawinterval{0000}{0011}{$00 \phi \phi$};
    \drawinterval{0100}{0100}{$0100$};
    
    \drawinterval{1001}{1001}{$1001$};
    \drawinterval{1010}{1011}{$101 \phi$};
    \drawinterval{1100}{1101}{$110 \phi$};
    \drawinterval{1110}{1110}{$1110$};
\end{scope}

\end{tikzpicture}
\end{figure}
\end{example}

\todonote[inline]{Tento způsob dekompozice je dobrý v tom, že umožňuje sestrojit ortogonální množinu, která dává horní odhad $2k$ na aproximační poměr.}

\end{frame}

\begin{frame}{Vylepšený $2k$-aproximační algoritmus}

\begin{algorithm}[Intervalová dekompozice]
Každý interval pokryjeme zvlášť optimálně.
\end{algorithm}

Aproximační poměr: mezi $2k-2$ a $2k$

\begin{example}[Intervalová dekompozice]
\begin{figure}
\centering
\begin{tikzpicture}[
    point/.style = {draw, circle, inner sep=2pt},
    truepoint/.style = {point, fill = black},
    trueinterval/.style = {line width = 2pt}
]

\pgfmathsetmacro{\nn}{4}

\pgfmathsetlengthmacro{\labeldist}{8pt}
\pgfmathsetlengthmacro{\spandist}{12pt}

\drawaxis{\nn}{\labeldist}{\textwidth - 1cm}

\node [right=0.5, yshift=4pt] at (1111) (A) {};

\node [above=\labeldist] at (0000) {$0000$};
\node [above=\labeldist] at (0100) {$0100$};
\node [above=\labeldist] at (1001) {$1001$};
\node [above=\labeldist] at (1110) {$1110$};

\drawinterval{0000}{0100}{}
\drawinterval{1001}{1110}{}

\begin{scope}[nodes = {below=\spandist}]
    \drawinterval{0000}{0011}{$00 \phi \phi$};
    \drawinterval{0100}{0100}{$0100$};
\end{scope}

\begin{scope}[nodes = {below=\spandist * 2}]
    \drawinterval{1010}{1011}{};
    \node at (A) {$101 \phi$};
\end{scope}

\begin{scope}[nodes = {below=\spandist * 3}]
    \drawinterval{1001}{1001}{};
    \drawinterval{1101}{1101}{};
    \node at (A) {$1 \phi 01$};
\end{scope}

\begin{scope}[nodes = {below=\spandist * 4}]
    \drawinterval{1100}{1100}{};
    \drawinterval{1110}{1110}{};
    \node at (A) {$11 \phi 0$};
\end{scope}

\end{tikzpicture}
\end{figure}
\end{example}

\end{frame}

\section{Závěr}

\begin{frame}{Shrnutí}

  % Keep the summary *very short*.
  \begin{itemize}
  \item
    Dosud používaná technika důkazu optimality pokrytí selhává u $3$- a vícezlomových funkcí.
  \pause
  \item
    Ortogonální množiny však stále mohou dávat horní odhad aproximačního poměru.
  \pause
  \item
    Dekompozice na intervaly se pro velké $k$ nechová o mnoho lépe než $2k$-aproximačně.
  \end{itemize}
  \pause
  % The following outlook is optional.
  \vskip0pt plus.5fill
  \begin{itemize}
  \item
    Do budoucna:
    \begin{itemize}
    \item
      Efektivní optimalizační algoritmus pro třídu všech $k$-intervalových funkcí pro nějaké $k \geq 2$
    \item
      Aproximační algoritmus pro obecné $k$ s~aproximačním poměrem lepším než $2k-2$
    \end{itemize}
  \end{itemize}
  \pause
  \begin{center}
  \alert{Děkuji za Vaši pozornost.}
  \end{center}
\end{frame}

\appendix

\section{Záložní slidy}

\frame{\sectionpage}

\begin{frame}{Výpočetní složitost algoritmů v~\acrshort{dp}}
Výpočetní složitost všech uvedených algoritmů je \emph{polynomiální} vzhledem k~$n$ i $k$.
\todonote[inline]{Nic víc než polynomialita mne nezajímalo, protože problém je pro obecné $k$ NP-těžký.}

\begin{itemize}
\item Procedury:
\begin{itemize}
\item Prefix: $T_{prefix}(n) \in O(n^2)$
%\citep[Theorem 1]{Schieber2005154}
\item 1 interval: $T_{INT(1)}(n) \in O(n^3)$
\begin{itemize}
\item
Algoritmus redukuje na jednu menší instanci + polynomiální processing:
$$T(n) \leq T(n-1) + O(n^2) \leq n O(n^2) + O(1) \in O(n^{3})$$
\end{itemize}
\end{itemize}
\item Vlastní algoritmy:
\begin{itemize}
\item 2-zlomové 2-intervalové funkce: $T_{SWITCH(2)}(n) \in O(n^3)$
\item Suffix-prefix dekompozice: $T_{SPD}(n, k) \in O(k T_{prefix}(n)) = O(kn^2)$
\item Intervalová dekompozice: $T_{ID}(n, k) \in O(k T_{INT(1)}(n)) = O(k n^3)$
\end{itemize}
\end{itemize}

\todomaybe[inline]{Ukaž přesnější odhady, zejm.~pro $T_{INT(1)}$ podrobněji rozeber Case 4.}
\end{frame}

\todonote[inline]{Ortogonální množiny jsem ze \citep{Schieber2005154} nevytáhl, protože jejich popis je poněkud komplikovaný (nicméně na stranu by se asi vešel, samozřejmě bez důkazu ortogonality).}

\begin{frame}{Permutace souřadnic}
Permutaci souřadnic jsem zvážil pouze jako pre-processing \emph{mezí}, kde obecně nefunguje (nezachovává pokrytí).

\begin{example}[Permutace souřadnic v~mezích]
$a = 001, b = 110, \mathcal{T}_{\interval{a}{b}} = \curly{01 \phi, \phi 01, 1 \phi 0}$

$\pi = (2,3)$

$\pi(a) = 010, \pi(b) = 101,
\mathcal{T}_{\interval{\pi(a)}{\pi(b)}} = \curly{01 \phi, 10 \phi},
\pi^{-1}(\mathcal{T}_{\interval{\pi(a)}{\pi(b)}}) = \curly{0 \phi 1, 1 \phi 0}$
\end{example}

Permutace souřadnic by šlo použít i pro post-processing výstupu aproximačního algoritmu a je to zajímavý nápad otevřený dalšímu zkoumání.

\todonote[inline]{Vskutku, pro malá $n$ a $l \geq 2$ ($k \geq 3$) pro každou $f_l^n$ z DP existuje permutace, která $f_l^n$ změní na $2$-intervalovou (zjištěno hrubou silou).}

\end{frame}

\begin{frame}{Suffix-prefix dekompozice -- zdůvodnění}

%The algorithm we will
%describe is a straightforward extension of Schieber et al.’s suffix-prefix approx-
%imation algorithm for disjoint spanning sets of 1-interval functions

\todonote[inline]{Algoritmus 4.1 (suffix-prefix dekompozice) je inspirovaný algoritmem z~\citet{Schieber2005154}.
Nejde o zcela přímočaré rozšíření.}

Proč jsem nezobecnil přímo algoritmus z~\citet{Schieber2005154}?

\begin{itemize}
\item Jednodušší analýza aproximačního poměru
\item Aproximační poměr je nezměněn ($2k$)
\item Algoritmus \emph{intervalová dekompozice} (uvedený v~následující kapitole) je lepší
\item Algoritmus \emph{suffix-prefix dekompozice} především dává horní odhad $2k$ na aproximační poměr algoritmu \emph{intervalová dekompozice}
\end{itemize}
\todonote[inline]{Proto jsem se nechtěl moc zabývat jeho analýzou.}

\end{frame}

\begin{frame}{Suffix-prefix dekompozice -- zlepšený algoritmus}

\begin{algorithm}[Zlepšená suffix-prefix dekompozice]
Každý interval pokryjeme zvlášť.
\emph{Pokud je interval po odstranění společného prefixu mezí prefixový ($a = \rep{0}{m}$) nebo suffixový ($b = \rep{1}{m}$), pokryjeme jej optimálně.}
Jinak jej rozdělíme na suffixový a prefixový interval a každý z~nich pokryjeme optimálně.
\end{algorithm}

\todonote[inline]{Důkaz horního odhadu $2k$ aproximačního poměru zůstává stejný.}

Dolní odhad aproximačního poměru $2k$ dostaneme pomocí množiny \uv{zlých} funkcí:

$$
f_l^n \sim \curly{\interval{p \rep{0}{n-l-1} 1}{p \rep{1}{n-l-1} 0} | p \in \booldom^l}
$$
%($l \geq 0$, $n \geq l + 2$)

Počet intervalů $f_l^n$: $k = 2^l$

Velikost optima: $n-l$.

Velikost výstupu algoritmu: $2k (n-l-1)$.

\begin{conclusion}
Proti každému aproximačnímu poměru menšímu než $2k$ (pro $k = 2^l$) existuje protipříklad.
\end{conclusion}
\end{frame}

\section{Bibliografie}

\begin{frame}{Bibliografie}
\bibliography{ibf}
\end{frame}

\end{document}
