% https://d3s.mff.cuni.cz/~ceres/adm/jak-obhajit-diplomovou-praci.php

% $Header: /Users/joseph/Documents/LaTeX/beamer/solutions/generic-talks/generic-ornate-15min-45min.en.tex,v 90e850259b8b 2007/01/28 20:48:30 tantau $

\documentclass{beamer}

% This file is a solution template for:

% - Giving a talk on some subject.
% - The talk is between 15min and 45min long.
% - Style is ornate.



% Copyright 2004 by Till Tantau <tantau@users.sourceforge.net>.
%
% In principle, this file can be redistributed and/or modified under
% the terms of the GNU Public License, version 2.
%
% However, this file is supposed to be a template to be modified
% for your own needs. For this reason, if you use this file as a
% template and not specifically distribute it as part of a another
% package/program, I grant the extra permission to freely copy and
% modify this file as you see fit and even to delete this copyright
% notice.

\usepackage[utf8]{inputenc}

\mode<presentation>
{
% https://www.hartwork.org/beamer-theme-matrix/

  %\usetheme{Warsaw}
  \usetheme{Boadilla}

  %\setbeamercovered{transparent}
  % or whatever (possibly just delete it)
}

\usepackage[english,czech]{babel}

%\usepackage{times} % Change font

\usepackage{natbib}
\bibliographystyle{plainnat}

% Local packages
\usepackage{defense}
\usepackage{ibf-commands}
\usepackage{glossary-cs}
\usepackage{beamer-cs}
\usepackage{ibf-todonotes}

\theoremstyle{remark}
\newtheorem{observation}{Pozorování}
\newtheorem{algorithm}{Algoritmus}
\newtheorem{conclusion}{Závěr}

\begin{document}

\begin{frame}
\titlepage
\end{frame}

%\frame{\tableofcontents}

% - Exactly two or three sections (other than the summary).
% - At *most* three subsections per section.
% - Talk about 30s to 2min per frame. So there should be between about
%   15 and 30 frames, all told.

\begin{frame}{Hlavní problém}

\begin{problem}[Minimalizace $k$-intervalové funkce]
\begin{description}
\item[Vstup]
Meze $k$ celočíselných intervalů

\item[Výstup]
\acrshort{dnf} reprezentace s~nejmenším počtem termů
%\citet[sekce~3.3]{Crama2011}
\end{description}
\end{problem}

Meze intervalů jsou $n$-bitová celá čísla.
Meze $i$-tého intervalu značíme $a_i, b_i$.

Termy \acrshort{dnf} reprezentujeme vektory nad abecedou $\curly{0, 1, \phi}$.
\todo[inline]{Definuj pokrytí.}

\begin{example}
\begin{description}
\item[Vstup] $a_1 = 001, b_1 = 110$

($k=1$, $n=3$)

\todo[inline]{Vysviť postupně ternární vektory spolu s odpovídajícími true pointy.}

\item[True pointy]
$001, 010, 011, 100, 101, 110$

\item[Výstup]
Pokrytí $\curly{01 \phi, \phi 01, 1 \phi 0}$

(odpovídá \acrshort{dnf}
$\nott{x_1} x_2
\wedge
\nott{x_2} x_3
\wedge \nott{x_3} x_1$)
\end{description}
\end{example}

\end{frame}

\begin{frame}

\begin{example}
\begin{description}
\item[Vstup] $a_1 = 001, b_1 = 110$

($k=1$, $n=3$)

\todo[inline]{Vysviť postupně ternární vektory spolu s odpovídajícími true pointy.}

\item[True pointy]
$001, 010, 011, 100, 101, 110$

\item[Výstup]
Pokrytí $\curly{01 \phi, \phi 01, 1 \phi 0}$

(odpovídá \acrshort{dnf}
$\nott{x_1} x_2
\wedge
\nott{x_2} x_3
\wedge \nott{x_3} x_1$)
\end{description}
\end{example}

\begin{figure}[h]
\centering
% http://www.texample.net/tikz/examples/coin-flipping/
\newcommand{\figwidth}{\textwidth}
\newcommand{\leveldistance}{8pt}
\newcommand{\tpdistance}{16pt}
\newcommand{\tphalfheight}{6pt}
\newcommand{\tpdescwidth}{32pt}

\begin{tikzpicture}[
    %scale = 1,
    %transform shape,
    %thick,
    %grow = down,  % alignment of characters
    level 1/.style = {sibling distance=\figwidth / 2},
    level 2/.style = {sibling distance=\figwidth / 4},
    level 3/.style = {sibling distance=\figwidth / 8},
    level distance = \leveldistance,
    inner/.style = {draw, circle, inner sep=2pt},
    leaf/.style = {inner, rectangle},
    root/.style = {inner, minimum size=8pt},
    point/.style = {draw, circle, inner sep=2pt},
    truepoint/.style = {point, fill = black},
    falsepoint/.style = {point, fill = white},
    trueinterval/.style = {line width = 2pt},
    tpdesc/.style = {text width = \tpdescwidth}
  ]
  \node[root] (eps) {}
   child { node[inner] (0) {}
     child { node[inner] (00) {}
       child { node[leaf] (000) {}}
       child { node[leaf] (001) {}}
     }
     child { node[inner] (01) {}
       child { node[leaf] (010) {}}
       child { node[leaf] (011) {}}
     }
   }
   child {   node[inner] (1) {}
     child { node[inner] (10) {}
       child { node[leaf] (100) {}}
       child { node[leaf] (101) {}}
     }
     child { node[inner] (11) {}
       child { node[leaf] (110) {}}
       child { node[leaf] (111) {}
         child [grow=right, level distance=32pt] {node (A) {} edge from parent[draw=none]}
       }
     }
   };

  \begin{scope}[nodes = {draw = none}]
    \begin{scope}[nodes = {below = \tpdistance}]
      \node [name = a, truepoint] at (001) {};
      \node [name = b, truepoint] at (110) {};
      \draw [trueinterval] (a) -- (b) node {};
    \end{scope}
    \node [tpdesc, below=\tpdistance - \tphalfheight] at (A) {$\interval{a}{b}$};
    \begin{scope}[nodes = {below = \tpdistance * 2}]
      \node [name = 01p010, truepoint] at (010) {};
      \node [name = 01p011, truepoint] at (011) {};
      \draw [trueinterval] (01p010) -- (01p011) node {};
    \end{scope}
    \node [tpdesc, below=\tpdistance * 2 - \tphalfheight] at (A) {$01 \phi$};
    \begin{scope}[nodes = {below = \tpdistance * 3}]
      \node [truepoint] at (001) {};
      \node [truepoint] at (101) {};
    \end{scope}
    \node [tpdesc, below=\tpdistance * 3 - \tphalfheight] at (A) {$\phi 01$};
    \begin{scope}[nodes = {below = \tpdistance * 4}]
      \node [truepoint] at (100) {};
      \node [truepoint] at (110) {};
    \end{scope}
    \node [tpdesc, below=\tpdistance * 4 - \tphalfheight] at (A) {$1 \phi 0$};
  \end{scope}
\end{tikzpicture}
%\label{fig:001100}
\end{figure}

\end{frame}

\begin{frame}{$l$-zlomové funkce}
\begin{definition}[$l$-zlomová funkce]
Funkce $f$ je $l$-zlomová,
pokud existuje právě $l$ vektorů $x$ takových,
že $f(x) \neq f(x+1)$.
\end{definition}

\begin{example}
Funkce definovaná intervalem $\interval{001}{110}$ má dva zlomy: $000$ a $110$.

\todo[inline]{Doplň obrázek.}
\end{example}

\begin{observation}
$k$-intervalová funkce má $2k$, $2k-1$ nebo $2k-2$ zlomů.
\end{observation}
\end{frame}

\begin{frame}{Stávající výsledky}
\begin{itemize}
\item
Optimalizační algoritmus pro $1$-intervalové funkce \citep{Schieber2005154}
\begin{itemize}
\item Používám jako proceduru
\end{itemize}

\item
Optimalizační algoritmus pro $2$-zlomové $2$-intervalové funkce \citep{Dubovsky2012}

\item
$2$-aproximační algoritmus pro $2$-intervalové funkce \citep{Dubovsky2012}

\item Existují $3$-zlomová a $4$-zlomová funkce,
které nejsou coverable \citep{Dubovsky2012}
\begin{itemize}
\item Používám jako výchozí protipříklad
\end{itemize}
\end{itemize}
\end{frame}

\begin{frame}{Nové výsledky v~diplomové práci}
\begin{itemize}
\item Zjednodušený optimalizační algoritmus pro $2$-zlomové $2$-intervalové funkce
\item Důkaz, že pro libovolné $l \geq 3$ není třída $l$-zlomových funkcí coverable
\item $2k$-aproximační algoritmus pro $k$-intervalové funkce pro každé $k \geq 0$
\item Důkaz, že přirozené vylepšení aproximačního algoritmu moc nepomůže
\end{itemize}
\end{frame}

\begin{frame}{Zjednodušený algoritmus pro $2$-zlomové funkce}
Instanci $2$-zlomové $2$-intervalové funkce lze převést na instanci $1$-intervalové funkce.

\todo[inline]{Přidej obrázek.}

\begin{example}
\begin{description}
\item[Vstup]
$a_1 = 000, b_1 = 010, a_2 = 101, b_2 = 111$

($\bit{b_1}{1} = 0, \bit{a_2}{1} = 1$)
\end{description}

Převrácením prvního bitu mezí dostaneme instanci $1$-intervalové funkce:

\begin{align*}
a = \compl{\bit{b_1}{1}} \bits{b_1}{2}{3} = 001 \\
b = \compl{\bit{a_2}{1}} \bits{a_2}{2}{3} = 110
\end{align*}

Převrácením prvního symbolu v pokrytí $\interval{a}{b}$ dostaneme pokrytí původní dvojice intervalů
\end{example}

\end{frame}

\begin{frame}{Pro libovolné $l \geq 3$ není třída $l$-zlomových funkcí coverable}
Důkazy optimality optimalizačních algoritmů jsou založené na konstrukci ortogonálních množin.

\begin{definition}[Ortogonální množina, coverability]
\definiendum{Ortogonální množina} funkce $f$ obsahuje true pointy $f$,
z nichž žádnou dvojici nelze pokrýt jedním ternárním vektorem.

Funkce je \definiendum{coverable},
pokud má nějakou ortogonální množinu a pokrytí, které jsou stejně velké.
\end{definition}

\citet{Dubovsky2012} ukázal příklad $3$-zlomové funkce,
která není coverable.

Tento příklad jsem rozšířil na $l$-zlomovou funkci pro libovolné $l \geq 3$.

\begin{conclusion}
Optimalitu algoritmu pro třídu všech $k$-intervalových funkcí pro $k \geq 2$ nelze dokázat konstrukcí ortogonální množiny.
\end{conclusion}
\end{frame}

\begin{frame}{Jednoduchý $2k$-aproximační algoritmus}
\begin{algorithm}[Suffix-prefix dekompozice]
Každý interval rozdělíme podle na dva \uv{hezké} podintervaly.
Každý z~těchto podintervalů pokryjeme zvlášť optimálně.
\end{algorithm}

Sjednocení takovýchto pokrytí je v~nejhorším případě $2k$-krát větší než optimální pokrytí.

\begin{example}
\begin{description}
\item[Vstup]
$a_1 = 000, b_1 = 001, a_2 = 100, b_2 = 101$

\item[Výstup]
$\curly{000, 001, 100, 101}$

\item[Optimum]
$\curly{\phi 0 \phi}$
\end{description}
\end{example}

Horní odhad aproximačního poměru jsem ukázal konstrukcí dostatečně velké ortogonální množiny.
\end{frame}

\begin{frame}{Vylepšený $2k$-aproximační algoritmus}
\begin{algorithm}[Intervalová dekompozice]
Každý interval pokryjeme zvlášť optimálně.
\end{algorithm}

Tento algoritmus má aproximační poměr mezi $2k-2$ a $2k$.

\begin{example}
\begin{description}
\item[Vstup]
$a_1 = 000, b_1 = 001, a_2 = 100, b_2 = 101$

\item[Výstup]
$\curly{00 \phi, 10 \phi}$

\item[Optimum]
$\curly{\phi 0 \phi}$
\end{description}
\end{example}

Dolní odhad aproximačního poměru jsem ukázal konstrukcí třídy \uv{zlých} funkcí.
\end{frame}

\todo[inline]{Ukaž tabulku dosud dosažených aproximačních poměrů.}

\begin{frame}{Shrnutí}

  % Keep the summary *very short*.
  \begin{itemize}
  \item
    Některé $l$-zlomové funkce pro $l \geq 3$ nejsou coverable.
  \item
    Ortogonální množiny však stále mohou dát horní odhad aproximačního poměru.
  \item
    Dekompozice na intervaly se pro velké $k$ nechová o mnoho lépe než $2k$-aproximačně.
  \end{itemize}
  
  % The following outlook is optional.
  \vskip0pt plus.5fill
  \begin{itemize}
  \item
    Stále chybí:
    \begin{itemize}
    \item
      Efektivní optimalizační algoritmus pro třídu všech $k$-intervalových funkcí pro nějaké $k \geq 2$
    \item
      Aproximační algoritmus pro obecné $k$ s~aproximačním poměrem lepším než $2k-2$
    \end{itemize}
  \end{itemize}
\end{frame}

\begin{frame}{Bibliografie}

\bibliography{ibf}

\end{frame}

\begin{frame}{Výpočetní složitost algoritmů v~\acrshort{dp}}
Výpočetní složitost všech uvedených algoritmů je \emph{polynomiální} vzhledem k~$n$ i $k$.
\todonote[inline]{Nic víc mne nezajímalo.}

\begin{itemize}
\item Procedury:
\begin{itemize}
\item Prefix: $T_{prefix}(n) \in O(n^2)$
%\citep[Theorem 1]{Schieber2005154}
\item 1 interval: $T_{INT(1)}(n) \in O(n^3)$
\begin{itemize}
\item
Algoritmus je typu \emph{decrease and conquer} (redukce na jednu menší instanci + polynomiální processing):
$$T(n) \leq T(n-1) + O(n^l) \leq n O(n^l) + c \in O(n^{l+1})$$
\end{itemize}
\end{itemize}
\item Vlastní algoritmy:
\begin{itemize}
\item 2-zlomové 2-intervalové funkce: $T_{SWITCH(2)}(n) \in O(n^3)$
\item Suffix-prefix dekompozice: $T_{SPD}(n, k) \in O(k T_{prefix}(n)) = O(kn^2)$
\item Intervalová dekompozice: $T_{ID}(n, k) \in O(k T_{INT(1)}(n)) = O(k n^3)$
\end{itemize}
\end{itemize}

\todomaybe[inline]{Ukaž přesnější odhady, zejm.~pro $T_{INT(1)}$ podrobněji rozeber Case 4.}
\end{frame}

\todonote[inline]{Ortogonální množiny jsem ze \citep{Schieber2005154} nevytáhl, protože jejich popis je poněkud komplikovaný (nicméně na stranu by se asi vešel, samozřejmě bez důkazu ortogonality).}

\begin{frame}{Permutace souřadnic}
Permutaci souřadnic jsem zvážil pouze jako pre-processing mezí, kde obecně nefunguje (nezachovává pokrytí).

\begin{example}[Permutace souřadnic v~mezích]
$a = 001, b = 110, \mathcal{T}_{\interval{a}{b}} = \curly{01 \phi, \phi 01, 1 \phi 0}$

$\pi = (2,3)$

$\pi(a) = 010, \pi(b) = 010,
\mathcal{T}_{\interval{\pi(a)}{\pi(b)}} = \curly{01 \phi, 10 \phi},
\pi(\mathcal{T}_{\interval{\pi(a)}{\pi(b)}}) = \curly{0 \phi 1, 1 \phi 0}$
\end{example}

Permutace souřadnic by šlo použít i pro post-processing v~aproximačním algoritmu a je to zajímavý nápad otevřený dalšímu zkoumání.

\todonote[inline]{Vskutku, pro malá $n$ a $l \geq 2$ ($k \geq 3$) pro každou $f_l^n$ z DP existuje permutace, která $f_l^n$ změní na $2$-intervalovou (zjištěno hrubou silou).}
\end{frame}

\begin{frame}{Suffix-prefix dekompozice}

Algoritmus 4.1 (suffix-prefix dekompozice) jsem navrhl takto špatně, abych zjednodušil analýzu aproximačního poměru.

Uvedené zlepšení aproximační poměr zlepší jen málo.

\begin{algorithm}[Zlepšená suffix-prefix dekompozice]
Každý interval pokryjeme zvlášť.
Pokud je interval po odstranění společného prefixu mezí prefixový ($a = \rep{0}{m}$) nebo suffixový ($b = \rep{1}{m}$), pokryjeme jej optimálně.
Jinak jej rozdělíme na suffixový a prefixový interval a každý z~nich pokryjeme optimálně.
\end{algorithm}

%Důkaz horního odhadu $2k$ aproximačního poměru zůstává stejný.

Dolní odhad aproximačního poměru $2k$ dostaneme pomocí množiny \uv{zlých} funkcí:

$$
f_l^n \sim \curly{\interval{p \rep{0}{n-l-1} 1}{p \rep{1}{n-l-1} 0} | p \in \booldom^l}
$$
%($l \geq 0$, $n \geq l + 2$)

Počet intervalů $f_l^n$: $k = 2^l$

Velikost optima: $n-l$.

Velikost výstupu algoritmu: $2k (n-l-1)$.

\begin{conclusion}
Proti každému aproximačnímu poměru menšímu než $2k$ (pro $k = 2^l$) existuje protipříklad.
\end{conclusion}
\end{frame}

\end{document}
