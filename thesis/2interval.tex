\chapter{\texorpdfstring{$2$}{2}-interval functions}

In this chapter we will revise the results specific to
$2$-interval functions.
\dubovsky{}'s
thesis \cite{Dubovsky2012}
provides more detailed information.

\dubovsky{} identified several classes
of $2$-interval functions \cite[p.~5]{Dubovsky2012}
which seem to require different approaches for spanning.
% TODO: Reword "seem to require".
I find \husek{}'s concept of
\quot{step} \cite[p.~13]{Husek2014}
(\quot{\foreignlanguage{czech}{zlom}} in Czech)
useful to differentiate between these classes.

\begin{definition}[$l$-step function]
A Boolean function $f$ is \definiendum{$l$-step}
if and only if
there are exactly $l$ input vectors $x$ such that
$f(x) \neq f(x+1)$.
\end{definition}

It is easy to see that
prefix and suffix interval functions are $1$-step.

Depending on $f(0)$,
an $l$-step Boolean function $f$
is $\ceil*{\frac{l}{2}}$-interval in case $f(0) = 0$,
% f(0) = 0:
% 1-step: 1-interval
% 2-step: 1-interval
% 3-step: 2-interval
% 4-step: 2-interval
% l-step: \ceil{l/2}-interval
or $\ceil*{\frac{l+1}{2}}$-interval in case $f(0) = 1$.
% f(0) = 1:
% 1-step: 1-interval
% 2-step: 2-interval
% 3-step: 2-interval
% 4-step: 3-interval
% l-step: \ceil{(l+1)/2}-interval

Namely,
a $2$-step Boolean function such that $f(0) = 0$
is $1$-interval.

\section{\texorpdfstring{$2$}{2}-step functions}

Let $f$ be a $2$-interval $2$-step Boolean function
(necessarily $f(\rep{0}{n}) = f(\rep{1}{n})
= 1$).\footnote{This corresponds to class $A_0$
in \dubovsky{}'s classification
\cite[p.~5]{Dubovsky2012}.}
We will show that $f$
can be optimally spanned by reduction to
a $2$-step $1$-interval Boolean function $f'$
of at most the same arity,
which can be spanned using the algorithm introduced
in chapter \ref{chap:1interval}.

Note that
\dubovsky{} introduced an optimization algorithm for
this class of functions \cite[section 3.2]{Dubovsky2012},
following an argument similar to the one used by
Schieber et al. to span $1$-interval functions
\cite{Schieber2005154}.
However,
the reduction approach gives
an interesting insight in the problem
and seems easier to describe.
% TODO: Consider rewording.

From the definition of $f$ we know that it is $2$-interval
with the outer interval endpoints being
$\rep{0}{n}$ and $\rep{1}{n}$:

\[
f = f^n_{\interval{\rep{0}{n}}{b_1},
\interval{a_2}{\rep{1}{n}}}
\]

Let's call the inner interval endpoints $b_1$ and $a_2$.
$\rep{0}{n} \leq b_1$,
$b_1 \leq a_2 - 2$,
$a_2 \leq \rep{1}{n}$.

\section{\texorpdfstring{$3$}{3}-step functions}

% TODO: Finish.
