\chapter{\texorpdfstring{$2$}{2}-interval functions}
\label{chap:2interval}

\todo[inline]{Add an example with pictures.}

In this chapter we will revise the results specific to
$2$-interval functions.
\citet{Dubovsky2012}
provides a more detailed overview.

\citeauthor{Dubovsky2012} identified several classes
of $2$-interval functions \citep[p.~5]{Dubovsky2012}
which seem to require different approaches to spanning.\todo{Reword \quot{seem to require}.}
It turns out that it is useful to consider
the number of switches
to differentiate between these classes.
Recall that an $l$-switch function has exactly $l$ input
vectors $x$ such that $\apply{f}{x} \neq \apply{f}{x+1}$.

It is easy to see that
prefix and suffix interval functions are $1$-switch.

Depending on $f(0)$,
an $l$-switch Boolean function $f$
is proper $\ceil*{\frac{l}{2}}$-interval
in case $f(0) = 0$,
% f(0) = 0:
% 1-switch: 1-interval
% 2-switch: 1-interval
% 3-switch: 2-interval
% 4-switch: 2-interval
% l-switch: \ceil{l/2}-interval
or proper $\ceil*{\frac{l+1}{2}}$-interval
in case $f(0) = 1$.
% f(0) = 1:
% 1-switch: 1-interval
% 2-switch: 2-interval
% 3-switch: 2-interval
% 4-switch: 3-interval
% l-switch: \ceil{(l+1)/2}-interval

Equivalently,
a proper $k$-interval $n$-ary function is:

\begin{itemize}
\item
$(2k)$-switch in case $f(\rep{0}{n}) = f(\rep{1}{n}) = 0$
\item
$(2k-1)$-switch in case $f(\rep{0}{n}) \neq f(\rep{1}{n})$
\item
$(2k-2)$-switch in case $f(\rep{0}{n}) = f(\rep{1}{n}) = 1$
\end{itemize}

Namely,
a $2$-switch Boolean function such that $f(0) = 0$
is $1$-interval,
and a $2$-switch function such that $f(0) = 1$
is $2$-interval.

\section{\texorpdfstring{$2$}{2}-switch
\texorpdfstring{$2$}{2}-interval functions}
\label{sec:2int2switch}

\newcommand{\ftwointtwoswitch}[4]
{#1^{#2}_{\interval{\rep{0}{#2}}{#3},
\interval{#4}{\rep{1}{#2}}}}

\newcommand{\fnba}{\ftwointtwoswitch{f}{n}{b_1}{a_2}}

Let $f$ be a $2$-switch $2$-interval Boolean function
(necessarily $f(\rep{0}{n}) = f(\rep{1}{n})
= 1$).\footnote{This definition corresponds to class $A_0$
in \citeauthor{Dubovsky2012}'s classification
\citep[p.~5]{Dubovsky2012}.}
\citeauthor{Dubovsky2012} described a polynomial minimization algorithm for this class of functions.
The algorithm in \citet[p.~17]{Dubovsky2012}
% pp.~17-30 (including proof of optimality)
% Section 3.2
follows the approach from \citet{Schieber2005154}
and relies on a quite complicated and technically involved case analysis.
Here we will show that we can find a minimum \acrshort{dnf} formula representing a $2$-switch $2$-interval function $f$ by reducing the function to a $1$-interval function $f'$.
\todomaybe{Use the name $f'$ in the algorithm.}
This approach is much simpler
\todomaybe{Change to \quot{simpler to describe}.
It is simpler because the complexity stays hidden in Zaks' algorithm.}
than the case analysis in \citet{Dubovsky2012}.

A $2$-switch $2$-interval function $f$ necessarily has $f(\rep{0}{n})=f(\rep{1}{n})=1$
and thus there are two numbers $b_1$ and $a_2$
so that $f = \fnba$.
We can assume that $b_1 < a_2-1$
as otherwise $f$ would be a tautology
(and thus $0$-switch).

Note that $f$ is a negation of a $2$-switch $1$-interval function.
We can also say that $f$ is a \emph{false point $1$-interval function}
as false points of $f$ form a single interval
$\interval{b_1+1}{a_2-1}$.

\subsection{\texorpdfstring{$\bit{b_1}{1} = \bit{a_2}{1}$}
{\textbit{b1}{1} = \textbit{a2}{1}}
}

\todo[inline]{Solve the whole common prefix at once.
Alg.:
1. Solve common prefix
2. Transform to 1-interval}

In case $b_1$ and $a_2$ have the same leading bit,
we recursively solve the $(n-1)$-bit
case
$\interval{\rep{0}{n-1}}{\bits{b_1}{2}{n}},
\interval{\bits{a_2}{2}{n}}{\rep{1}{n-1}}$,
and add an extra ternary vector to span the remainder
of the true points.

Without loss of generality,
let $\bit{b_1}{1} = \bit{a_2}{1} = 0$.
The complementary case is symmetric.

Let $b_1' = \bits{b_1}{2}{n}$
and $a_2' = \bits{a_2}{2}{n}$.
Let $\mathcal{T}_0'$ be an optimal spanning set of
$f' = \ftwointtwoswitch{f}{n-1}{b_1'}{a_2'}$.
Note that $f'$ is of the same type
(that is $2$-interval $2$-switch)
and has a lower arity,
so we can span it by recursion.
Let $\mathcal{T}_0$ be the $n$-bit set we get
by prepending the symbol $0$
to each vector from $\mathcal{T}_0'$
($\mathcal{T}_0 = \curly{0 T | T \in \mathcal{T}_0'}$).
Let $\mathcal{T}_1 = \curly{1 \rep{\phi}{n-1}}$.
Let $\mathcal{T} = \mathcal{T}_0 \cup \mathcal{T}_1$.
We claim that $\mathcal{T}$ spans exactly
$\fnba$
and that $\mathcal{T}$ is minimum such.

\begin{theorem}
$\mathcal{T}$ spans exactly
$\fnba$ (feasibility).
\end{theorem}

\begin{proof}
\todo[inline]{Add an introductory paragraph.}

\hfill % Break line to align the enumerate items
\begin{enumerate}
\item{$span(\mathcal{T}) \subseteq TP(f)$}

Let $c \in span(T)$, $T \in \mathcal{T}$.
We need to show that $c \leq b_1$ or $c \geq a_2$.

\begin{enumerate}
\item $\bit{c}{1} = 1$: Since $\bit{a_2}{1} = 0$,
necessarily $c \geq a_2$.

\item $\bit{c}{1} = 0$: Necessarily $T \in \mathcal{T}_0$.
Let $T' = \bits{T}{2}{n}$ and $c' = \bits{c}{2}{n}$.
Since $T$ spans $c$, $T'$ spans $c'$.
By construction of $\mathcal{T}_0$
necessarily $T' \in \mathcal{T}_0'$,
so $T'$ does not span any number
in $\interval{b_1' + 1}{a_2' - 1}$.
Thus $c' \leq b_1'$ or $c' \geq a_2'$.
Observe that prepending a $0$ bit
to $c'$, $b_1'$ and $a_2'$
preserves the inequalities:
\[
c' \leq b_1' \text{ or } c' \geq a_2'
\Rightarrow
0 c' \leq 0 b_1' \text{ or } 0 c' \geq 0 a_2'
\]

Since $0 c' = c$, $0 a' = a$ and $0 b' = b$,
we get $c \leq b_1$ or $c \geq a_2$.
% meaning f(c) = 1
\end{enumerate}

\item{$TP(f) \subseteq span(\mathcal{T})$}

Let $c \leq b_1$ or $c \geq a_2$.
We need a $T \in \mathcal{T}$ that spans $c$.

\begin{enumerate}
\item $\bit{c}{1} = 1$:
$1 \phi^{n-1} \in \mathcal{T}_1 \subseteq \mathcal{T}$
spans $c$.

\item $\bit{c}{1} = 0$:
Similarly to $b_1'$ and $a_2'$, let $c' = \bits{c}{2}{n}$.
Since $b_1$, $a_2$ and $c$ all start with a $0$ bit,
$c' \leq b_1'$ or $c' \geq a_2'$.
Since $\mathcal{T}_0'$ spans
$\ftwointtwoswitch{f}{n-1}{b_1'}{a_2'}$,
there is $T \in \mathcal{T}_0'$ that spans $c'$.
Prepending the symbol $0$ to both $T$ and $c'$
preserves spanning,
so $0 T \in \mathcal{T}_0 \subseteq \mathcal{T}$ spans $c$.
\end{enumerate}

\end{enumerate}
\end{proof}

\begin{theorem}
$\mathcal{T}$ is a minimum spanning set of
$\fnba$ (optimality).
\end{theorem}

\begin{proof}
Let $\mathcal{T}_{opt}$ be an optimal spanning set of
$\fnba$.
We'll show that
$\size{\mathcal{T}_{opt}} \geq \size{\mathcal{T}}$.

First recall that
$\mathcal{T}_0'$ is an optimal spanning set of
$\ftwointtwoswitch{f}{n-1}{b_1'}{a_2'}$
and
$\size{\mathcal{T}} = \size{\mathcal{T}_0'} + 1$.
Thus we need to show
$\size{\mathcal{T}_{opt}} \geq \size{\mathcal{T}_0'} + 1$

$\mathcal{T}_{opt}$ must span the true point
$\bar{b}_1 = 1 \bits{(b_1 + 1)}{2}{n}$
(since $a_2$ starts with a $0$,
all numbers that start with a $1$ must be true points).
We don't need more than $n$ bits for $\bar{b}_1$
because $b_1 < a_2 \leq 0 \rep{1}{n-1}$.
Let $T_{\bar{b}_1} \in \mathcal{T}_{opt}$
be a ternary vector that spans $\bar{b}_1$.
If $T_{\bar{b}_1}$ spanned
any vector that starts with a $0$,
then $T_{\bar{b}_1}$ would also span
the vector $b_1 + 1$.
However,
since $f$ is \emph{proper} $2$-interval,
$b_1 + 1$ is a false point ($b_1 < b_1 + 1 < a_2$).
So $T_{\bar{b}_1}$
can't span any vector that starts with a $0$.

To get a contradiction,
let's suppose that
$\size{\mathcal{T}_{opt}} \leq \size{\mathcal{T}_0'}$.
Since we need a dedicated vector
$T_{\bar{b}_1} \in \mathcal{T}_{opt}$ to span $\bar{b}_1$,
there are at most
$\size{\mathcal{T}_{opt}} - 1
\leq \size{\mathcal{T}_0'} - 1$
vectors in $\mathcal{T}_{opt}$ that span a number
that starts with $0$
(so these vectors start with $0$ or $\phi$).
Removing the first symbol of these vectors,
we get a spanning set of
$\ftwointtwoswitch{f}{n-1}{b_1'}{a_2'}$
of size at most $\size{\mathcal{T}_0'} - 1$,
which contradicts $\mathcal{T}_0'$'s optimality.

We conclude that we need at least
$\size{\mathcal{T}_0'} + 1$ vectors to span
$\fnba$,
proving the lower bound of
$\size{\mathcal{T}_{opt}}$
and the optimality of $\mathcal{T}$.
\end{proof}

\todo[inline]{Construct an orthogonal set explicitly.}

We have shown that in case $\bit{b_1}{1} = \bit{a_2}{1}$,
our algorithm produces a feasible and optimal spanning set.

\subsection{\texorpdfstring{$\bit{b_1}{1}
\neq \bit{a_2}{1}$}
{\textbit{b1}{1} \textneq{} \textbit{a2}{1}}
}

\todo[inline]{Use longer paragraphs.}

Let $\bit{b_1}{1} \neq \bit{a_2}{1}$.
Since $b_1 \leq a_2$,
necessarily $\bit{b_1}{1} = 0$ and $\bit{a_2}{1} = 1$.

By flipping the leading bit in each of the endpoints
$b_1$ and $a_2$,
we will reduce this case
to an instance of spanning a single true point interval
on $n$ bits.
% TODO: Rephrase.

%Let $a_1 = a - 1$ and $b_1 = b + 1$.

Let $b_1' = 1 \bits{b_1}{2}{n}$
and $a_2' = 0 \bits{a_2}{2}{n}$
(flipping the leading bits of $b_1$ and $a_2$).
Note that $a_2' \leq b_1'$.

Let $\mathcal{T}'$ be a minimum spanning set
of $\intervalfn{f}{n}{a_2'}{b_1'}$.
This set can be computed efficiently
using the method shown
in chapter \ref{chap:1interval}.

Let's flip all the non-$\phi$ leading symbols
of vectors in $\mathcal{T}'$,
forming $\mathcal{T}$.
Note that $\size{\mathcal{T}} = \size{\mathcal{T}'}$
and also that all the vectors in $\mathcal{T}'$
and $\mathcal{T}$ have the length $n$.

We claim that $\mathcal{T}$ is a minimum spanning set
of $\fnba$.

The argument is based on a pair of lemmas.

\begin{lemma}
\label{lemma:cflip}
Let $c$ be an $n$-bit binary vector
and $a, b$ be $(n-1)$-bit binary vectors.
Let $c' = \compl{\bit{c}{1}} \bits{c}{2}{n}$
(flipping the leading bit of $c$).
Then ($c \leq 0 b$ or $c \geq 1 a$)
if and only if
($c' \geq 0 a$ and $c' \leq 1 b$).
\end{lemma}

\begin{proof}
We shall distinguish two cases
based on the leading bit of $c$.
Since the cases are symmetric,
without loss of generality let $\bit{c}{1} = 0$.

\todo[inline]{Extend the paragraphs.}

Let $c = 0 \bits{c}{2}{n}$ and $c' = 1 \bits{c}{2}{n}$.

Since $\bit{c}{1} = 0$,
$c \ngeq 1 a$.
Thus ($c \leq 0 b$ or $c \geq 1 a$)
if and only if
$c \leq 0 b$.

Since $\bit{c}{1} = 0$,
$c \leq 0 b$
if and only if
$\bits{c}{2}{n} \leq b$.

Since $\bit{c'}{1} = 1$
and $\bits{c'}{2}{n} = \bits{c}{2}{n}$,
$\bits{c}{2}{n} \leq b$
if and only if
$c' \leq 1 b$.

From $\bit{c'}{1} = 1$ we immediately get $c' \geq 0 a$.

Chaining the equivalences together:
\begin{align*}
c \leq 0 b \text{ or }c \geq 1 a
&\iff c \leq 0 b \\
&\iff \bits{c}{2}{n} \leq b \\
&\iff c' \leq 1 b \\
&\iff c' \leq 1 b \text{ and } c' \geq 0 a
\end{align*}
\end{proof}

\begin{lemma}
\label{lemma:tflip}
Let $\mathcal{T}$ be a set of ternary vectors
and $\mathcal{T}'
= \curly{\compl{\bit{T}{1}} \bits{T}{2}{n}
|T \in \mathcal{T}}$.
Let $a, b$ be any $(n-1)$-bit binary vectors.
% such that $0 b \leq 1 a - 2$.
Then $\mathcal{T}$ spans $\ftwointtwoswitch{f}{n}{0 b}{1 a}$
if and only if
$\mathcal{T}'$ spans
$\intervalfn{f}{n}{0 a}{1 b}$.\footnote{Note that
the statement holds even in the single case that
the function is not proper $2$-interval,
that is $a = \rep{1}{n-1}$ and $b = \rep{0}{n-1}$.}
% TODO: Add "However, we won't use this case."
\end{lemma}

\begin{proof}
\hfill % Break line to align the enumerate items
% TODO: Prevent page break here.
\begin{enumerate}
\item
If $\mathcal{T}$ spans $\ftwointtwoswitch{f}{n}{0 b}{1 a}$,
then $\mathcal{T}'$ spans $\intervalfn{f}{n}{0 a}{1 b}$:

Let $\mathcal{T}$ span $\ftwointtwoswitch{f}{n}{0 b}{1 a}$.

Let $c'$ be an $n$-bit binary vector.
We need to show that
$\mathcal{T}'$ spans $c'$
if and only if
$c' \geq 0 a$ and $c' \leq 1 b$.

Let $c = \compl{\bit{c'}{1}} \bits{c'}{2}{n}$.
By definition of $\mathcal{T}'$,
$\mathcal{T}'$ spans $c'$
if and only if
$\mathcal{T}$ spans $c$.

Since $\mathcal{T}$ spans
$\ftwointtwoswitch{f}{n}{0 b}{1 a}$,
$\mathcal{T}$ spans $c$
if and only if
$c \leq 0 b$ or $c \geq 1 a$.

From Lemma \ref{lemma:cflip}
we get ($c \leq 0 b$ or $c \geq 1 a$)
if and only if
($c' \geq 0 a$ and $c' \leq 1 b$).

Chaining the equivalences together:
\begin{align*}
\mathcal{T}' \text{ spans } c' &\iff \mathcal{T} \text{ spans } c \\
&\iff c \leq 0 b \text{ or } c \geq 1 a \\
&\iff c' \geq 0 a \text{ and } c' \leq 1 b \\
&\iff \intervalfn{f}{n}{0 a}{1 b}(c') = 1
\end{align*}

\item
If $\mathcal{T}'$ spans $\intervalfn{f}{n}{0 a}{1 b}$,
then $\mathcal{T}$ spans $\ftwointtwoswitch{f}{n}{0 b}{1 a}$:

Let $\mathcal{T}'$ span $\intervalfn{f}{n}{0 a}{1 b}$.

Let $c$ be an $n$-bit binary vector.
We need to show that $\mathcal{T}$ spans $c$
if and only if
$c \leq 0 b$ or $c \geq 1 a$.

Let $c' = \compl{\bit{c}{1}} \bits{c}{2}{n}$.
By definition of $\mathcal{T}'$,
$\mathcal{T}$ spans $c$
if and only if
$\mathcal{T}'$ spans $c'$.

Since $\mathcal{T}'$ spans $\intervalfn{f}{n}{0 a}{1 b}$,
$\mathcal{T}'$ spans $c'$
if and only if
$c' \geq 0 a$ and $c' \leq 1 b$.

Using Lemma \ref{lemma:cflip},
we get
($c' \geq 0 a$ and $c' \leq 1 b$)
if and only if
($c \leq 0 b$ or $c \geq 1 a$).

Chaining the equivalences together:
\begin{align*}
\mathcal{T} \text{ spans } c
&\iff \mathcal{T}' \text{ spans } c' \\
&\iff c' \geq 0 a \text{ and } c' \leq 1 b \\
&\iff c \leq 0 b \text{ or } c \geq 1 a \\
&\iff \ftwointtwoswitch{f}{n}{0 b}{1 a}(c) = 1
\end{align*}

\end{enumerate}
\end{proof}

Lemma \ref{lemma:tflip} shows
a size preserving
bijection
between
the spanning sets of functions of form
$\ftwointtwoswitch{f}{n}{0 b}{1 a}$
and $\intervalfn{f}{n}{0 a}{1 b}$
for general $a$ and $b$.
Since $b_1$ starts with a $0$ ($b_1 = 0 b$),
$a_2$ starts with a $1$ ($a_2 = 1 a$)
and the set $\mathcal{T}$ was constructed
from an optimal spanning set $\mathcal{T}'$
of
$\intervalfn{f}{n}{0 \bits{a_2}{2}{n}}{1 \bits{b_1}{2}{n}}$
using the bijection
shown in \ref{lemma:tflip},
that is flipping the leading symbols
of all the ternary vectors,
both feasibility and optimality of $\mathcal{T}$
with respect to $\fnba$
follow from the feasibility and optimality
of $\mathcal{T}'$ with respect to
$\intervalfn{f}{n}{0 \bits{a_2}{2}{n}}{1 \bits{b_1}{2}{n}}$.

\section{Functions with \texorpdfstring{$3$}{3}
and more switches}
\label{sec:3switch}
\todomaybe[inline]{Separate in a chapter.}

The situation changes in $3$-switch functions.
In $1$-switch (prefix and suffix)
and $2$-switch ($1$-interval and $2$-interval
with extreme outer endpoints)
functions,
we managed to efficiently find minimum spanning sets.
The proofs of optimality of the spanning sets
depend on construction of sufficiently large
orthogonal sets.\footnote{While we have only shown
an orthogonal set explicitly
in the case of prefix interval (\autoref{sec:prefix}),
the other cases use orthogonal sets implicitly,
both in \citep{Schieber2005154}
and \autoref{sec:2int2switch}.}
\todo{Update footnote once we construct an orthogonal set explicitly in \autoref{sec:2int2switch}.}
The fact that this aproach succeeds
depends on the fact that all the functions we have considered so far ($1$- and $2$-switch)
are \emph{coverable}
(according to \cref{def:coverable}).

\citeauthor{Dubovsky2012} has showed that
\todomaybe{Use \quot{shown} instead of \quot{showed}.}
the $3$-switch function
$f = \twointfn{f}{4}{0}{4}{11}{14}$ is not coverable.
The size of a maximum orthogonal set of $f$ is $4$
while the size of a minimum spanning set of $f$ is $5$.
\citep[p.~32]{Dubovsky2012}
This shows that $3$-switch functions are not coverable in general.
\citeauthor{Dubovsky2012} proved these bounds
on sizes of orthogonal sets and spanning sets
by exhaustion using software tools.
We will generalize the result for functions
with larger number of switches.

\begin{lemma}
\label{lemma:noncoverableinduction}
If there is an $l$-switch function that is not coverable,
then there is also an $(l+1)$-switch function
that is not coverable.
\end{lemma}

\begin{proof}
Let $f$ be an $n$-ary $l$-switch function
that is not coverable.
We will construct an $(n+1)$-ary $(l+1)$-switch function $f'$
that is not coverable.
We shall distinguish two cases depending on the value of $f$ on $\rep{1}{n}$.

\begin{enumerate}
\item $\apply{f}{\rep{1}{n}} = 1$:

Let $f'$ be an $(n+1)$-ary function
defined in the following way:

\[
\apply{f'}{x'} =
\begin{cases}
\apply{f}{\bits{x'}{2}{n+1}} & \text{if } \bit{x'}{1} = 0 \\
0 & \text{if } \bit{x'}{1} = 1
\end{cases}
\]

Since $f$ is $l$-switch,
there are exactly $l$ vectors $x$ of length $n$
($x < \rep{1}{n}$)
such that $\apply{f}{x} \neq \apply{f}{x+1}$.
Prepending a $0$ to each of these vectors,
we get $l$ vectors $x'$ of length $n+1$
such that $\apply{f'}{x'} \neq \apply{f'}{x'+1}$.
The $(l+1)$-st \quot{switch} vector of $f'$
is $0 \rep{1}{n}$,
since $\apply{f'}{0 \rep{1}{n}} = 1$
and $\apply{f'}{1 \rep{0}{n}} = 0$.
\todo{Prove that $f'$ doesn't have more switch vectors.}
Thus we have shown that $f'$ is $(l+1)$-switch.

There is
a size preserving
\todokucera{1-1 correspondence je bijekce a z toho plyne, že je samozřejmě size preserving, to není třeba zvlášť zmiňovat.}
\todobartek{Dovedu si představit bijekci mezi spanning množinami, která nezachovává velikost těchto množin.}
1-1 correspondence
between the spanning sets of $f$ and $f'$.
A spanning set $\mathcal{T}$ of $f$
corresponds to the spanning set
$0 \mathcal{T} = \curly{0 T | T \in \mathcal{T}}$ of $f'$.
The same holds for orthogonal sets.
This proves that if $f$ is not coverable,
neither is $f'$.

\item $\apply{f}{\rep{1}{n}} = 0$:

Let $f'$ be an $(n+1)$-ary function
defined in the following way:

$$
\apply{f'}{x'} =
\begin{cases}
\apply{f}{\bits{x'}{2}{n+1}} & \text{if } \bit{x'}{1} = 0 \\
0 & \text{if } 1 \rep{0}{n} \leq x' \leq \rep{1}{n} 0 \\
1 & \text{if } x' = \rep{1}{n+1}
\end{cases}
$$

\todo[inline]{Note that with Kucera on 15 April 2015, we found it necessary for $f$ to also have the false point $\rep{1}{n-1} 0$ for the procedure to work. I believe that the proof also works in the other case as well.}

%f' is l+1 swtich
Again,
the $l$ switches of $f$ translate to $l$ switches of $f'$
by prepending a $0$.
The $(l+1)$-st switch is $\rep{1}{n} 0$ in this case.

To show that $f'$ is not coverable,
we will prove bounds on the size of its minimum spanning set and maximum orthogonal set.

%f' is not coverable
\begin{description}
\item[$dnf(f') \geq dnf(f) + 1$]
We need a dedicated ternary vector to span the true point
$\rep{1}{n+1}$ in $f'$.
If a ternary vector spanned
\todokucera{\quot{would span}}
both $\rep{1}{n+1}$
and $0 x$ for any $n$-bit $x$,
it would necessarily also span the false point
$0 \rep{1}{n}$.

If we could span $f'$ with less than $dnf(f) + 1$ vectors,
we could span $f$ with less than $dnf(f)$ vectors,
since given a spanning set of $f'$,
leaving out the vector
that spans the true point $\rep{1}{n+1}$
and removing the leading symbol of the rest
(necessarily a $0$)
gives a spanning set of $f$.

\item[$ortho(f') \leq ortho(f) + 1$]
Let us proceed by contradiction
and let $V'$ be an orthogonal set of $f'$
of size $ortho(f) + 2$.
Let $V = \curly{\bits{v'}{2}{n+1} | v' \in V'
\text{ and } \bit{v'}{1} = 0}$.
Since $V'$ only consists of true points of $f'$,
there can be at most one vector in $V'$
that starts with a $1$.
It follows that
$\size{V} \geq \size{V'} - 1 = ortho(f) + 1$.
$V$ is an orthogonal set of $f$ --
if $0 u$ and $0 v$ are orthogonal in $f'$,
every ternary vector that spans them must span
a false point $0 x$.
Leaving out the leading symbol preserves the relation,
and thus orthogonality as well.
We have shown an orthogonal set of $f$ of size at least
$ortho(f) + 1$,
which contradicts the premise that $ortho(f)$ is the size
of a maximum orthogonal set of $f$.
\end{description}

Since $f$ is not coverable,
$ortho(f) < dnf(f)$.

Putting the inequalities together:

$$
ortho(f') \leq ortho(f) + 1 < dnf(f) + 1 \leq dnf(f')
$$

We conclude that $f'$ is an $(l+1)$-switch non-coverable
function.
\end{enumerate}
\end{proof}

We conclude with the following theorem:

\begin{theorem}
For any $l \geq 3$,
there is an $l$-switch non-coverable function.

Moreover, the arity of such function can be as small as
$l+1$.
\end{theorem}

\begin{proof}
Starting with the $4$-ary
\todomaybe{Use a better word.}
$3$-switch function $\twointfn{f}{4}{0}{4}{11}{14}$ from \citet{Dubovsky2012},
we can get a non-coverable $l$-switch function for any $l \geq 3$
by using \cref{lemma:noncoverableinduction} inductively.
The arity of $\twointfn{f}{4}{0}{4}{11}{14}$ is $4$
and every time we increment $l$ in \cref{lemma:noncoverableinduction},
we increase the arity by $1$.
\end{proof}

This result indicates that minimizing the \acrshort{dnf}
representation of functions that have at least $3$ switches
requires a different approach for proving optimality.
\todo{Review.}

\todomaybe[inline]{Also differentiate the functions based on their values in the points $\rep{0}{n}$ and $\rep{1}{n}$ and show we can find an $(l+1)$-switch function for all the relevant cases.
-- Not necessary, but if the proof is simple, present it.}
