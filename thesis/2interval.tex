\chapter{\texorpdfstring{$2$}{2}-interval functions}
\label{chap:2interval}

\todo[inline]{Add an example with pictures.}

In this chapter we will revise the existing results specific to
$2$-interval functions.
\citet{Dubovsky2012}
provides a more detailed overview.
We will also introduce a simplified minimization algorithm
for $2$-switch $2$-interval functions
and generalize the non-coverability property to all
classes of $l$-switch functions for $l \geq 3$.

\citeauthor{Dubovsky2012} identified several classes
of $2$-interval functions \citep[p.~5]{Dubovsky2012}
based on the number of switches,
although he did not use the term \quot{switch}.
Recall that an $l$-switch function has exactly $l$ input
vectors $x$ such that $\apply{f}{x} \neq \apply{f}{x+1}$.
It is easy to see that a $2$-interval function
can only have $2$, $3$ or $4$ switches.
For $2$-switch functions
\citeauthor{Dubovsky2012} proposed
a polynomial minimization algorithm
while for $3$- and $4$-switch functions
he proposed a polynomial $2$-approximation algorithm.

%It is easy to see that
%prefix and suffix interval functions are $1$-switch.
%Following \cref{example:switchesintervals},
%a $2$-switch Boolean function such that $f(0) = 0$
%is $1$-interval,
%and a $2$-switch function such that $f(0) = 1$
%is $2$-interval.

In \cref{sec:2int2switch}
we will present a simplified minimization algorithm
for $2$-switch $2$-interval functions.
In \cref{sec:3switch}
we will show that functions with $3$ and more switches
are not coverable in general.


\section{\texorpdfstring{$2$}{2}-switch
\texorpdfstring{$2$}{2}-interval functions}
\label{sec:2int2switch}

\newcommand{\ftwointtwoswitch}[4]
{#1^{#2}_{\interval{\rep{0}{#2}}{#3},
\interval{#4}{\rep{1}{#2}}}}

\newcommand{\fnba}{\ftwointtwoswitch{f}{n}{b_1}{a_2}}

In this section
we will introduce a polynomial algorithm
that solves the following problem:
\begin{problem}
[Minimization of $2$-switch $2$-interval function]
Given a proper $2$-interval $n$-ary function $f$
such that $f(\rep{0}{n}) = 1$ and $f(\rep{1}{n}) = 1$,
find a minimum spanning set of $f$.
\end{problem}

Let $f$ be a $2$-switch $2$-interval Boolean function
(necessarily $f(\rep{0}{n}) = f(\rep{1}{n})
= 1$).\footnote{This definition corresponds to class $A_0$
in \citeauthor{Dubovsky2012}'s classification
\citep[p.~5]{Dubovsky2012}.}
\citeauthor{Dubovsky2012} described a polynomial minimization algorithm for this class of functions.
The algorithm in \citet[p.~17]{Dubovsky2012}
% pp.~17-30 (including proof of optimality)
% Section 3.2
follows the approach from \citet{Schieber2005154}
and relies on a quite complicated and technically involved case analysis.
Here we will show that we can find a minimum \acrshort{dnf} formula representing a $2$-switch $2$-interval function $f$ by reducing the function to a $1$-interval function $f'$.
The $1$-interval function $f'$ is then spanned optimally
using the polynomial algorithm
shown in \cref{chap:1interval}.
This approach is much simpler
than the case analysis in \citet{Dubovsky2012}.

A $2$-switch $2$-interval function $f$ necessarily has $f(\rep{0}{n})=f(\rep{1}{n})=1$
and thus there are two numbers $b_1$ and $a_2$
so that $f = \fnba$.
Since $f$ is $2$-switch,
a false point must separate the true points $b_1$ and $a_2$,
thus $b_1 < a_2-1$
($f$ is \emph{proper} $2$-interval).

Note that $f$ is a negation of a $2$-switch $1$-interval function.
We can also say that $f$ is a \emph{false point $1$-interval function}
as the false points of $f$ form a single interval
$\interval{b_1+1}{a_2-1}$.

\subsection{\algdesctitle}

\begin{description}
\item[Input] A pair of $n$-bit numbers $b_1, a_2$ that satisfy the inequality $b_1 < a_2 - 1$.

The numbers $b_1$ and $a_2$
are two of the four endpoints
of the $2$-switch proper $2$-interval
function $\fnba = f$.

\item[Output] A spanning set of $f$.
We will denote the output set with $\mathcal{T}$.

\item[Procedure]
The algoritm constructs $\mathcal{T}$ from two partial spanning sets, $\mathcal{T}_{out}$ and $\mathcal{T}_{in}$.

Let $c$ be the longest common prefix of $b_1$ and $a_2$
and let $j$ be its length.
Since $b_1 < a_2$, necessarily $j < n$.

We will split the true points of $f$
into two sets $S_{out}$ and $S_{in}$
based on the $j$-bit prefix.
We will span the sets $S_{out}$ and $S_{in}$ separately.

Let $S_{out}$ be the set of all true points of $f$
that do not start with the prefix $c$.
Note that all vectors
that do not start with the prefix $c$
are true points of $f$,
since each of them is either smaller than $b_1$
or larger than $a_2$.
The set $S_{out}$ then contains precisely
all the $n$-bit binary vectors
that are strictly smaller than $c \rep{0}{n-j}$
or strictly larger than $c \rep{1}{n-j}$:
\begin{align*}
S_{out}
&= \curly{x \in \booldom^n
| \binand{f(x) = 1}{\bits{x}{1}{j} \neq c}} \\
&= \curly{x \in \booldom^n
| \bits{x}{1}{j} \neq c} \\
&= \curly{x \in \booldom^n
| \binor{x < c \rep{0}{n-j}}{x > c \rep{1}{n-j}}}
\end{align*}

We get the set $\mathcal{T}_{out}$ that spans exactly
$S_{out}$
in the following way:
$$
\mathcal{T}_{out} = \curly{\bits{c}{1}{o-1} \compl{\bit{c}{o}} \rep{\phi}{n-o} | o \in \curly{1, \ldots, j}}
$$

%Note that $\size{\mathcal{T}_{out}} = j$.

We still need to span all the true points of $f$
that start with the prefix $c$.
Let $S_{in}$ denote the set of these points:
\begin{align*}
S_{in}
&= \curly{x \in \booldom^n
| f(x) = 1 \text{ and } \bits{x}{1}{j} = c} \\
&= \curly{x \in \booldom^n
| \binand{(\binor{x \leq b_1}{x \geq a_2})}{\bits{x}{1}{j} = c}}
\end{align*}

Since $b_1 < a_2$,
necessarily $\bit{b_1}{j+1} = 0$ and $\bit{a_2}{j+1} = 1$.
Let $b_1' = \bits{b_1}{j+2}{n}$
and $a_2' = \bits{a_2}{j+2}{n}$.
Note that $b_1 = c 0 b_1'$ and $a_2 = c 1 a_2'$.

Let's span the $1$-interval function
$f' = f^{n-j}_{\interval{0 a_2'}{1 b_1'}}$
optimally using the procedure shown in \cref{chap:1interval}.
Let's denote the resulting minimum spanning set of $f'$
with $\mathcal{T}_{\interval{0 a_2'}{1 b_1'}}$.
We can use this set to span $S_{in}$:
$$
\mathcal{T}_{in} =
\curly{c \compl{\bit{T}{1}} \bits{T}{2}{n-j}
| T \in \mathcal{T}_{\interval{0 a_2'}{1 b_1'}}}
$$

Having constructed both of the sets
$\mathcal{T}_{out}$ and $\mathcal{T}_{in}$,
the algorithm returns their union.
\end{description}

\subsection{\titlefeasibility}

\begin{theorem}
\label{theorem:2switch2intfeasibility}
$\mathcal{T}$ spans exactly $\fnba$.
\end{theorem}

We will separately prove that $\mathcal{T}_{out}$ spans exactly $S_{out}$
and that $\mathcal{T}_{in}$ spans exactly $S_{in}$.
Since $\mathcal{T} = \mathcal{T}_{out} \cup \mathcal{T}_{in}$
and $TP(\fnba) = S_{out} \cup S_{in}$,
the correctness of
\cref{theorem:2switch2intfeasibility} will follow.

\begin{lemma}
$\mathcal{T}_{out}$ spans exactly $S_{out}$.
\end{lemma}

\begin{proof}
In order to show that
$span(\mathcal{T}_{out}) = S_{out}$,
we will prove the two inclusions separately.

\begin{description}
\item[$span(\mathcal{T}_{out}) \subseteq S_{out}$]
Let $\bits{c}{1}{o-1} \compl{\bit{c}{o}} \rep{\phi}{n-o}
\in \mathcal{T}_{out}$
span $x$.
Since the $j$-bit prefix of $x$ differs from $c$,
we conclude that $x \in S_{out}$.

\item[$span(\mathcal{T}_{out}) \supseteq S_{out}$]
Let $x \in S_{out}$ such that $x < c \rep{0}{n-j}$.
Recall that by \Cref{observation:lexicographicorder},
there must be binary vectors $\hat{c}$, $x'$ and $c'$
such that $x = \hat{c} 0 x'$
and $c \rep{0}{n-j} = \hat{c} 1 c'$
(namely $\hat{c}$ is a proper prefix of $c$).
Let $o = \size{\hat{c}} + 1$.
It is easy to see that the vector
$\hat{c} 0 \rep{\phi}{n-o} \in \mathcal{T}_{out}$
spans $x$.
If $x > c \rep{1}{n-j}$,
a symmetric argument shows a vector in $\mathcal{T}_{out}$
that spans $x$.
We conclude that $\mathcal{T}_{out}$ spans $x$.
\end{description}
\end{proof}

\begin{lemma}
$\mathcal{T}_{in}$ spans exactly $S_{in}$.
\end{lemma}

\begin{proof}
Again
we will prove the two inclusions separately.
\begin{enumerate}
\item $span(\mathcal{T}_{in}) \subseteq S_{in}$:

Let $c \compl{\bit{T}{1}} \bits{T}{2}{n-j} \in \mathcal{T}_{in}$ span $x$.
Clearly $x$ starts with the prefix $c$.
Let $x = c \alpha x'$ where $\alpha \in \booldom$.
Without loss of generality let $\alpha = 0$
(the complementary case is symmetric).

Since $c \compl{\bit{T}{1}} \bits{T}{2}{n-j}$ spans
$x = c 0 x'$,
$T$ spans $1 x'$.
Since
$T \in \mathcal{T}_{\interval{0 a_2'}{1 b_1'}}$,
we have $1 x' \in \interval{0 a_2'}{1 b_1'}$
and especially $1 x' \leq 1 b_1'$.
It follows that $x' \leq b_1'$
and $x = c 0 x' \leq c 0 b_1' = b_1$.

\item $span(\mathcal{T}_{in}) \supseteq S_{in}$:

Let $x \in S_{in}$.
This means that $\bits{x}{1}{j} = c$,
and $x \leq b_1$ or $x \geq a_2$.
Without loss of generality let $\bit{x}{j+1} = 0$
(the complementary case is symmetric)
and let $x = c 0 x'$.

Since $\bits{x}{1}{j+1} = c 0 < c 1 = \bits{a_2}{1}{j+1}$,
$x$ cannot be greater than or equal to $a_2$.
Thus necessarily $x \leq b_1$.
Since $x$ and $b_1$ share the prefix $c 0$,
necessarily $x' \leq b_1'$
and by extension $1 x' \leq 1 b_1'$.
Trivially $1 x' \geq 0 a_2'$.

Since $1 x' \in \interval{0 a_2'}{1 b_1'}$,
there must be some
$T \in \mathcal{T}_{\interval{0 a_2'}{1 b_1'}}$
that spans $1 x'$.
The corresponding
$c \compl{\bit{T}{1}} \bits{T}{2}{n-j}
\in \mathcal{T}_{in}$ clearly spans $x = c 0 x'$.
\end{enumerate}
\end{proof}

We conclude that $\mathcal{T}$ spans exactly $\fnba$.

\subsection{\titleoptimality}

To show that $\mathcal{T}$
is a minimum spanning set of $f = \fnba$,
we will construct an orthogonal set $V$ of $f$
of size $\size{\mathcal{T}}$.
By \cref{theorem:orthodnf} this proves
that the spanning set $\mathcal{T}$ is minimum.

The set $V$ consists of three subsets:
\begin{itemize}
\item $V_{in} = \curly{c \compl{\bit{v}{1}} \bits{v}{2}{n-j}
| v \in V_{\interval{0 a_2'}{1 b_1'}}}$,
where $V_{\interval{0 a_2'}{1 b_1'}}$ is an orthogonal set of $f_{\interval{0 a_2'}{1 b_1'}}$
of size $\size{\mathcal{T}_{\interval{0 a_2'}{1 b_1'}}} = \size{\mathcal{T}_{in}}$.
Such set exists because all $1$-interval functions
are coverable
(see \Cref{observation:1intervalcoverable}).

\item $V_0 =
\curly{\bits{c}{1}{o - 1} 0 \bits{c}{o + 1}{j} \bits{(b_1 + 1)}{j+1}{n}
| \bit{c}{o} = 1}$

\item $V_1 =
\curly{\bits{c}{1}{o - 1} 1 \bits{c}{o + 1}{j} \bits{(a_2 - 1)}{j+1}{n}
| \bit{c}{o} = 0}$
\end{itemize}

Note that the vector
$\bits{c}{1}{o - 1} 0 \bits{c}{o + 1}{j} \bits{(b_1 + 1)}{j+1}{n} \in V_0$
(for any $o$ such that $\bit{c}{o} = 1$)
is equal to
$\bits{(b_1+1)}{1}{o - 1} 0 \bits{(b_1+1)}{o + 1}{n}$
because $\bit{b_1}{j+1} = 0$,
and by extension
$\bits{(b_1+1)}{1}{j} = \bits{b_1}{1}{j} = c$.
A symmetric observation can be made for vectors in $V_1$.

It is easy to see that
$\size{V_{in}} = \size{\mathcal{T}_{in}}$.
Since there is one vector in $V_0$
for each $1$ bit of $c$
and one vector in $V_1$
for each $0$ bit of $c$,
$\size{V_0} + \size{V_1} = \size{c} = j = \size{\mathcal{T}_{out}}$.

To see that the sets $V_{in}$, $V_0$ and $V_1$ are pairwise disjoint,
consider any $v_{in} \in V_{in}$, $v_0 \in V_0$ and $v_1 \in V_1$.
The $j$-prefix of both $v_0$ and $v_1$ differs from $c$,
so $v_0 \neq v_{in}$ and $v_1 \neq v_{in}$.
If $\bits{v_0}{1}{j} = \bits{c}{1}{o-1} 0 \bits{c}{o+1}{j}$,
then $v_0$ differs from $v_1$ in the $o$-th position.
We conclude that $\size{V} = \size{V_{in}} + \size{V_0} + \size{V_1}$.
Since $\size{V_{in}} = \size{\mathcal{T}_{in}}$
and $\size{V_0} + \size{V_1} = \size{\mathcal{T}_{out}}$,
$\size{V} = \size{\mathcal{T}}$.

We still need to prove that $V$ is orthogonal
with respect to $f$.
Recall that by \cref{def:orthogonal},
the set $V$ is orthogonal
if and only if
it only contains true points
and every pair of vectors in $V$ is orthogonal.

\begin{lemma}
$V \subseteq TP(\fnba)$
\end{lemma}

\begin{proof}
It is easy to see that $v_0 \leq b_1$ and $v_1 \geq a_2$
for any $v_0 \in V_0$ and $v_1 \in V_1$.

Let $v_{in} \in V_{in}$.
Let $v$ be the corresponding vector
in $V_{\interval{0 a_2'}{1 b_1'}}$
($v_{in} = c \compl{\bit{v}{1}} \bits{v}{2}{n-j}$).
Since $V_{\interval{0 a_2'}{1 b_1'}}$
is orthogonal with respect to
$f_{\interval{0 a_2'}{1 b_1'}}$,
necessarily $v \geq 0 a_2'$ and $v \leq 1 b_1'$.
Without loss of generality let $\bit{v}{1} = 0$
(the complementary case is symmetric).
Since $v = 0 \bits{v}{2}{n-j} \geq 0 a_2'$,
$v_{in} = c 1 \bits{v}{2}{n-j}
\geq c 1 a_2' = a_2$.

We conclude that all the vectors in $V$
are true points of $\fnba$.
\end{proof}

\begin{lemma}
Let $x, y \in V$, $x \neq y$.
Then $x$ and $y$ are orthogonal with respect to $\fnba$.
\end{lemma}

\begin{proof}
We need to show that $x$ and $y$
cannot be spanned by a single ternary vector.
Let $T$ be any ternary vector that spans both $x$ and $y$.
We shall consider all six possible combinations
one by one:
\begin{itemize}
\item $x,y \in V_{in}$:
Let $x = c \compl{\bit{v_x}{1}} \bits{v_x}{2}{n-j}$
and $y = c \compl{\bit{v_y}{1}} \bits{v_y}{2}{n-j}$.
The vectors $v_x$ and $v_y$ are orthogonal with respect to $f_{\interval{0 a_2'}{1 b_1'}}$
by definition of $V_{\interval{0 a_2'}{1 b_1'}}$.

Let $\hat{T} = \compl{\bit{T}{j+1}} \bits{T}{j+2}{n}$.
Since $v_x$ and $v_y$ are orhtogonal with respect to
$f_{\interval{0 a_2'}{1 b_1'}}$
and since $\hat{T}$ spans both $v_x$ and $v_y$,
let $\hat{z}$ be a false point
of $f_{\interval{0 a_2'}{1 b_1'}}$
that is spanned by $\hat{T}$.
Without loss of generality let $\bit{\hat{z}}{1} = 0$
(the complementary case is symmetric).
Since $\hat{z}$ is a false point
of $f_{\interval{0 a_2'}{1 b_1'}}$
and $\bit{\hat{z}}{1} = 0$,
necessarily
$\hat{z} = 0 \bits{\hat{z}}{2}{n-j} < 0 a_2'$
and by extension
\todomaybe{Replace with \quot{contraction}.}
$\bits{\hat{z}}{2}{n-j} < a_2'$.

Let $z
= c \compl{\bit{\hat{z}}{1}} \bits{\hat{z}}{2}{n-j}
= c 1 \bits{\hat{z}}{2}{n-j}$.
To see that $z$ is a false point of $\fnba$,
note that $z > b_1$ since $\bit{z}{j+1} = 1$
and $z < a_2$ since $\bits{z}{j+2}{n} < a_2'$.
To see that $T$ spans $z$,
note that
$\bits{T}{1}{j}$ spans
$\bits{x}{1}{j} = c = \bits{z}{1}{j}$
and $\bits{T}{j+1}{n} = \compl{\bit{\hat{T}}{1}} \bits{\hat{T}}{2}{n-j}$ spans $\compl{\bit{\hat{z}}{1}} \bits{\hat{z}}{2}{n-j} = \bits{z}{j+1}{n}$.
\item $x,y \in V_0$:
$T$ spans the false point $b_1 + 1$.
\item $x,y \in V_1$:
$T$ spans the false point $a_2 - 1$.
\item $x \in V_{in}, y \in V_0$:
$T$ spans the false point $b_1 + 1$.
\item $x \in V_{in}, y \in V_1$:
$T$ spans the false point $a_2 - 1$.
\item $x \in V_0, y \in V_1$:
$T$ spans the false point $b_1 + 1$.
\end{itemize}

We have shown that no matter which combination
of different vectors $x, y$ from $V$
and a ternary vector that spans both $x$ and $y$
we choose,
the ternary vector necessarily spans some false point
of $\fnba$.
\end{proof}

We conclude that $V$ is orthogonal
with respect to $\fnba$.
Since $\size{V} = \size{\mathcal{T}}$,
by \cref{theorem:orthodnf} $\mathcal{T}$
is a minimum spanning set of $\fnba$.

\begin{corollary}
\label{corollary:2switch2intervalcoverable}
Every $2$-switch $2$-interval function is coverable.
\end{corollary}

Since every $2$-switch function is either $1$- or $2$-interval (see \cref{example:switchesintervals})
and every $1$-switch function is $1$-interval,
we collect the results from \Cref{observation:1intervalcoverable}
and \cref{corollary:2switch2intervalcoverable}
in the following statement:
\begin{corollary}
Every function with at most $2$ switches is coverable.
\end{corollary}

\section{Functions with \texorpdfstring{$3$}{3}
and more switches}
\label{sec:3switch}
\todomaybe[inline]{Separate in a chapter.}

The situation changes in $3$-switch functions.
In $1$-switch (prefix and suffix)
and $2$-switch ($1$-interval and $2$-interval
with extreme outer endpoints)
functions,
we managed to efficiently find minimum spanning sets.
The proofs of optimality of the spanning sets
depend on construction of sufficiently large
orthogonal sets.\footnote{While we have only shown
an orthogonal set explicitly
in the case of prefix interval (\autoref{sec:prefix}),
the other cases use orthogonal sets implicitly,
both in \citep{Schieber2005154}
and \autoref{sec:2int2switch}.}
\todo{Update footnote once we construct an orthogonal set explicitly in \autoref{sec:2int2switch}.}
The fact that this aproach succeeds
depends on the fact that all the functions we have considered so far ($1$- and $2$-switch)
are \emph{coverable}
(according to \cref{def:coverable}).

\citeauthor{Dubovsky2012} has shown that
the $3$-switch function
$f = \twointfn{f}{4}{0}{4}{11}{14}$ is not coverable.
The size of a maximum orthogonal set of $f$ is $4$
while the size of a minimum spanning set of $f$ is $5$.
\citep[p.~32]{Dubovsky2012}
This shows that $3$-switch functions are not coverable in general.
\citeauthor{Dubovsky2012} proved these bounds
on sizes of orthogonal sets and spanning sets
by exhaustion using software tools.
We will generalize the result for functions
with larger number of switches.

\begin{lemma}
\label{lemma:noncoverableinduction}
If there is an $l$-switch function that is not coverable,
then there is also an $(l+1)$-switch function
that is not coverable.
\end{lemma}

\begin{proof}
Let $f$ be an $n$-ary $l$-switch function
that is not coverable.
We will construct an $(n+1)$-ary $(l+1)$-switch function $f'$
that is not coverable.
We shall distinguish two cases depending on the value of $f$ on $\rep{1}{n}$.

\begin{enumerate}
\item $\apply{f}{\rep{1}{n}} = 1$:

Let $f'$ be an $(n+1)$-ary function
defined in the following way:

\[
\apply{f'}{x'} =
\begin{cases}
\apply{f}{\bits{x'}{2}{n+1}} & \text{if } \bit{x'}{1} = 0 \\
0 & \text{if } \bit{x'}{1} = 1
\end{cases}
\]

Since $f$ is $l$-switch,
there are exactly $l$ vectors $x$ of length $n$
($x < \rep{1}{n}$)
such that $\apply{f}{x} \neq \apply{f}{x+1}$.
Prepending a $0$ to each of these vectors,
we get $l$ vectors $x'$ of length $n+1$
such that $\apply{f'}{x'} \neq \apply{f'}{x'+1}$.
The $(l+1)$-st switch vector of $f'$
is $0 \rep{1}{n}$,
since $\apply{f'}{0 \rep{1}{n}} = 1$
and $\apply{f'}{1 \rep{0}{n}} = 0$.
On the other hand,
observe that all switch vectors of $f'$
correspond to switch vectors of $f$
with the single exception of $0 \rep{1}{n}$.
Thus we have shown that $f'$ is $(l+1)$-switch.

There is
a size preserving
1-1 correspondence
between the spanning sets of $f$ and $f'$.
A spanning set $\mathcal{T}$ of $f$
corresponds to the spanning set
$0 \mathcal{T} = \curly{0 T | T \in \mathcal{T}}$ of $f'$.
The same holds for orthogonal sets.
This proves that if $f$ is not coverable,
neither is $f'$.

\item $\apply{f}{\rep{1}{n}} = 0$:

Let $f'$ be an $(n+1)$-ary function
defined in the following way:

$$
\apply{f'}{x'} =
\begin{cases}
\apply{f}{\bits{x'}{2}{n+1}} & \text{if } \bit{x'}{1} = 0 \\
0 & \text{if } 1 \rep{0}{n} \leq x' \leq \rep{1}{n} 0 \\
1 & \text{if } x' = \rep{1}{n+1}
\end{cases}
$$

%f' is l+1 swtich
Again,
the $l$ switches of $f$ translate to $l$ switches of $f'$
by prepending a $0$.
The $(l+1)$-st switch is $\rep{1}{n} 0$ in this case.

To show that $f'$ is not coverable,
we will prove bounds on the size of its minimum spanning set and maximum orthogonal set.

%f' is not coverable
\begin{description}
\item[$dnf(f') \geq dnf(f) + 1$]
We need a dedicated ternary vector to span the true point
$\rep{1}{n+1}$ in $f'$.
If a ternary vector spanned
both $\rep{1}{n+1}$
and $0 x$ for any $n$-bit $x$,
it would necessarily also span the false point
$0 \rep{1}{n}$.

If we could span $f'$ with less than $dnf(f) + 1$ vectors,
we could span $f$ with less than $dnf(f)$ vectors,
since given a spanning set of $f'$,
leaving out the vector
that spans the true point $\rep{1}{n+1}$
and removing the leading symbol of the rest
(necessarily a $0$)
gives a spanning set of $f$.

\item[$ortho(f') \leq ortho(f) + 1$]
Let us proceed by contradiction
and let $V'$ be an orthogonal set of $f'$
of size $ortho(f) + 2$.
Let $V = \curly{\bits{v'}{2}{n+1} | v' \in V'
\text{ and } \bit{v'}{1} = 0}$.
Since $V'$ only consists of true points of $f'$,
there can be at most one vector in $V'$
that starts with a $1$.
It follows that
$\size{V} \geq \size{V'} - 1 = ortho(f) + 1$.
$V$ is an orthogonal set of $f$ --
if $0 u$ and $0 v$ are orthogonal in $f'$,
every ternary vector that spans them must span
a false point $0 x$.
Leaving out the leading symbol preserves the relation,
and thus orthogonality as well.
We have shown an orthogonal set of $f$ of size at least
$ortho(f) + 1$,
which contradicts the premise that $ortho(f)$ is the size
of a maximum orthogonal set of $f$.
\end{description}

Since $f$ is not coverable,
$ortho(f) < dnf(f)$.

Putting the inequalities together:

$$
ortho(f') \leq ortho(f) + 1 < dnf(f) + 1 \leq dnf(f')
$$

We conclude that $f'$ is an $(l+1)$-switch non-coverable
function.
\end{enumerate}
\end{proof}

We conclude with the following theorem:

\begin{theorem}
For any $l \geq 3$,
there is an $l$-switch non-coverable function.

Moreover, the arity of such function can be as small as
$l+1$.
\end{theorem}

\begin{proof}
Starting with the $4$-ary
$3$-switch function $\twointfn{f}{4}{0}{4}{11}{14}$ from \citet{Dubovsky2012},
we can get a non-coverable $l$-switch function for any $l \geq 3$
by using \cref{lemma:noncoverableinduction} inductively.
The arity of $\twointfn{f}{4}{0}{4}{11}{14}$ is $4$
and every time we increment $l$ in \cref{lemma:noncoverableinduction},
we increase the arity by $1$.
\end{proof}

This result indicates that
if we consider an algorithm that minimizes the representations of all $l$-switch functions for any $l \geq 3$,
then we have to use a different tool than orthogonal sets to show that the algorithm really finds an optimal spanning set.

\todomaybe[inline]{Also differentiate the functions based on their values in the points $\rep{0}{n}$ and $\rep{1}{n}$ and show we can find an $(l+1)$-switch function for all the relevant cases.
-- Not necessary, but if the proof is simple, present it.}
