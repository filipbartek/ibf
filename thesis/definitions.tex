\chapter{Definitions}

The basic terminology used in this thesis
was introduced by
\todo{Add Hammer.}
\citet{Schieber2005154},
\citet{Dubovsky2012}
and \citet{Husek2014, Husek2015}.
\todokucera{Dále používáte obecnou booleovskou terminologii, ke které by se hodila citace třeba té booleovský knížky od Y. Cramy a P.L. Hammera.
Crama, Hammer: Boolean Functions - Theory, Algorithms, and Applications}

\section{Vector operations}

Throughout the thesis we will use vectors
over small finite domains extensively.
We shall write a vector as a sequence of symbols,
for example $00101$ is a vector of length $5$.
The concatenation of two vectors $u$ and $v$
will be denoted simply as $uv$,
for example $00v11$ denotes a vector
which is formed as a concatenation
of the three vectors $00$, $v$ and $11$.

\begin{definition}
[Component extraction $\bit{v}{i}$
{\definitionsource{Schieber2005154}}] % [p.~156]
Let $v$ be a vector of length $n$ and $1 \leq i \leq n$.
Then $\bit{v}{i}$ is the $i$-th component of $v$.
\end{definition}

\begin{definition}
[Subsequence extraction $\bits{v}{a}{b}$
{\definitionsource{Schieber2005154}}] % [p.~157]
Let
$v$ be a vector of length $n$
and
$1 \leq a \leq b \leq n$.
Then $\bits{v}{a}{b}$ is the subvector of $v$
that starts at $a$-th position
and ends at $b$-th position
($\bits{v}{a}{b}
= \bit{v}{a} \bit{v}{a+1}
\ldots \bit{v}{b-1} \bit{v}{b}$).
\end{definition}

Notably,
$v = \bit{v}{1} \ldots \bit{v}{n} = \bits{v}{1}{n}$.

\begin{definition}
[Symbol repetition $\rep{\alpha}{n}$
{\definitionsource{Schieber2005154}}] % [p.~157]
Let $\alpha$ be a symbol (for example $0$ or $1$)
and $n \in \nat$.
Then $\rep{\alpha}{n}$ is the vector of length $n$
each component of which is equal to $\alpha$.
\end{definition}

For example
$\rep{0}{2} = 00$.

\section{Binary vectors}

\begin{definition}
[Binary vector
{\definitionsource{Schieber2005154}}] % [p.~156]
We will use the term \definiendum{binary vector}
to denote a vector over the Boolean domain.
That is, $x$ is a binary vector if and only if
$x \in \booldom^n$ for some $n \in \nat$.
We call such $x$ an \definiendum{$n$-bit binary vector}.
\end{definition}

Note that an $n$-bit binary vector $x$
represents the natural number
$\sum_{i = 1}^n \bit{x}{i} 2^{n-i}
= \sum_{i|\bit{x}{i}=1} 2^{n-i}$.
An $n$-bit vector
corresponds to a number between $0$ and $2^n - 1$.
We will not differentiate between a number
and its binary vector representation.
Specifically,
we will use the standard linear order of natural numbers
to compare binary vectors.
Note that the linear order on natural numbers
corresponds to lexicographic order
on binary vectors:
\begin{observation}
\label{observation:lexicographicorder}
If $a$ and $b$ are $n$-bit binary vectors,
then $a < b$
if and only if
there are binary vectors $c$,
$a'$ and $b'$
such that
$a = c 0 a'$ and $b = c 1 b'$.
We call such $c$ the \definiendum{longest common prefix}
of $a$ and $b$.
\end{observation}

\section{Boolean functions}

\begin{definition}[Boolean function]
\todo{Cite a source.}
$f: \booldom^n \rightarrow \booldom$ is
an \definiendum{$n$-ary Boolean function}.
\end{definition}

Since we will not be dealing with any non-Boolean functions,
in the remainder of the text,
we will use the terms
\quot{function} and \quot{Boolean function}
interchangeably.

\begin{definition}[True point]
\todo{Cite a source.}
Let $f$ be an $n$-ary Boolean function.
An $n$-bit binary vector $x$ is a \definiendum{true point}
of $f$ if and only if $f(x)  = 1$.
\end{definition}

Equivalently,
we define a \definiendum{false point} of $f$
as any point $x$
such that $f(x) = 0$.

We'll denote the set of all true points of $f$ as $TP(f)$
and the set of all false points as $FP(f)$.

\section{\texorpdfstring{\acrshort{dnf}}{DNF}
representations of Boolean functions}

% http://en.wikipedia.org/wiki/Boolean_function
Every $n$-ary Boolean function $f$ can be expressed
as a propositional formula $\mathcal{F}$ on $n$ variables
$x_1, \ldots, x_n$.
There's a natural bijection
between binary vectors of length $n$
and valuations of $n$ variables
($1$-bits in the binary vector
correspond exactly to $1$-valued variables
in the valuation).

% TODO: Add example Boolean function with logical representation etc.

% http://en.wikipedia.org/wiki/Disjunctive_normal_form
\begin{definition}[\Acrlong{dnf}]
\todo{Cite a source.}
% Dubovsky2012: Definitions 1.9-1.11, page 3
A \definiendum{literal} is an occurrence
of a variable ($x_i$) or its negation ($\nott{x_i}$)
in a propositional formula.
A \definiendum{term} is
an elementary conjunction of literals,
that is a conjunction in which every variable appears
at most once.
A \definiendum{\acrfull{dnf}} propositional formula
is a disjunction of terms.
We will often view terms as sets of literals
and \acrshort{dnf} formulas as sets of terms.
\end{definition}

% http://en.wikipedia.org/wiki/Valuation_%28logic%29
A \acrshort{dnf} formula $\mathcal{F}$ is
a \definiendum{\acrshort{dnf} representation}
of a Boolean function $f$
if and only if
$\fora{v \in \booldom^n}{f(v) = 1 \iff
\valuation{\mathcal{F}}{v} \equiv 1}$,
where $\valuation{\mathcal{F}}{v}$ is the formula
given by substituting
each occurrence of a variable $x_i$ in $\mathcal{F}$
with the value $\bit{v}{i}$
for every $i \in \curly{1, \ldots, n}$.

\begin{definition}[Minimum \acrshort{dnf} representation]
\todo{Cite a source.}
A \acrshort{dnf} representation of function $f$ is
\definiendum{minimum}
if and only if
there is no \acrshort{dnf} representation of $f$
with fewer terms.

We shall denote the size of
a minimum \acrshort{dnf} representation
of function $f$
by $\apply{dnf}{f}$.
\end{definition}

Note that a function may have more than one minimum
\acrshort{dnf} representation.

The focus of this thesis is
minimizing the number of terms
of \acrshort{dnf} representations
of certain Boolean functions.

\section{Ternary vectors}

\begin{definition}
[Ternary vector
{\definitionsource{Schieber2005154}}] % [p.~156]
A \definiendum{ternary vector}
is a vector over the set
$\{0, 1, \phi\}$.
\end{definition}

There's a natural bijection between ternary vectors
and \acrshort{dnf} terms
(elementary conjunctions of literals).
Each of the $n$ components of a ternary vector $T$
corresponds to an occurrence (or its absence)
of one of the $n$ variables of a term $C$:

\begin{center}
\begin{tabular}{cc}
$\bit{T}{i}$ & $C \cap \curly{x_i, \overline{x_i}}$ \\
\hline
$\phi$ & $\emptyset$ \\
$1$ & $\curly{x_i}$ \\
$0$ & $\curly{\overline{x_i}}$
\end{tabular}
\end{center}

Note that since $C$ is elementary,
it can not contain both $x_i$ and $\overline{x_i}$
for any $i$.

Namely,
a \emph{binary} vector
(a ternary vector that does not contain any $\phi$)
corresponds to a full term
(that is one that contains every variable),
and the ternary vector $\rep{\phi}{n}$
corresponds to the empty term (tautology).
% TODO: Consider defining "more qualified", "most qualified", "least qualified".

It is easy to see that a set of ternary vectors,
each of length $n$,
corresponds to a \acrshort{dnf} \emph{formula}
on $n$ variables.

\begin{definition}
[Spanning
{\definitionsource{Schieber2005154}}] % [p.~156]
A ternary vector $T$ of length $n$ \definiendum{spans}
a binary vector $x$ of length $n$
if and only if
$\fora{i \in \curly{1, \ldots, n}}
{\bit{T}{i} = \phi \text{ or } \bit{T}{i} = \bit{x}{i}}$.

A set of ternary vectors $\mathcal{T}$
\definiendum{spans}
a binary vector $x$
if and only if
some vector in $\mathcal{T}$ spans $x$.
%$\ex{T \in \mathcal{T}}
%{T \text{ spans } x}$.
\end{definition}

The $i$-th symbol of a ternary vector
constrains the $i$-th symbol of the spanned binary vector
(or, equivalently, the valuation of the $i$-th variable).
If $\bit{T}{i}$ is a \quot{fixed bit}
(that is $\bit{T}{i} \in \curly{0, 1}$),
the spanned binary vector $x$
must have the same value
in its $i$-th position,
that is $\bit{x}{i} = \bit{T}{i}$.
If $\bit{T}{i}$ is the \quot{don't care symbol} $\phi$,
% "don't" is used in Schieber2005
the spanned binary vector
may have any value in its $i$-th position.

In the correspondence between ternary vectors and terms,
spanning corresponds to satisfying.
The ternary vector $T$ of length $n$ spans $x$
if and only if
$C(x) \equiv 1$,
where $C$ is the term on $n$ variables
that corresponds to $T$.
Similarly,
it is easy to see that
a set of ternary vectors $\mathcal{T}$ spans $x$
if and only if
$\mathcal{F}(x) \equiv 1$,
where $\mathcal{F}$ is the \acrshort{dnf} formula
on $n$ variables
that corresponds to $\mathcal{T}$.

\begin{definition}[Spanned set]
Let $T$ be a ternary vector of length $n$.
We denote the set of points spanned by $T$
as $span(T)$:

\begin{equation*}
span(T) = \curly{
x \in \booldom^n |
T\text{ spans }x
}
\end{equation*}
\end{definition}

Note that a ternary vector
with $m$ $\phi$-positions
spans $2^m$ binary vectors.

We'll also use a natural extension of spanning
to sets of ternary vectors --
if $\mathcal{T}$ is a set of ternary vectors, then

\begin{equation*}
span(\mathcal{T}) =
\bigcup_{T \in \mathcal{T}} span(T)
\end{equation*}

\begin{definition}[Exact spanning]
A ternary vector $T$ of length $n$
\definiendum{spans exactly}
a set of $n$-bit binary vectors $S$
if and only if
$span(T) = S$.

$T$ \definiendum{spans exactly}
an $n$-ary Boolean function $f$
if and only if
$span(T) = TP(f)$.
\end{definition}

Again,
we will generalize \definiendum{exact spanning}
to sets of ternary vectors
-- the set of ternary vectors $\mathcal{T}$
\definiendum{spans exactly} $S$
if and only if
$span(\mathcal{T}) = S$.

\begin{definition}
[Spanning set
{\definitionsource{Schieber2005154}}] % [p.~156]
Let $f$ be an $n$-ary Boolean function.
The set $\mathcal{T}$ of $n$-bit ternary vectors
is a \definiendum{spanning set} of $f$
if and only if
it $\mathcal{T}$ spans exactly $f$,
that is
$span(\mathcal{T}) = TP(f)$.
\end{definition}

In other words,
a spanning set of a function
spans all of its true points
and none of its false points.

Note that spanning sets of a function
correspond to \acrshort{dnf} representations
of the function.
The correspondence preserves size
(number of ternary vectors and terms, respectively),
so a minimum spanning set
corresponds to a minimum \acrshort{dnf} representation.

\begin{definition}
[Disjoint spanning set
{\definitionsource[Section 4]{Schieber2005154}}] % [p.~165]
A spanning set $\mathcal{T}$ is \definiendum{disjoint}
if and only if the spanned sets of its vectors
do not overlap,
that is
\[
\fora{T_i, T_j \in \mathcal{T}}{T_i = T_j \text{ or }
span(T_i) \cap span(T_j) = \emptyset}
\]
\end{definition}

In terms of \acrshort{dnf},
disjoint spanning sets
correspond to \acrshort{dnf} formulas
such that every valuation satisfies at most one term.

We have introduced two equivalent representations
of Boolean functions --
one based on ternary vectors
and the other being the \acrshort{dnf} representation.
\Cref{table:representations}
shows the corresponding terms side by side.
% Note the inconsistent use of "term" here and in the table
% caption.

\begin{table}[h]
\centering
\begin{tabular}{ll}
ternary vector & term \\
ternary vector set & \acrshort{dnf} formula \\
binary vector & variable valuation \\
ternary vector \emph{spans} a binary vector &
valuation \emph{satisfies} a term \\
ternary vector set \emph{spans} a binary vector &
valuation \emph{satisfies} a \acrshort{dnf} formula
\end{tabular}
\caption{
Corresponding terms used
in Boolean function representations
}
\label{table:representations}
\end{table}

\begin{definition}[Complement
{\definitionsource{Schieber2005154}}] % [p.~156]
\label{def:complement}
The complement of a binary symbol $\alpha$
($\alpha \in \booldom$),
denoted $\compl{\alpha}$,
is $1 - \alpha$.

We get the complement of a binary vector $x$
by flipping all of its bits,
that is $\compl{x}
= (\compl{x_1}, \ldots, \compl{x_n})
= \rep{1}{n} - x$.

The complement of the $\phi$ symbol is $\phi$
($\compl{\phi} = \phi$).

It follows that
we obtain the complement of a ternary vector
by flipping all of its fixed bits.

We generalize the notion of complement to sets of ternary vectors
-- if $\mathcal{T}$ is a set of ternary vectors,
then $\compl{\mathcal{T}} = \curly{\compl{T} | T \in \mathcal{T}}$.
\end{definition}

\section{Orthogonal sets and coverable functions}

Orthogonal set is an object
that can be used to show a lower bound on the size of all spanning sets of a function.
Orthogonal sets have been used both
by \citet{Schieber2005154} (implicitly)
and \citet{Dubovsky2012} (explicitly)
in the proofs of optimality of spanning algorithms.

\begin{definition}
[Collision]
Let $x$, $y$ and $z$ be $n$-bit binary vectors.
We say that \definiendum{$x$ and $y$ collide
on the vector $z$}
if and only if
every ternary vector that spans both $x$ and $y$
necessarily also spans $z$.
\end{definition}

\begin{definition}
[Orthogonality
{\definitionsource[p.~6]{Dubovsky2012}}] % Definition 1.15
Let $x$ and $y$ be true points of a Boolean function $f$
(that is $f(x) = f(y) = 1$).
$x$ and $y$ are \definiendum{orthogonal}
with respect to $f$
if and only if
they collide on a false point of $f$.

A set of true points $S$ is \definiendum{orthogonal}
if and only if
all the vectors in $S$ are pairwise orthogonal.

We shall denote the size of
a maximum orthogonal set
of function $f$
with \definiendum{$\apply{ortho}{f}$}.
\end{definition}

\todomaybe[inline]{Relate orthogonality to disjointness of essential sets. Don't forget to deal with DNF/CNF difference. Orthogonal vectors are a special case of disjoint essential sets. They suffice in our case.}

\todo{Special case of essential set;
the other article shows that they suffice.}

\todo[inline]{cite some original author
"It turns out that this is the only type of ess. sets we need to consider."
"The following preposition appears in several texts, for example..."
Cite:
Boros - where the inequality is shown
Berge - Graphs and \emph{Hypergraphs}
ortho je dolni odhad ess}

\todobartek[inline]{Kde v knize Graphs and Hypergraphs Berge dokazuje toto tvrzení?
pp. 420-424
Theorem 5}

\begin{theorem}
[$\apply{ortho}{f} \leq \apply{dnf}{f}$
{\theoremsource[p.~89]{Boros2010}}
% Corollary 6.12
{\theoremsource[p.~6]{Dubovsky2012}}]
% Observation 1.1
\label{theorem:orthodnf}
For any Boolean function,
the size of its maximum orthogonal set
is at most as great
as the size of its minimum \acrshort{dnf} representation.
\end{theorem}

\begin{proof}
We will prove the statement by contradiction.
Let $f$ be a Boolean function such that
$\apply{ortho}{f} > \apply{dnf}{f}$.
Let $V$ be an orthogonal set of size $\apply{ortho}{f}$
and $\mathcal{T}$ be any spanning set
of size $\apply{dnf}{f}$
(such spanning set exists because of the correspondence
between \acrshort{dnf} representations and spanning sets).
Since $\size{\mathcal{T}} < \size{V}$
and $\mathcal{T}$ spans $TP(f) \supseteq S$,
there must be a ternary vector $T \in \mathcal{T}$
that spans
two distinct vectors $x, y \in V$.
However,
since $V$ is orthogonal,
$x$ is orthogonal to $y$,
so $T$ necessarily spans a false point,
which contradicts the premise that $\mathcal{T}$
is a feasible spanning set of $f$.
\end{proof}

We conclude that the size of any orthogonal set
is a lower bound on the size of minimum
\acrshort{dnf} representation.
Specifically,
if we show a \acrshort{dnf} representation of a function
and an orthogonal set of the same size,
we conclude that the representation is minimum.
The orthogonal set certifies
optimality of the \acrshort{dnf} representation.

\begin{definition}
[Coverability
{\definitionsource[Definition 2.9]{Cepek2012}}] % [p.~368]
% Definition 2.9
\label{def:coverable}
Let $f$ be a Boolean function.
$f$ is \definiendum{coverable}
if and only if
$\apply{ortho}{f} = \apply{dnf}{f}$.
\end{definition}

Informally speaking,
a coverable function is one for which it may be
easy to prove optimality of a spanning set.
We will see
in Chapters \ref{chap:1interval} and \ref{chap:2interval}
that all $1$-interval
and a subclass of $2$-interval functions
are coverable.
We will show examples of functions which are not coverable
in \autoref{sec:3switch}.

The concept of coverability was originally introduced by
\citet{Cepek2012}.
They use a different yet equivalent definition.
\todo{neni ekvivalentni - ortogonalni vektor je specialni typ esencialni mnoziny
Cepek, Savicky: Staci uvazovat esencialni mnoziny true pointu pro coverabilitu.}
For brevity,
we will omit proof of the equivalence
and simply use Definition \ref{def:coverable},
which is sufficient for our purpose.

\section{\texorpdfstring{$k$}{k}-interval
Boolean functions}

The notation for $1$- and $2$-interval
functions was introduced by \citeauthor{Schieber2005154} and \citeauthor{Dubovsky2012} respectively.
We shall generalize it to $k$-interval functions for any $k \geq 1$.

\begin{definition}[$k$-interval Boolean function]
\label{def:kibf}
Let $n \geq 0$ and $k \geq 1$.
Let $a_1, b_1, \ldots, a_k, b_k$ be $n$-bit binary vectors
such that $\rep{0}{n} \leq a_1 \leq b_1 < a_2
\leq \cdots < a_i \leq b_i < a_{i+1}
\leq \cdots < a_k \leq b_k \leq \rep{1}{n}$.
Then $\fnkab: \booldom{}^n \rightarrow \booldom{}$ is a function defined as follows:
$$
\fnkab(x) =
\begin{cases}
1 & \text{if } a_i \leq x \leq b_i \text{ for some } i \\
0 & \text{otherwise}
\end{cases}
$$

We call
$f^n_{\interval{a_1}{b_1}, \ldots, \interval{a_k}{b_k}}$
a \definiendum{$k$-interval Boolean function}
and the vectors $a_1, b_1, \ldots, a_k, b_k$ its
\definiendum{endpoints}.
\end{definition}

Note in particular that in case $n=0$,
the $1$-interval $0$-ary function
$\intervalfn{f}{0}{\epsilon}{\epsilon}$,
where $\epsilon$ denotes the binary vector of length $0$,
is properly defined and equivalent to tautology.

Note that the inequalities ensure
that the intervals $\interval{a_i}{b_i}$
are non-empty and don't intersect.
The inequalities, however, allow the intervals to be adjoint.
This means that a $k$-interval function may also be $k'$-interval for some $k' < k$.
We will often find it useful to use a less ambiguous notion of the number of intervals of a function:

\begin{definition}
[Proper $k$-interval Boolean function]
\label{def:properkibf}
Let
$f^n_{\interval{a_1}{b_1}, \ldots, \interval{a_k}{b_k}}$
be a $k$-interval Boolean function.
If the endpoints satisfy the inequalities
$b_1 < a_2 - 1$, \ldots, $b_i < a_{i+1} - 1$, \ldots,
$b_{k-1} < a_k - 1$
(in other words, adjacent intervals
are separated by at least one false point),
we call
$f^n_{\interval{a_1}{b_1}, \ldots, \interval{a_k}{b_k}}$
a \definiendum{proper $k$-interval Boolean function}.
\end{definition}

%Note that if $f$ is proper $k$-interval,
%it is also $k$-interval,
%and that if $f$ is $k$-interval
%and $g$ is \emph{proper} $l$-interval
%for $l > k$,
%then $f$ and $g$ differ.

Especially,
$\fnab$ is a function
whose true points form the interval $\interval{a}{b}$.

\Cref{tab:exampleintfns}
shows examples of interval functions.

\begin{table}[h]
\centering
\begin{tabular}{lll}
Interval function & Propositional formula & Description \\
\hline
%$\intervalfn{f}{2}{11}{11}$ &
%$x_1 \wedge x_2$ & Conjunction \\
%$\intervalfn{f}{2}{01}{11}$ &
%$x_1 \vee x_2$ & Disjunction \\
$\intervalfn{f}{n}{\rep{1}{n}}{\rep{1}{n}}$ &
$\bigwedge_i{x_i}$ & Conjunction \\
$\intervalfn{f}{n}{\rep{0}{n-1} 1}{\rep{1}{n}}$ &
$\bigvee_i{x_i}$ & Disjunction \\
$\intervalfn{f}{n}{\rep{0}{n}}{\rep{1}{n}}$ &
$1$ (an empty term) & Tautology \\
$\intervalfn{f}{n}{\rep{0}{n-1} 1}{\rep{1}{n-1} 0}$ &
$\nott{x_1 \ldots x_n} \wedge
\nott{\nott{x_1} \ldots \nott{x_n}}
\equiv \bigvee_{i \neq j}{x_i \nott{x_j}}$ &
Non-equivalence \\
$f^n_{\interval{\rep{0}{n}}{\rep{0}{n}},
\interval{\rep{1}{n}}{\rep{1}{n}}}$ &
$x_1 \ldots x_n \vee \nott{x_1} \ldots \nott{x_n}$ &
Equivalence
\end{tabular}
\caption{Examples of interval functions}
\label{tab:exampleintfns}
\end{table}

\section{\texorpdfstring{$l$}{l}-switch Boolean functions}

Given a $k$-interval function,
apparently the most important values are the endpoints of the intervals.
The important property of an interval endpoint is that
it is a place where the value of the function changes
(considering proper $k$-interval functions).
We can view these values as places of switch,
where $f$ switches its value from $0$ to $1$ or back.

The notion of switch was introduced by \citeauthor{Husek2014}\footnote{
\citeauthor{Husek2014} first used the Czech term
\quot{\foreignlanguage{czech}{zlom}}
in his thesis \citep[p.~13]{Husek2014}
and later translated the term as \quot{switch}
in personal communication \citep{Husek2015}.
}
and it turns out it is very useful when studying $2$-interval functions.

\begin{definition}
[$l$-switch function
{\definitionsource[pp.~13-14]{Husek2015}}] % Definitions 2.1, 2.3
A Boolean function $f$ is \definiendum{$l$-switch}
if and only if
there are exactly $l$ input vectors $x$ such that
$f(x) \neq f(x+1)$.

We call such $x$ a \definiendum{switch} of $f$.
\end{definition}

\begin{example}
[Switches and intervals]
\label{example:switchesintervals}
Depending on $f(0)$,
an $l$-switch Boolean function $f$
is proper $\ceil*{\frac{l}{2}}$-interval
in case $f(0) = 0$,
% f(0) = 0:
% 1-switch: 1-interval
% 2-switch: 1-interval
% 3-switch: 2-interval
% 4-switch: 2-interval
% l-switch: \ceil{l/2}-interval
or proper $\ceil*{\frac{l+1}{2}}$-interval
in case $f(0) = 1$.
% f(0) = 1:
% 1-switch: 1-interval
% 2-switch: 2-interval
% 3-switch: 2-interval
% 4-switch: 3-interval
% l-switch: \ceil{(l+1)/2}-interval

Equivalently,
a proper $k$-interval $n$-ary function is:
\begin{itemize}
\item
$(2k)$-switch in case $f(\rep{0}{n}) = f(\rep{1}{n}) = 0$
\item
$(2k-1)$-switch in case $f(\rep{0}{n}) \neq f(\rep{1}{n})$
\item
$(2k-2)$-switch in case $f(\rep{0}{n}) = f(\rep{1}{n}) = 1$
\end{itemize}
\end{example}

\todomaybe[inline]{Show that complementing bits preserves number of switches.}
