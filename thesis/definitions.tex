\chapter{Definitions}

The basic terminology used in this thesis
was introduced by
\citet{Crama2011},
\citet{Schieber2005154},
\citet{Dubovsky2012}
and \citet{Husek2014,Husek2015}.

\section{Vector operations}

Throughout the thesis we will use vectors
over small finite domains extensively.
We shall write a vector as a sequence of symbols,
for example $00101$ is a vector of length $5$.
The concatenation of two vectors $u$ and $v$
will be denoted simply as $uv$,
for example $00v11$ denotes a vector
which is formed as a concatenation
of the three vectors $00$, $v$ and $11$.

\begin{definition}
[Component extraction $\bit{v}{i}$
{\definitionsource{Schieber2005154}}] % [p.~156]
Let $v$ be a vector of length $n$ and $1 \leq i \leq n$.
Then $\bit{v}{i}$ is the $i$-th component of $v$.
\end{definition}

\begin{definition}
[Subsequence extraction $\bits{v}{a}{b}$
{\definitionsource{Schieber2005154}}] % [p.~157]
Let
$v$ be a vector of length $n$
and
$1 \leq a \leq b \leq n$.
Then $\bits{v}{a}{b}$ is the subvector of $v$
that starts at $a$-th position
and ends at $b$-th position
($\bits{v}{a}{b}
= \bit{v}{a} \bit{v}{a+1}
\ldots \bit{v}{b-1} \bit{v}{b}$).
\end{definition}

Notably,
$v = \bit{v}{1} \ldots \bit{v}{n} = \bits{v}{1}{n}$.

\begin{definition}
[Symbol repetition $\rep{\alpha}{n}$
{\definitionsource{Schieber2005154}}] % [p.~157]
Let $\alpha$ be a symbol (for example $0$ or $1$)
and $n \in \nat$.
Then $\rep{\alpha}{n}$ is the vector of length $n$
each component of which is equal to $\alpha$.
\end{definition}

For example
$\rep{0}{2} = 00$.

\section{Binary vectors}

\begin{definition}
[Binary vector
{\definitionsource{Schieber2005154}}] % [p.~156]
We will use the term \definiendum{binary vector}
to denote a vector over the Boolean domain.
That is, $x$ is a binary vector if and only if
$x \in \booldom^n$ for some $n \in \nat$.
We call such $x$ an \definiendum{$n$-bit binary vector}.
\end{definition}

Note that an $n$-bit binary vector $x$
represents the natural number
$\sum_{i = 1}^n \bit{x}{i} 2^{n-i}
= \sum_{i|\bit{x}{i}=1} 2^{n-i}$.
An $n$-bit vector
corresponds to a number between $0$ and $2^n - 1$.
We will not differentiate between a number
and its binary vector representation.
Specifically,
we will use the standard linear order of natural numbers
to compare binary vectors.
Note that the linear order on natural numbers
corresponds to lexicographic order
on binary vectors:
\begin{observation}
\label{observation:lexicographicorder}
If $a$ and $b$ are $n$-bit binary vectors,
then $a < b$
if and only if
there are binary vectors $c$,
$a'$ and $b'$
such that
$a = c 0 a'$ and $b = c 1 b'$.
We call such $c$ the \definiendum{longest common prefix}
of $a$ and $b$.
\end{observation}

\section{Boolean functions}

\begin{definition}
[Boolean function
{\definitionsource{Crama2011}}] % Definition 1.1
$f: \booldom^n \rightarrow \booldom$ is
an \definiendum{$n$-ary Boolean function}.
\end{definition}

Since we will not be dealing with any non-Boolean functions,
in the remainder of the text,
we will use the terms
\quot{function} and \quot{Boolean function}
interchangeably.

\begin{definition}
[True point
{\definitionsource{Crama2011}}] % Definition 1.1
Let $f$ be an $n$-ary Boolean function.
An $n$-bit binary vector $x$ is a \definiendum{true point}
of $f$ if and only if $f(x)  = 1$.
\end{definition}

Equivalently,
we define a \definiendum{false point} of $f$
as any point $x$
such that $f(x) = 0$.

We'll denote the set of all true points of $f$ as $TP(f)$
and the set of all false points as $FP(f)$.
\todonote[inline]{These sets are denoted by $T(f)$ and $F(f)$ in \citet[Definition 1.1]{Crama2011}. We keep the notation $TP(f)$ to avoid confusion with ternary vectors denoted as $T$ in \citet{Schieber2005154} and in this thesis.}

\section{\texorpdfstring{\acrshort{dnf}}{DNF}
representations of Boolean functions}

% http://en.wikipedia.org/wiki/Boolean_function
Every $n$-ary Boolean function $f$ can be expressed
as a Boolean formula $\mathcal{F}$ on $n$ variables
$x_1, \ldots, x_n$.
There's a natural bijection
between binary vectors of length $n$
and valuations of $n$ variables
($1$-bits in the binary vector
correspond exactly to $1$-valued variables
in the valuation).

% TODO: Add example Boolean function with logical representation etc.

% http://en.wikipedia.org/wiki/Disjunctive_normal_form
\begin{definition}
[\Acrlong{dnf}
{\definitionsource{Crama2011}}]
% Definition 1.10
% Dubovsky2012: Definitions 1.9-1.11, page 3
A \definiendum{literal} is
a variable ($x_i$) or its negation ($\nott{x_i}$).
A \definiendum{term} is
an elementary conjunction of literals,
that is a conjunction in which every variable appears
at most once.
A \definiendum{\acrfull{dnf}} Boolean formula
is a disjunction of terms.
We will often view terms as sets of literals
and \acrshort{dnf} formulas as sets of terms.
\end{definition}

% http://en.wikipedia.org/wiki/Valuation_%28logic%29
A \acrshort{dnf} formula $\mathcal{F}$ is
a \definiendum{\acrshort{dnf} representation}
of a Boolean function $f$
if and only if
$\fora{v \in \booldom^n}{f(v) = 1 \iff
\valuation{\mathcal{F}}{v} \equiv 1}$,
where $\valuation{\mathcal{F}}{v}$ is the formula
given by substituting
each occurrence of a variable $x_i$ in $\mathcal{F}$
with the value $\bit{v}{i}$
for every $i \in \curly{1, \ldots, n}$.

\begin{definition}
[Minimum \acrshort{dnf} representation
{\definitionsource{Schieber2005154,Crama2011}}]
% Section 1, Section 3.3
A \acrshort{dnf} representation of function $f$ is
\definiendum{minimum}
if and only if
there is no \acrshort{dnf} representation of $f$
with fewer terms.

We shall denote the size of
a minimum \acrshort{dnf} representation
of function $f$
by $\apply{dnf}{f}$.
\end{definition}

Note that a function may have more than one minimum
\acrshort{dnf} representation.

The focus of this thesis is
minimizing the number of terms
of \acrshort{dnf} representations
of certain Boolean functions.

\section{Ternary vectors}

\begin{definition}
[Ternary vector
{\definitionsource{Schieber2005154}}] % [p.~156]
A \definiendum{ternary vector}
is a vector over the set
$\{0, 1, \phi\}$.
\end{definition}

There's a natural bijection between ternary vectors
and \acrshort{dnf} terms
(elementary conjunctions of literals).
Each of the $n$ components of a ternary vector $T$
corresponds to an occurrence (or its absence)
of one of the $n$ variables of a term $C$:
\begin{center}
$$\begin{array}{cc}
\bit{T}{i} & C \cap \curly{x_i, \overline{x_i}} \\
\hline
\phi & \emptyset \\
1 & \curly{x_i} \\
0 & \curly{\overline{x_i}}
\end{array}$$
\end{center}

Note that since $C$ is elementary,
it can not contain both $x_i$ and $\overline{x_i}$
for any $i$.

Namely,
a \emph{binary} vector
(a ternary vector that does not contain any $\phi$)
corresponds to a full term
(that is one that contains every variable),
and the ternary vector $\rep{\phi}{n}$
corresponds to the empty term (tautology).
% TODO: Consider defining "more qualified", "most qualified", "least qualified".

It is easy to see that a set of ternary vectors,
each of length $n$,
corresponds to a \acrshort{dnf} \emph{formula}
on $n$ variables.

\begin{definition}
[Spanning
{\definitionsource{Schieber2005154}}] % [p.~156]
\label{def:spans}
A ternary vector $T$ of length $n$ \definiendum{spans}
a binary vector $x$ of length $n$
if and only if
$\fora{i \in \curly{1, \ldots, n}}
{\bit{T}{i} = \phi \text{ or } \bit{T}{i} = \bit{x}{i}}$.

A set of ternary vectors $\mathcal{T}$
\definiendum{spans}
a binary vector $x$
if and only if
some vector in $\mathcal{T}$ spans $x$.
%$\ex{T \in \mathcal{T}}
%{T \text{ spans } x}$.
\end{definition}

The $i$-th symbol of a ternary vector
constrains the $i$-th symbol of the spanned binary vector
(or, equivalently, the valuation of the $i$-th variable).
If $\bit{T}{i}$ is a \quot{fixed bit}
(that is $\bit{T}{i} \in \curly{0, 1}$),
the spanned binary vector $x$
must have the same value
in its $i$-th position,
that is $\bit{x}{i} = \bit{T}{i}$.
If $\bit{T}{i}$ is the \quot{don't care symbol} $\phi$,
% "don't" is used in Schieber2005
the spanned binary vector
may have any value in its $i$-th position.

In the correspondence between ternary vectors and terms,
spanning corresponds to satisfying.
The ternary vector $T$ of length $n$ spans $x$
if and only if
$C(x) \equiv 1$,
where $C$ is the term on $n$ variables
that corresponds to $T$.
Similarly,
it is easy to see that
a set of ternary vectors $\mathcal{T}$ spans $x$
if and only if
$\mathcal{F}(x) \equiv 1$,
where $\mathcal{F}$ is the \acrshort{dnf} formula
on $n$ variables
that corresponds to $\mathcal{T}$.

\begin{definition}[Spanned set]
Let $T$ be a ternary vector of length $n$.
We denote the set of points spanned by $T$
as $span(T)$:

\begin{equation*}
span(T) = \curly{
x \in \booldom^n |
T\text{ spans }x
}
\end{equation*}
\end{definition}

Note that a ternary vector
with $m$ $\phi$-positions
spans $2^m$ binary vectors.

We'll also use a natural extension of spanning
to sets of ternary vectors --
if $\mathcal{T}$ is a set of ternary vectors, then

\begin{equation*}
span(\mathcal{T}) =
\bigcup_{T \in \mathcal{T}} span(T)
\end{equation*}

\begin{definition}[Exact spanning]
A ternary vector $T$ of length $n$
\definiendum{spans exactly}
a set of $n$-bit binary vectors $S$
if and only if
$span(T) = S$.

$T$ \definiendum{spans exactly}
an $n$-ary Boolean function $f$
if and only if
$span(T) = TP(f)$.
\end{definition}

Again,
we will generalize \definiendum{exact spanning}
to sets of ternary vectors
-- the set of ternary vectors $\mathcal{T}$
\definiendum{spans exactly} $S$
if and only if
$span(\mathcal{T}) = S$.

\begin{definition}
[Spanning set
{\definitionsource{Schieber2005154}}] % [p.~156]
Let $f$ be an $n$-ary Boolean function.
The set $\mathcal{T}$ of $n$-bit ternary vectors
is a \definiendum{spanning set} of $f$
if and only if
it $\mathcal{T}$ spans exactly $f$,
that is
$span(\mathcal{T}) = TP(f)$.
\end{definition}

In other words,
a spanning set of a function
spans all of its true points
and none of its false points.

Note that spanning sets of a function
correspond to \acrshort{dnf} representations
of the function.
The correspondence preserves size
(number of ternary vectors and terms, respectively),
so a minimum spanning set
corresponds to a minimum \acrshort{dnf} representation.

\begin{definition}
[Disjoint spanning set
{\definitionsource[Section 4]{Schieber2005154}}] % [p.~165]
A spanning set $\mathcal{T}$ is \definiendum{disjoint}
if and only if the spanned sets of its vectors
do not overlap,
that is
\[
\fora{T_i, T_j \in \mathcal{T}}{T_i = T_j \text{ or }
span(T_i) \cap span(T_j) = \emptyset}
\]
\end{definition}

In terms of \acrshort{dnf},
disjoint spanning sets
correspond to \acrshort{dnf} formulas
such that every valuation satisfies at most one term.

We have introduced two equivalent representations
of Boolean functions --
one based on ternary vectors
and the other being the \acrshort{dnf} representation.
\Cref{table:representations}
shows the corresponding terms side by side.
% Note the inconsistent use of "term" here and in the table
% caption.

\begin{table}[h]
\centering
\begin{tabular}{ll}
ternary vector & term \\
ternary vector set & \acrshort{dnf} formula \\
binary vector & variable valuation \\
ternary vector \emph{spans} a binary vector &
valuation \emph{satisfies} a term \\
ternary vector set \emph{spans} a binary vector &
valuation \emph{satisfies} a \acrshort{dnf} formula
\end{tabular}
\caption{
Corresponding terms used
in Boolean function representations
}
\label{table:representations}
\end{table}

\begin{definition}[Complement
{\definitionsource{Schieber2005154}}] % [p.~156]
\label{def:complement}
The complement of a binary symbol $\alpha$
($\alpha \in \booldom$),
denoted $\compl{\alpha}$,
is $1 - \alpha$.

We get the complement of a binary vector $x$
by flipping all of its bits,
that is $\compl{x}
= (\compl{x_1}, \ldots, \compl{x_n})
= \rep{1}{n} - x$.

The complement of the $\phi$ symbol is $\phi$
($\compl{\phi} = \phi$).

It follows that
we obtain the complement of a ternary vector
by flipping all of its fixed bits.

We generalize the notion of complement to sets of ternary vectors
-- if $\mathcal{T}$ is a set of ternary vectors,
then $\compl{\mathcal{T}} = \curly{\compl{T} | T \in \mathcal{T}}$.
\end{definition}

\section{Orthogonal sets and coverable functions}

Orthogonal set is a useful tool
for showing a lower bound on the size of all spanning sets of a function.
Orthogonal sets have been used both
by \citet{Schieber2005154} (implicitly)
and \citet{Dubovsky2012} (explicitly)
in the proofs of optimality of spanning algorithms.
The concept of orthogonal sets was also studied
in \citet{Cepek2012},
albeit using a different notation
(see \cref{remark:cepeknotation} below).

\begin{definition}
[Orthogonality
{\definitionsource[p.~6]{Dubovsky2012}}] % Definition 1.15
\label{def:orthogonal}
Let $x$ and $y$ be true points of a Boolean function $f$
(that is $f(x) = f(y) = 1$).
The vectors $x$ and $y$ are \definiendum{orthogonal}
with respect to $f$
if and only if
every ternary vector that spans both $x$ and $y$
spans some false point of $f$.
\todonote[inline]{The definition requires $x$ and $y$ to be different to be orthogonal.}

A set of true points $S$ is \definiendum{orthogonal}
if and only if
all the vectors in $S$ are pairwise
orthogonal.\footnote{This notion of orthogonality is not related to orthogonal \acrshort{dnf} formulas \citep[Definition 1.13]{Crama2011}.}
\todonote[inline]{
$S$ is orthogonal:
$$
\fora{x,y \in S}{x \neq y \Rightarrow
\fora{T}{T \text{ spans } x,y \Rightarrow
\ex{z}{\binand{f(z) = 0}{T \text{ spans } z}}}
}
$$
All pairs of points in $S$ have a common separating false point (equivalent to orthogonality):
$$
\fora{x,y \in S}{x \neq y \Rightarrow
\ex{z}{\binand{f(z) = 0}{\fora{T}{T \text{ spans } x,y
\Rightarrow T \text{ spans } z}}}
}
$$
All points in $S$ have a common separating false point:
$$
\ex{z}{
\binand{f(z)=0}{
\fora{x,y \in S}{x \neq y \Rightarrow
\fora{T}{T \text{ spans } x,y
\Rightarrow T \text{ spans } z}
}
}
}
$$
}

We shall denote the size of
a maximum orthogonal set
of function $f$
with \definiendum{$\apply{ortho}{f}$}.
\end{definition}

The following proposition appears
in \citet{Dubovsky2012}.
It was also shown for falsepoint
essential sets in \acrfull{cnf} terminology
in \citet{Cepek2012}.
To have the thesis self-contained,
we include the proof here as well.

\begin{theorem}
[$\apply{ortho}{f} \leq \apply{dnf}{f}$
{\theoremsource[Observation 1.1]{Dubovsky2012}} % p.~6
%{\theoremsource[Corollary 6.12]{Boros2010}}] % p.~89
{\theoremsource[Theorem 2.8, Corollary 3.2]{Cepek2012}}] % p.~368
\todonote{Also appears in Berge: Graphs and Hypergraphs, Theorem 5, pp. 420-42.}
\label{theorem:orthodnf}
For any Boolean function,
the size of its maximum orthogonal set
is at most as great
as the size of its minimum \acrshort{dnf} representation.
\end{theorem}

\begin{proof}
We will prove the statement by contradiction.
Let $f$ be a Boolean function such that
$\apply{ortho}{f} > \apply{dnf}{f}$.
Let $V$ be an orthogonal set of size $\apply{ortho}{f}$
and $\mathcal{T}$ be any spanning set
of size $\apply{dnf}{f}$
(such spanning set exists because of the correspondence
between \acrshort{dnf} representations and spanning sets).
Since $\size{\mathcal{T}} < \size{V}$
and $\mathcal{T}$ spans $TP(f) \supseteq S$,
there must be a ternary vector $T \in \mathcal{T}$
that spans
two distinct vectors $x, y \in V$.
However,
since $V$ is orthogonal,
$x$ is orthogonal to $y$,
so $T$ necessarily spans a false point,
which contradicts the premise that $\mathcal{T}$
is a feasible spanning set of $f$.
\end{proof}

\begin{remark}
\label{remark:cepeknotation}
Orthogonal sets were studied in \citet{Cepek2012}.
\citeauthor{Cepek2012}, however,
use the notion of a falsepoint essential set
for the case of \acrlong{cnf}
\todo{Don't break page in \acrlong{cnf}.}
which in a \acrshort{dnf} setting
translates into a truepoint essential set.
A truepoint essential set $\mathcal{E}(x)$
associated with a true point $x$
of a given function $f$
is a set of implicants
\todonote{\quot{Implicant} is not defined in the thesis.}
of $f$ which are satisfied on $x$.
The fact that the two true points $x$ and $y$
are orthogonal then naturally corresponds to
the fact that their associated truepoint essential sets
$\mathcal{E}(x)$ and $\mathcal{E}(y)$ are disjoint.
An orthogonal set then corresponds
to a set of pairwise disjoint truepoint essential sets.
The truepoint essential set defined in this way
is in fact just a special case of a more general notion
of essential set which was introduced in \citet{Boros2010}.
The results in \citet{Cepek2012}
show that for our purpose it is enought
to consider the truepoint essential sets.
\end{remark}

We conclude that the size of any orthogonal set
is a lower bound on the size of any
(especially minimum)
\acrshort{dnf} representation.
Specifically,
if we show a \acrshort{dnf} representation of a function
and an orthogonal set of the same size,
we conclude that the representation is minimum.
%(and the orthogonal set is maximum).
The orthogonal set certifies
the optimality of the \acrshort{dnf} representation.
This is captured by the notion of coverability
defined in \citet{Cepek2012}:
\begin{definition}
[Coverability
{\definitionsource[Definition 2.9]{Cepek2012}}] % [p.~368]
\label{def:coverable}
Let $f$ be a Boolean function.
The function $f$ is \definiendum{coverable}
if and only if
$\apply{ortho}{f} = \apply{dnf}{f}$.
\end{definition}

Informally speaking,
a coverable function is one for which
we have a simple certificate of the
optimality of a spanning set.
We will see
in \cref{chap:1interval,sec:2int2switch}
that all the $1$-interval
and $2$-switch $2$-interval functions
are coverable.
We will show examples of functions which are not coverable
in \cref{sec:3switch}.

\begin{definition}
[Separating vector
{\definitionsource[Definition 3.3]{Cepek2012}}]
\label{def:separatingvector}
Let $x$, $y$ and $z$ be $n$-bit binary vectors.
We say that $z$ \definiendum{separates} $x$ and $y$
if and only if
for every $i \in \curly{1, \ldots, n}$
we have
$\bit{z}{i} = \bit{x}{i}$ or $\bit{z}{i} = \bit{y}{i}$.
\end{definition}

The following proposition appears
in \citet[Lemma 3.7]{Cepek2012}.
We include the proof for completeness.

\begin{theorem}
[Separating false point and orthogonality]
\label{theorem:separatingfportho}
Let $x$ and $y$ be true points of function $f$.
If any false point of $f$ separates $x$ and $y$,
then $x$ and $y$ are orthogonal.
\todonote[inline]{The definition forces $x$ and $y$ to differ.}
\end{theorem}

\begin{proof}
Let $x$ and $y$ be true points of an $n$-ary function $f$.
Let $z$ be a false point of $f$
that separates $x$ and $y$.
Let an arbitrary ternary vector $T$ span $x$ and $y$.
We claim that $T$ necessarily also spans $z$.

For every position $i \in \curly{1, \ldots, n}$
such that $\bit{T}{i} \neq \phi$
we have $\bit{T}{i} = \bit{x}{i} = \bit{y}{i}$
since $T$ spans both $x$ and $y$
(see \cref{def:spans}).
Because $z$ separates $x$ and $y$,
we have
$\bit{z}{i} = \bit{x}{i} = \bit{y}{i} = \bit{T}{i}$.

We conclude that $T$ spans $z$.
\end{proof}

Note that the inverse implication holds as well
\citep[Lemma 3.7]{Cepek2012},
but we will not use it in this thesis.

\section{\texorpdfstring{$k$}{k}-interval
Boolean functions}

The notation for $1$- and $2$-interval
functions was introduced by \citeauthor{Schieber2005154}
and \citeauthor{Dubovsky2012} respectively.
We shall generalize it to $k$-interval functions
for every $k \geq 0$.

\begin{definition}[$k$-interval Boolean function]
\label{def:kibf}
Let $n \geq 0$ and $k \geq 0$.
\todo{Update $k \geq 1$ to $k \geq 0$ in the rest of the text.}
Let $a_1, b_1, \ldots, a_k, b_k$ be $n$-bit binary vectors
such that $\rep{0}{n} \leq a_1 \leq b_1 < a_2
\leq \cdots < a_i \leq b_i < a_{i+1}
\leq \cdots < a_k \leq b_k \leq \rep{1}{n}$.
Then $\fnkab: \booldom{}^n \rightarrow \booldom{}$ is a function defined as follows:
$$
\fnkab(x) =
\begin{cases}
1 & \text{if } a_i \leq x \leq b_i \text{ for some } i \\
0 & \text{otherwise}
\end{cases}
$$

We call
$f^n_{\interval{a_1}{b_1}, \ldots, \interval{a_k}{b_k}}$
a \definiendum{$k$-interval Boolean function}
and the vectors $a_1, b_1, \ldots, a_k, b_k$ its
\definiendum{endpoints}.
\end{definition}

Note in particular that in case $n=0$,
the $1$-interval $0$-ary function
$\intervalfn{f}{0}{\varepsilon}{\varepsilon}$,
where $\varepsilon$ denotes the binary vector of length $0$,
is properly defined and equivalent to tautology.
Similarly note that for $k=0$
a $0$-interval function $f^n_{\emptyset}$
is equivalent to contradiction.

Note that the inequalities ensure
that the intervals $\interval{a_i}{b_i}$ for all $i$s
are non-empty and do not intersect.
The inequalities, however, allow the intervals to be adjoint.
This means that a $k$-interval function
may be equivalent to a $k'$-interval function
for some $k' < k$.
We will often find it useful to use a less ambiguous notion of the number of intervals of a function:
\begin{definition}
[Proper $k$-interval Boolean function]
\label{def:properkibf}
Let
$f^n_{\interval{a_1}{b_1}, \ldots, \interval{a_k}{b_k}}$
be a $k$-interval Boolean function.
If the endpoints satisfy the inequalities
$b_1 < a_2 - 1$, \ldots, $b_i < a_{i+1} - 1$, \ldots,
$b_{k-1} < a_k - 1$
(in other words,
there is at least one false point
between any pair of adjacent intervals),
we call
$f^n_{\interval{a_1}{b_1}, \ldots, \interval{a_k}{b_k}}$
a \definiendum{proper $k$-interval Boolean function}.
\end{definition}

\todomaybe[inline]{Add an example with a picture.}

%Note that if $f$ is proper $k$-interval,
%it is also $k$-interval,
%and that if $f$ is $k$-interval
%and $g$ is \emph{proper} $l$-interval
%for $l > k$,
%then $f$ and $g$ differ.

Especially,
$\fnab$ is the function
whose true points form the interval $\interval{a}{b}$.

\Cref{tab:exampleintfns}
shows examples of interval functions.

\begin{table}[h]
\centering
\begin{tabular}{lll}
Interval function & Boolean formula & Description \\
\hline
%$\intervalfn{f}{2}{11}{11}$ &
%$x_1 \wedge x_2$ & Conjunction \\
%$\intervalfn{f}{2}{01}{11}$ &
%$x_1 \vee x_2$ & Disjunction \\
$\intervalfn{f}{n}{\rep{1}{n}}{\rep{1}{n}}$ &
$\bigwedge_i{x_i}$ & Conjunction \\
$\intervalfn{f}{n}{\rep{0}{n-1} 1}{\rep{1}{n}}$ &
$\bigvee_i{x_i}$ & Disjunction \\
$\intervalfn{f}{n}{\rep{0}{n}}{\rep{1}{n}}$ &
$1$ (an empty term) & Tautology \\
$\intervalfn{f}{n}{\rep{0}{n-1} 1}{\rep{1}{n-1} 0}$ &
$\nott{x_1 \ldots x_n} \wedge
\nott{\nott{x_1} \ldots \nott{x_n}}
\equiv \bigvee_{i \neq j}{x_i \nott{x_j}}$ &
Non-equivalence \\
$f^n_{\interval{\rep{0}{n}}{\rep{0}{n}},
\interval{\rep{1}{n}}{\rep{1}{n}}}$ &
$x_1 \ldots x_n \vee \nott{x_1} \ldots \nott{x_n}$ &
Equivalence
\end{tabular}
\caption{Examples of interval functions}
\label{tab:exampleintfns}
\end{table}

\begin{definition}
[$\intervalfns{k}$]
\label{def:intervalfns}
For any $k \geq 0$,
$\intervalfns{k}$ is the class
of all proper $k$-interval Boolean functions,
each of them represented by its endpoints.
\todoforkucera{Is this a correct definition? Isn't the representation requirement too vague or unnecessary?}
\end{definition}

\section{Minimization of \texorpdfstring{$k$}{k}-interval functions}

\begin{problem}
[Boolean minimization]
\label{problem:minimization}
Let $F$ be a class of Boolean functions.
Then the problem $\minimization{F}$ is defined as follows:
Given a Boolean function $f \in F$,
find a minimum spanning set of $f$.
\end{problem}

Having defined all the necessary terms,
let's present the central problem
we will discuss throughout the thesis:
\begin{problem}
[Minimization of proper $k$-interval functions]
\label{problem:allintervalfnsminimization}
The problem $\minimization{\intervalfns{k}}$ is defined as follows:
Given the endpoints
of a proper $k$-interval function $f$
for a fixed $k \geq 0$,
find a minimum spanning set of $f$.
\end{problem}

Note that since the problem
of general Boolean minimization
is $\Sigma_2$-hard \citep{Umans1998}
and every Boolean function is $k$-interval for some $k$,
we do not expect to be able to find
a general polynomial minimization algorithm,
that is one that is polynomial
with respect to both $k$ (number of intervals) and $n$ (function arity).
However,
an algorithm polynomial with respect to $n$ still may exist and would be interesting to show.
\todoforkucera{Is there a $\Sigma_2$-hard problem that has a parametrized polynomial solution?}
\todo{Mention in Conclusion.}

We will also touch upon the alternative version
of the minimization problem
that involves disjoint representations:
\begin{problem}
[Disjoint minimization of proper $k$-interval functions]
\label{problem:disjointminimum}
Given the endpoints
of a proper $k$-interval function $f$,
find a minimum disjoint spanning set of $f$.
\end{problem}

\section{\texorpdfstring{$l$}{l}-switch Boolean functions}

Given a $k$-interval function,
apparently the most important values are the endpoints of the intervals.
The important property of an interval endpoint is that
it is a place where the value of the function changes
(considering proper $k$-interval functions).
We can view these values as places of switch,
where $f$ switches its value from $0$ to $1$ or back.

The notion of switch was introduced by \citeauthor{Husek2014}\footnote{
\citeauthor{Husek2014} first used the Czech term
\quot{\foreignlanguage{czech}{zlom}}
in his thesis \citep[p.~13]{Husek2014}
and later translated the term as \quot{switch}
in personal communication \citep{Husek2015}.
}
and it turns out it is very useful when studying $2$-interval functions.

\begin{definition}
[$l$-switch function
{\definitionsource[pp.~13-14]{Husek2015}}] % Definitions 2.1, 2.3
Let $l \geq 0$.
A Boolean function $f$ is \definiendum{$l$-switch}
if and only if
there are exactly $l$ input vectors $x$ such that
$f(x) \neq f(x+1)$.

We call such $x$ a \definiendum{switch} of $f$.
\end{definition}

\begin{example}
[Switches and intervals]
\label{example:switchesintervals}
Depending on $f(0)$,
an $l$-switch Boolean function $f$
is proper $\ceil*{\frac{l}{2}}$-interval
in case $f(0) = 0$,
% f(0) = 0:
% 1-switch: 1-interval
% 2-switch: 1-interval
% 3-switch: 2-interval
% 4-switch: 2-interval
% l-switch: \ceil{l/2}-interval
or proper $\ceil*{\frac{l+1}{2}}$-interval
in case $f(0) = 1$.
% f(0) = 1:
% 1-switch: 1-interval
% 2-switch: 2-interval
% 3-switch: 2-interval
% 4-switch: 3-interval
% l-switch: \ceil{(l+1)/2}-interval

Equivalently,
a proper $k$-interval $n$-ary function is:
\begin{itemize}
\item
$(2k)$-switch in case $f(\rep{0}{n}) = f(\rep{1}{n}) = 0$
\item
$(2k-1)$-switch in case $f(\rep{0}{n}) \neq f(\rep{1}{n})$
\item
$(2k-2)$-switch in case $f(\rep{0}{n}) = f(\rep{1}{n}) = 1$
\end{itemize}
\end{example}

\todomaybe[inline]{Show that complementing bits preserves number of switches.}

\begin{definition}
[$\switchfns{l}$]
\label{def:switchfns}
For any $l \geq 0$,
$\switchfns{l}$ is the class
of all $l$-switch Boolean functions.
\todonote{Note that we don't specify any representation here. This is due to the fact that we only use $\switchfns{l}$ in intersection with $\intervalfns{k}$, which requires interval endpoint representation.}
\end{definition}
