\chapter*{Conclusion}
\addcontentsline{toc}{chapter}{Conclusion}

\todo[inline]{Cite sources and reference sections throughout this chapter.}

The central question of this thesis
is finding minimum spanning sets
of $k$-interval Boolean functions
for general $k$.

\citeauthor{Schieber2005154} showed a polynomial algorithm
that minimizes spanning sets of $1$-interval functions
\citep[Section 3]{Schieber2005154}.
\citeauthor{Schieber2005154}'s proof of optimality
of the algorithm
is based on the fact that every $1$-interval function
is coverable.

While \citeauthor{Dubovsky2012} showed
a polynomial minimization algorithm
for the class of $2$-switch $2$-interval functions
\citep[Section 4]{Dubovsky2012},
we presented a simpler algorithm
solving the same problem.
The algorithm presented in \cref{sec:2int2switch}
reduces every $2$-switch $2$-interval function
to a $1$-interval function.
The proofs of optimality of both of these algorithms
are based on the fact that every $2$-switch $2$-interval
function is coverable.

\citeauthor{Dubovsky2012} showed
a $3$-switch $2$-interval function
that is not coverable
\citep[p.~32]{Dubovsky2012}.
We extended his result
in \cref{sec:3switch}
by showing
a non-coverable $l$-switch function for every $l \geq 3$.
These functions include some $k$-interval function
for every $k \geq 2$.
This suggests that a possible future algorithm
that optimally spans a whole class
of $k$-interval functions
for any $k \geq 2$
is going to require a different approach
for proving optimality.

In \cref{chap:2kapprox}
we presented
a polynomial-time
% http://mathworld.wolfram.com/PolynomialTime.html
$2k$-approximation algorithm
for minimizing spanning sets
of $k$-interval functions for general $k$.
The algorithm naturally extends
\citeauthor{Schieber2005154}'s suffix-prefix
approximation algorithm
for disjoint spanning sets of $1$-interval functions
\citep[Section 6]{Schieber2005154}.

In \cref{chap:betterapprox}
we proposed an improved approximation algorithm
for $k$-interval functions.
This algorithm has the approximation ratio $1$
on $1$-interval functions
and the approximation ratio $2$
on $2$-interval functions.
We continued to show an upper bound $2k$
and a lower bound $2k-2$
on the approximation ratio
for every $k$ that is of the form $2^{l-1}+1$
for some $l$.
These bounds are especially interesting
because their ratio converges to $1$
for large $k$.

We mentioned in \cref{sec:1intervaldiscussion}
that \citeauthor{Schieber2005154} also introduced
a polynomial algorithm that finds
a minimum disjoint spanning set
representing a $1$-interval function
\citep[Section 4]{Schieber2005154}.
Disjoint spanning sets can be
especially useful
in automatic generating of test patterns
\citep{Schieber2005154}.
It is worth noting that
the $2k$ approximation algorithm we described
in \cref{chap:2kapprox} produces a disjoint spanning set
for any $k$-interval function.

An interesting question left to be answered
is whether there is a polynomial minimization algorithm
for any class of all $k$-interval functions
for $k \geq 2$.
Such algorithm would require an innovative approach
to proving optimality
since all the proofs of optimality
mentioned in this thesis depend on coverability
of the spanned functions.
An approximation algorithm with an approximation ratio
lower than $2k-2$ for all $k \geq 2$
would also be an interesting improvement.
