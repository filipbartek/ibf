\chapter*{Conclusion}
\addcontentsline{toc}{chapter}{Conclusion}

\todo[inline]{Cite sources and reference sections throughout this chapter.}

The central question of this thesis
is finding minimum spanning sets
of $k$-interval Boolean functions
for general $k$.

\citeauthor{Schieber2005154} showed a polynomial algorithm
that minimizes spanning sets of $1$-interval functions
\citep[Section 3]{Schieber2005154}.
\citeauthor{Schieber2005154}'s proof of optimality
of the algorithm
is based on the fact that every $1$-interval function
is coverable.

While \citeauthor{Dubovsky2012} showed
a polynomial minimization algorithm
for the class of $2$-switch $2$-interval functions
\citep[Section 4]{Dubovsky2012},
we presented a simpler algorithm
solving the same problem.
The algorithm presented in \cref{sec:2int2switch}
reduces every $2$-switch $2$-interval function
to a $1$-interval function.
The proofs of optimality of both of these algorithms
are based on the fact that every $2$-switch $2$-interval
function is coverable.

\citeauthor{Dubovsky2012} showed
a $3$-switch $2$-interval function
that is not coverable
\citep[p.~32]{Dubovsky2012}.
We extended his result
in \cref{sec:3switch}
by showing
a non-coverable $l$-switch function for every $l \geq 3$.
These functions include some $k$-interval function
for every $k \geq 2$.
This suggests that a possible future algorithm
that optimally spans a whole class
of $k$-interval functions
for any $k \geq 2$
is going to require a different approach
for proving optimality.

In \cref{chap:2kapprox}
we presented
a polynomial-time
% http://mathworld.wolfram.com/PolynomialTime.html
$2k$-approximation algorithm
for minimizing spanning sets
of $k$-interval functions for general $k$.
The algorithm naturally extends
\citeauthor{Schieber2005154}'s suffix-prefix
approximation algorithm
for disjoint spanning sets of $1$-interval functions
\citep[Section 6]{Schieber2005154}.

In \cref{chap:betterapprox}
we proposed an improved approximation algorithm
for $k$-interval functions.
This algorithm has the approximation ratio $1$
on $1$-interval functions
and the approximation ratio $2$
on $2$-interval functions.
We continued to show an upper bound $2k$
and a lower bound $2k-2$
for $k$ that is of the form $2^{l-1}+1$ for some $l$.
These bounds are especially interesting
because their ratio converges to $1$
for large $k$.

Disjoint spanning sets have found use in generating test data,
\todo{Cite source.}
since they allow fast uniform polling of true points.
\citeauthor{Schieber2005154} showed
a polynomial algorithm that minimizes
disjoint spanning sets
of $1$-interval functions.
We have shown a $2k$-approximation algorithm
for minimization of disjoint spanning sets of $k$-interval Boolean functions.

\todo[inline]{Outdated text follows.}

We have shown that
with respect to finding
a minimum \acrshort{dnf} representation,
$2$-switch $2$-interval functions
can be reduced to $1$-interval functions.

Then we showed that for every $l \geq 3$,
there is an $l$-switch function that is not coverable,
extending \citeauthor{Dubovsky2012}'s result
about $3$-switch functions.
This suggests that possible future algorithms
that optimally span multi-interval functions
are going to require a different approach
for proving optimality.

To remedy the difficulty with spanning multi-interval
functions,
we introduced a $2k$-approximation algorithm
for minimizing \acrshort{dnf} representations
of $k$-interval functions.
The algorithm naturally extends
\citeauthor{Schieber2005154}'s suffix-prefix
approximation algorithm
for disjoint spanning sets of $1$-interval functions
\citep[section 6]{Schieber2005154}.
Our algorithm produces disjoint spanning sets
and has the approximation ratio of $2k$
for the problem of finding
a minimum disjoint representation as well.

We proposed an improved approximation algorithm for the problem of minimization of $k$-interval Boolean functions.
We continued to analyze its approximation ratio.
We managed to find an upper bound $2k$ and a lower bound $2(k-1)$
for $k = 2^{l-1}+1$ for some $l \geq 1$.

The $2k$-approximation algorithm presented in this thesis
is constructed so that we can use the orthogonal sets
for proving the approximation ratio.
There may be a better approximation algorithm
for $k$-interval functions
\todo{We may have shown a better one in \texttt{betterapprox}.}
and searching for it could provide an interesting insight
in proving lower bounds for minimum \acrshort{dnf}
representation size of the functions in question.

\todonote[inline]{Collected results follow.}

Note that since the problem
of general Boolean minimization
is $\Sigma_2$-hard \citep{Umans1998}
and every Boolean function is $k$-interval for some $k$,
we do not expect to be able to find
a general polynomial minimization algorithm,
that is one that is polynomial
with respect to both $k$ (number of intervals) and $n$ (function arity).
However,
an algorithm polynomial with respect to $n$ still may exist and would be interesting to show.
