\chapter*{Conclusion}
\addcontentsline{toc}{chapter}{Conclusion}

\todo[inline]{
Make a section about disjoint representations, collect (review) the results from the thesis. Can be done in Conclusion.
When we discuss Scheiber's results, mention disjoint dnfs and the results - Sheiber has two algs. and results.
Motivation: generating test data for software testing - esp. with disjoint dnfs - uniform polling
}

We have shown that
with respect to finding
a minimum \acrshort{dnf} representation,
$2$-switch $2$-interval functions
can be reduced to $1$-interval functions.

Then we showed that for every $l \geq 3$,
there is an $l$-switch function that is not coverable,
extending \citeauthor{Dubovsky2012}'s result
about $3$-switch functions.
This suggests that possible future algorithms
that optimally span multi-interval functions
are going to require a different approach
for proving optimality.

To remedy the difficulty with spanning multi-interval
functions,
we introduced a $2k$-approximation algorithm
for minimizing \acrshort{dnf} representations
of $k$-interval functions.
The algorithm naturally extends
\citeauthor{Schieber2005154}'s suffix-prefix
approximation algorithm
for disjoint spanning sets of $1$-interval functions
\citep[section 6]{Schieber2005154}.
Our algorithm produces disjoint spanning sets
and has the approximation ratio of $2k$
for the problem of finding
a minimum disjoint representation as well.

We have shown an improved approximation algorithm for the problem of minimization of $k$-interval Boolean functions.
We continued to analyze its approximation ratio.
We managed to find an upper bound $2k$ and a lower bound $2(k-1)$
for $k = 2^{l-1}+1$ for some $l \geq 1$.

Some interesting questions are still left to be answered
regarding multi-interval Boolean functions.
Since general Boolean minimization is
$\Sigma_2^p$-hard \citep{Umans1998},
there must
\todo{No -- there may be an approximation scheme. For example, if the complexity is $n^k$, there's no $k$ for which the problem is hard, while the problem remains hard in general.}
be some $k$ such that minimization
of $k$-interval functions is $\Sigma_2^p$-hard,
and a $k'$ for which minimization is NP-hard.
\citeauthor{Schieber2005154} showed that such $k' > 1$,
which is the only bound shown so far.
There is a substantial difference
between $1$- and $2$-interval functions
since the former are coverable in general,
while the latter are not.
This prevents us from using a similar approach
to minimizing $2$-interval functions.
I suspect this difference could correspond to
a difference in computational complexity of the problem.
\todo{Is it a good idea to "suspect" in a thesis?}
\todomaybe{Suspect more explicitly, e.g. $3$-switch functions are NP-hard.}

The $2k$-approximation algorithm presented in this thesis
is constructed so that we can use the orthogonal sets
for proving the approximation ratio.
There may be a better approximation algorithm
for $k$-interval functions
\todo{We may have shown a better one in \texttt{betterapprox}.}
and searching for it could provide an interesting insight
in proving lower bounds for minimum \acrshort{dnf}
representation size of the functions in question.

\todonote[inline]{Collected results follow.}

Note that since the problem
of general Boolean minimization
is $\Sigma_2$-hard \citep{Umans1998}
and every Boolean function is $k$-interval for some $k$,
we do not expect to be able to find
a general polynomial minimization algorithm,
that is one that is polynomial
with respect to both $k$ (number of intervals) and $n$ (function arity).
However,
an algorithm polynomial with respect to $n$ still may exist and would be interesting to show.
