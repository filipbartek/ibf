\chapter{Interval decomposition approximation algorithm
for \texorpdfstring{$k$}{k}-interval functions}
\label{chap:betterapprox}

The $2k$-approximation
suffix-prefix decomposition algorithm
introduced in \autoref{chap:2kapprox}
can be naturally improved
by spanning each of the $k$ intervals
of the given $k$-interval function
optimally using the minimization algorithm
introduced in \autoref{chap:1interval}
(originally in \citet{Schieber2005154}).
We will call this improved algorithm
\quot{interval decomposition algorithm}.

Since the interval decomposition algorithm
clearly performs at least as well
as the suffix-prefix decomposition algorithm,
it is $2k$-approximation as well.
\citeauthor{Dubovsky2012} has managed
to prove that on $2$-interval functions,
the interval decomposition algorithm
has an approximation ratio $2$
\citep[p.~39]{Dubovsky2012}, % Theorem 4.2
while the suffix-prefix decomposition algorithm
has a tight approximation ratio $4$
on $2$-interval functions
(see \autoref{theorem:2kapproxtight}).
While this might lead us to expect
that the interval decomposition algorithm
performs significantly better in general,
we will show in this chapter that its approximation ratio
converges to $2k$ for large $k$
(and namely that it is strictly larger than $k$
for $k > 2$).
\todo{Do we really prove this for all $k$s? Try to!}

\section{Algorithm description}

\begin{description}
\item[Input] Numbers $a_1, b_1, \ldots, a_k, b_k$
encoded as binary vectors of length $n$
that satisfy the inequalities
$\rep{0}{n} \leq a_1$,
$a_1 \leq b_1$,
$b_1 < a_2 - 1$,
\ldots,
$a_i \leq b_i$,
$b_i < a_{i+1} - 1$
(for $i \in \curly{1, \ldots, k-1}$),
\ldots,
$b_{k-1} < a_k - 1$,
$a_k \leq b_k$,
$b_k \leq \rep{1}{n}$.

Such numbers are endpoints of intervals
of a proper $k$-interval Boolean function.

\item[Output] A set of ternary vectors.
\todomaybe{Add \quot{that spans
$\fnkab$}.}

\item[Procedure]
The algorithm spans each of the intervals
$\interval{a_i}{b_i}$ separately.
For every $i$,
let $\mathcal{T}_i$ be the spanning set of the interval
$\interval{a_i}{b_i}$ returned by the $1$-interval
optimization algorithm
introduced in \autoref{chap:1interval}.
The algorithm returns
$\mathcal{T} = \bigcup_{i=1}^k{\mathcal{T}_i}$.
\end{description}

Since the algorithm
spans each of the $k$ intervals separately and optimally,
while the suffix-prefix decomposition algorithm
spans each of them separately and possibly non-optimally,
the approximation ratio of $2k$
(as shown in \autoref{theorem:2kapproxratio})
holds for the interval decomposition algorithm as well.
However,
the ratio $2k$ need not be tight
for interval decomposition algorithm.

\citeauthor{Dubovsky2012} has shown
that the approximation ratio is strictly better than $2k$
for $k = 2$,
as the algorithm is $2$-approximation
in this case \citep[p.~39]{Dubovsky2012}. % Theorem 4.2
Similarly,
it is obvious that the algorithm
gives an optimal result in case $k = 1$
(which corresponds to approximation ratio $1$).

\section{Approximation ratio converges
to \texorpdfstring{$2k$}{2k}}

In this section
we will construct a class of Boolean functions
on which the interval decomposition algorithm
performs strictly worse than $k$-approximately
for $k > 2$.
\todo{Do we really do that for all $k$?}
We will show that the approximation ratio converges to $2k$
for large $k$.

\subsection{Approximation-hard functions}

For every $l \geq 1$ and $n \geq l+2$,
we will construct
a proper $(2^{l-1} + 1)$-interval $n$-ary function $f_l^n$.

Let $l \geq 1$
denote the length of the interval \quot{position} prefix
and $n \geq l+2$
denote the arity of the target function $f_l^n$
(and thus the length of its points,
especially the interval endpoints).
Let $k$ denote the number of intervals of $f_l^n$.
After we define the intervals,
we will show that $k = 2^{l-1} + 1$.

All the interval endpoints have a prefix of length $l$
that specifies the \quot{position} of the interval
and a suffix of length $n - l$
that specifies the interval's \quot{shape}.
We will denote the $l$-bit \quot{position} prefix
of an $n$-bit binary vector $x$ with $p_x$
($p_x = \bits{x}{1}{l}$)
and the $(n-l)$-bit \quot{shape} suffix with $s_x$
($s_x = \bits{x}{l+1}{n}$).
Note that $x = p_x s_x$.

The \quot{shape} is identical for all the intervals
of a given $f_l^n$
-- for every $i \in \curly{1, \ldots, k}$:

\begin{itemize}
\item $s_{a_i} = \rep{0}{n-l-1} 1$
\item $s_{b_i} = \rep{1}{n-l-1} 0$
\end{itemize}

Note that for $n \geq l + 3$,
this \quot{shape} corresponds to an extreme
non-trivial endpoint combination in the $1$-interval
optimization algorithm
(namely case \quot{$b' \geq a' - 1$}
in \autoref{sec:1interval0011}),
which happens to be the case where the improvement
of the optimization algorithm
over the suffix-prefix approximation algorithm
is the largest.

We construct the \quot{position} prefixes of the endpoints
so that they
satisfy the following requirements:

\begin{enumerate}
\item \label{item:hardint1}
$p_{a_1} = p_{b_1} = \rep{0}{l}$
\item \label{item:hardintk}
$p_{a_k} = p_{b_k} = \rep{1}{l}$
\item \label{item:hardlength}
$p_{b_i} = p_{a_i} + 1$
for $i \in \curly{2, \ldots, k-1}$
\item \label{item:hardspacing}
$p_{a_{i+1}} = p_{b_i} + 1$
for $i \in \curly{1, \ldots, k-1}$
\end{enumerate}

Using these requirements,
we can inductively construct a sequence of $l$-bit numbers
$p_{a_1}, p_{b_1}, p_{a_2}, p_{b_2}, \ldots,
p_{a_k}, p_{b_k}$.
We start with
$p_{a_1} = p_{b_1} = \rep{0}{l}$
by requirement \ref{item:hardint1}.
We follow by
constructing $p_{a_{i+1}}$ from $p_{b_i}$
for $i \in \curly{1, 2, \ldots}$
using requirement \ref{item:hardspacing} and
constructing $p_{b_i}$ from $p_{a_i}$
for $i \in \curly{2, 3, \ldots}$
using requirement \ref{item:hardlength}.
Note that the $p_{a_i}$s constructed this way
for $i \in \curly{2, 3, \ldots}$
systematically
\todomaybe{Use a better term.}
exhaust odd numbers,
since $p_{a_2} = \rep{0}{l-1} 1$
(recall that we require $l \geq 1$)
and $p_{a_{i+1}} = p_{a_i} + 2$.
We stop the induction
as soon as we get to an index $k$ such that
$p_{a_k} = \rep{1}{l}$.
Since $p_{a_i}$s systematically exhaust odd numbers,
there must be such $k$.
\todomaybe{Add \quot{Moreover, $k = 2^{l-1} + 1$}, possibly with some simple analysis.}
We terminate by constructing $p_{b_k} = \rep{1}{l}$
by requirement \ref{item:hardintk}.

Note that the sequence is defined unambiguously
for a given $l$.

The construction of $p_{a_i}$ and $p_{b_i}$
together with the definition
of $s_{a_i}$ and $s_{b_i}$ define a set of points
$a_i = p_{a_i} s_{a_i}$ and $b_i = p_{b_i} s_{b_i}$
for $i \in \curly{1, \ldots, k}$.
To see that points defined this way
satisfy the requirements on endpoints
of a proper $k$-interval function
(see \autoref{def:properkibf}),
note that:

\begin{itemize}
\item $a_1 \leq b_1$ by requirement \ref{item:hardint1}
and the fact that $s_{a_1} \leq s_{b_1}$
\item $a_i \leq b_i$
for $i \in \curly{2, \ldots, k-1}$
by requirement \ref{item:hardlength}
\item $b_i < a_{i+1} - 1$
for $i \in \curly{1, \ldots, k-1}$
by requirement \ref{item:hardspacing}
and the fact that $s_{a_i} > \rep{0}{l}$
(or $s_{b_i} < \rep{1}{l}$)
\item $a_k \leq b_k$ by requirement \ref{item:hardintk}
and the fact that $s_{a_k} \leq s_{b_k}$
\end{itemize}

To see that the number of intervals $k$ is $2^{l-1} + 1$,
we will count the \quot{subtrees}
intersected by the intervals.

\begin{definition}[Subtree]
Let $p$ be a binary vector of length $l$
and $n \geq l$.
Then \definiendum{level $l$ $p$-subtree} is the interval
$\interval{p \; \rep{0}{n-l}}{p \; \rep{1}{n-l}}$.
\end{definition}

Clearly there are $2^l$ level $l$ subtrees
that partition the full interval
$\interval{\rep{0}{n}}{\rep{1}{n}}$.

The requirements
\ref{item:hardint1} and \ref{item:hardintk}
force each of the intervals
$\interval{a_1}{b_1}$ and $\interval{a_k}{b_k}$
to intersect exactly 1 level $l$ subtree
(the $\rep{0}{l}$-subtree
and the $\rep{1}{l}$-subtree respectively).

The requirement \ref{item:hardlength}
forces each of the intervals
$\interval{a_i}{b_i}$ for $i \in \curly{2, \ldots, k-1}$
to intersect exactly 2 level $l$ subtrees
(the $p_{a_i}$-subtree and the $(p_{a_i}+1)$-subtree).

The requirement \ref{item:hardspacing}
forces each adjacent pair of intervals
to intersect adjacent level $l$ subtrees
and not to intersect a common subtree.
Since by requirements
\ref{item:hardint1} and \ref{item:hardintk}
the $\rep{0}{l}$-subtree
and the $\rep{1}{l}$-subtree
are intersected by some intervals,
each of the level $l$ subtrees is intersected
by some interval.

Since there are $2^l$ level $l$ subtrees,
each of them is intersected by exactly one interval,
the first and the $k$-th interval intersect
one subtree each
and the remaining intervals intersect 2 subtrees each,
there are $2 + \frac{2^l - 2}{2} = 2^{l-1} + 1 = k$
intervals in total.

\subsection{Interval decomposition spanning set size}

Let $f_l^n =
f^n_{\interval{a_1}{b_1}, \ldots, \interval{a_k}{b_k}}$
be a $k$-interval function
constructed using the definition
shown in the previous section.
In this section we will show the size of the spanning set
returned by the interval decomposition algorithm
when run on this function.
We will denote this size with $hard_{id}(l, n)$.

The interval decomposition algorithm spans
each of the intervals separately
using the optimization algorithm
from \autoref{chap:1interval},
so $hard_{id}(l, n) = \sum_{i=1}^k{\size{\mathcal{T}_i}}$,
where $\mathcal{T}_i$ is the spanning set
of the interval $\interval{a_i}{b_i}$
returned by the optimization algorithm
for $1$-interval functions.

For every $i$,
we will analyse how the $1$-interval algorithm solves
the instance $\interval{a_i}{b_i}$.
We will divide the analysis between the intervals
such that $p_{b_i} = p_{a_i}$
and the ones such that $p_{b_i} = p_{a_i} + 1$.
We will analyse these types separately.

\subsubsection{$i \in \curly{1, k}$}

In case $i \in \curly{1, k}$,
the endpoints have common prefixes of length $l$.
Thus the instance $\interval{a_i}{b_i}$
is reduced to the instance
$\interval{s_{a_i}}{s_{b_i}}
= \interval{\rep{0}{n-l-1} 1}{\rep{1}{n-l-1} 0}$.

Recall that $n \geq l + 2$.

If $n = l + 2$,
the instance
$\interval{\rep{0}{n-l-1} 1}{\rep{1}{n-l-1} 0}
= \interval{01}{10}$
is spanned using the procedure
shown in \autoref{sec:1interval0110}.
This procedure spans the two trivial sub-intervals
$\interval{01}{01}$ and $\interval{10}{10}$
separately,
producing $2 = n-l$ ternary vectors.

If $n \geq l + 3$,
the instance
$\interval{\rep{0}{n-l-1} 1}{\rep{1}{n-l-1} 0}$
is spanned using the procedure
shown in \autoref{sec:1interval0011}.
This procedure reduces the instance
to the instance $\interval{01}{10}$,
which again contributes $2$ ternary vectors,
and adds $n-l-2$ extra ternary vectors
that span the sub-interval
$\interval{\rep{0}{n-l-2} 10}{\rep{1}{n-l-2} 01}$.
The spanning set of the interval in this case has the size
$2 + n - l - 2 = n - l$.

We conclude that in case $i \in \curly{1, k}$,
$\size{\mathcal{T}_i} = n-l$.

\subsubsection{$i \in \curly{2, \ldots, k-1}$}

In case $i \in \curly{2, \ldots, k-1}$,
recall that $p_{b_i} = p_{a_i} + 1$.
This means that $p_{a_i} = c 0 \rep{1}{m}$
and $p_{b_i} = c 1 \rep{0}{m}$
for some binary vector $c$
and number $m \in \curly{0, \ldots, l-1}$.

To see that $m \geq 1$,
recall that $p_{a_i}$ must be odd.

This means that $a_i$ and $b_i$
start with a common prefix $c$
followed by the bits $01$ in $a_i$
and $10$ in $b_i$.
In the $1$-interval optimization algorithm,
the common prefix $c$ is left out
and then the procedure
shown in \autoref{sec:1interval0110}
is applied to the remainder of the endpoints
(that is
$\interval{01 \rep{1}{m-1} s_{a_i}}
{10 \rep{0}{m-1} s_{b_i}}$).
This procedure spans the two sub-intervals
$\interval{\rep{1}{m} s_{a_i}}{\rep{1}{m+n-l}}$
and $\interval{\rep{0}{m+n-l}}{\rep{0}{m} s_{b_i}}$
separately using the suffix and prefix procedure
respectively.

Let's consider the sub-interval
$\interval{\rep{0}{m+n-l}}{\rep{0}{m} s_{b_i}}$;
the other one is symmetric.

Note that this sub-interval can be immediately reduced
to $\interval{\rep{0}{n-l}}{s_{b_i}}$
by removing the common prefix of the endpoints.

Recall that the prefix algorithm
produces one ternary vector
for each bit that is set to $1$
in $s_{b_i} + 1 = \rep{1}{n-l}$.
The returned spanning set of
$\interval{\rep{0}{n-l}}{s_{b_i}}$
thus has the size $n-l$.

Joining it with the spanning set of the complementary
sub-interval
$\interval{s_{a_i}}{\rep{1}{n-l}}$
and prepending the respective common prefixes,
we get a spanning set $\mathcal{T}_i$ of size $2(n-l)$.

\hfill

Putting the interval spanning sets together,
we get:
\begin{align*}
hard_{id}(l, n) &= \sum_{i=1}^k{\size{\mathcal{T}_i}} \\
&= 2 (n-l) + \sum_{i=2}^{k-1}{2 (n-l)} \\
&= 2 (n-l) + (k-2) 2 (n-l) \\
%&= 2 (n-l) (1 + (k-2)) \\
&= 2 (n-l) (k-1) \\
&= 2 (n-l) ((2^{l-1} + 1) - 1) \\
%&= 2 (n-l) 2^{l-1} \\
&= 2^l (n-l)
\end{align*}

\subsection{Efficient spanning set size}

In this section we will show a better spanning set
of $f^n_l$.
We will denote its size with
$\mathit{hard}_{\mathit{eff}}(l,n)$.

We will span the union of the intervals
$\interval{p \rep{0}{n-l-1} 1}{p \rep{1}{n-l-1} 0}$
for all $p \in \booldom^l$
using $n-l$ ternary vectors
and then we will add some extra vectors
to span the remaining true points.

First observe that all the false points of $f_l^n$
end with either $\rep{0}{n-l}$ or $\rep{1}{n-l}$.
\todo{Is this clear enough?}
This means we need to span all the points that end
with different $(n-l)$-bit suffixes.
We can do that with the set
$\mathcal{T}_{\mathit{out}}
= \curly{\rep{\phi}{l} s
| s \text{ is a cyclic shift of } 01 \rep{\phi}{n-l-2}}$.
\todomaybe{Write the vectors more explicitly.}
To see that $\mathcal{T}_{\mathit{out}}$
spans all the points
that do not end with $\rep{0}{n-l}$ or $\rep{1}{n-l}$,
note that for every binary vector $x$
that contains both a $0$ and a $1$
there is a position $i$
such that $\bit{x}{i} = 0$
and $\bit{x}{(i+1) \bmod (n-l)} = 1$.
Thus we have spanned the union of the intervals
$\interval{p \rep{0}{n-l-1} 1}{p \rep{1}{n-l-1} 0}$
using $n-l$ ternary vectors.

Observe that
all the true points
that end with $\rep{1}{n-l}$
start with $p_{a_i}$
for some $i \in \curly{2, \ldots, k-1}$
and that all such $p_{a_i}$ are odd numbers,
which means they end with a $1$.
In fact,
$p_{a_i}$ for $i \in \curly{2, \ldots, k-1}$
are all the $l$-bit odd numbers except $\rep{1}{l}$.
We will use the technique of cyclic shifts
to span these points efficiently.

Let $\mathcal{T}_1 = \curly{c 1 \rep{1}{n-l}
| c \text{ is a cyclic shift of } 0 \rep{\phi}{l-2}}$.
To see that $\mathcal{T}_1$ spans all the true points
that end with $\rep{1}{n-l}$,
note that the cyclic shifts of $0 \rep{\phi}{l-2}$
span exactly all the $(l-1)$-bit numbers that contain a $0$
and the vectors $c 1$
where $c$ is a cyclic shift of $0 \rep{\phi}{l-2}$
span exactly all the $l$-bit \emph{odd} numbers
that contain a $0$.
Clearly $\size{\mathcal{T}_1} = l-1$.

Similarly we construct $\mathcal{T}_0$ of size $l-1$
that spans all the true points
that end with $\rep{0}{n-l}$.

Putting the partial spanning sets together,
we get the size of a spanning set of $f_l^n$:

\begin{align*}
\mathit{hard}_{\mathit{eff}}(l,n)
&= \size{\mathcal{T}_{\mathit{out}}}
+ \size{\mathcal{T}_0} + \size{\mathcal{T}_1} \\
&= (n-l) + (l-1) + (l-1) \\
&= n + l - 2
\end{align*}

\subsection{Approximation ratio}

The spanning set shown in the previous section
gives an upper bound on the size
of the optimal spanning set:
$$
\mathit{dnf}(f^l_n)
\leq \mathit{hard}_{\mathit{eff}}(l,n)
$$

We can use this to prove that the approximation ratio
of the interval decomposition algorithm is close to $2k$
for large $k$.

The approximation ratio
of the interval decomposition algorithm
on the function $f_l^n$,
denoted $\mathit{approx}(l,n)$,
can be bounded:
$$
\mathit{approx}(l,n)
= \frac{\mathit{hard}_{id}(l,n)}{\mathit{dnf}(f_l^n)}
\geq \frac{\mathit{hard}_{id}(l,n)}
{\mathit{hard}_{\mathit{eff}}(l,n)}
= \frac{2^l(n-l)}{n+l-2}
$$

First see that for a fixed $l$ and large $n$,
the approximation ratio converges to $2^l$
or a larger number:
\begin{align*}
\mathit{approx}(l)
&= \lim_{n \rightarrow \infty}{\mathit{approx}(l,n)} \\
&\geq \lim_{n \rightarrow \infty}
{\frac{\mathit{hard}_{\mathit{id}}(l,n)}
{\mathit{hard}_{\mathit{eff}}(l,n)}} \\
&= \lim_{n \rightarrow \infty}{\frac{2^l(n-l)}{n+l-2}} \\
&= \lim_{n \rightarrow \infty}{\frac{2^l n}{n}} \\
&= 2^l
\end{align*}

In other words,
for any $l \geq 1$ and a number $a < 2^l$,
we can find $n$ such that
$\mathit{hard}_{\mathit{id}}(l,n)
\geq a \cdot \mathit{dnf}(f_l^n)$,
which corresponds to finding an instance
on which the interval decomposition algorithm performs
at least $a$ times worse than the optimum.

\begin{example}
\label{example:3inthard}
For example,
let's try to find a $3$-interval function $f$
that can be spanned using
as few as
$\mathit{dnf}(f)$ ternary vectors
while the interval decomposition algorithm uses
strictly more than $3 \cdot \mathit{dnf}(f)$ vectors
to span $f$.

We will find a combination of $l$ and $n$
such that $f_l^n$ is a $3$-interval function
and
$\mathit{hard}_{\mathit{id}}(l,n)
> 3 \cdot \mathit{dnf}(f_l^n)$.

Recall that $k = 2^{l-1}+1$,
so we need to use $l = 2$
\todomaybe{Define an inverse mapping.}
in order for $f_l^n$ to be $3$-interval.

Now let's choose the parameter $a$.
It needs to satisfy the inequalities $a > 3$
and $a < 4 = 2^l$.
Let's take for example $a = 3.5 = \frac{7}{2}$.

We need to find $n$ such that:
\begin{align*}
\mathit{hard}_{\mathit{id}}(l,n) &= 2^l(n-l) \\
&= 2^2(n-2) \\
&= 4 (n-2) \\
&= 4n - 16 \\
&\geq \frac{7}{2} \cdot \mathit{dnf}(f_2^n) \\
&= a \cdot \mathit{dnf}(f_l^n)
\end{align*}

We will actually find an $n$ that satisfies
an even stricter condition:
\begin{align*}
\mathit{hard}_{\mathit{id}}(l,n) &= 4n - 16 \\
&\geq \frac{7}{2} n \\
&= \frac{7}{2} (n+2-2) \\
&= a (n+l-2) \\
&= a \cdot \mathit{hard}_{\mathit{eff}}(l,n) \\
&\geq a \cdot \mathit{dnf}(f_l^n)
\end{align*}

Any $n \geq 32$ satisfies the inequality:

\begin{align*}
4n - 16 &\geq \frac{7}{2} n \\
8n - 32 &\geq 7n \\
n &\geq 32
\end{align*}

We conclude, for example for $n = 32$:
\begin{align*}
\mathit{hard}_{\mathit{id}}(2, 32)
%&= 4 \cdot 32 - 16 \\
%&= 128 - 16 \\
&= 112 \\
%&\geq 112 \\
%&= 7 \cdot 16 \\
%&= \frac{7}{2} 32 \\
&> 96 \\
&= 3 \cdot 32 \\
&= 3 \cdot \mathit{hard}_{\mathit{eff}}(2,32) \\
&\geq 3 \cdot \mathit{dnf}(f_2^{32})
\end{align*}

which means that the interval decomposition algorithm
uses strictly more
than $3$-times as many ternary vectors to span
the $3$-interval function
$f_2^{32}$ as the optimum.

Note that this example namely shows
that interval decomposition algorithm
is not $k$-approximative in general,
since it is not $3$-approximative
on the $3$-interval function $f_2^{32}$.
\end{example}

\todomaybe[inline]{Start a subsection here.}

In \cref{example:3inthard},
we have shown a $k$-interval function
on which the interval decomposition algorithm performs
more than $k$ times worse than optimum.
More specifically,
the algorithm produces a spanning set
of a $(k=3)$-interval function
that is at least
$\frac{7}{2}
%= \frac{7}{2} \frac{3}{3}
= \frac{7}{6} 3
= \frac{7}{6} k$
times as large as optimum.
On $3$ intervals,
we could get this error ratio arbitrarily close
to $\frac{4}{3} k$
by choosing larger $n$.
We will show that if we allow using more intervals
(that is larger $k$),
we can get the error ratio arbitrarily close to $2k$.

Let's consider $k$ a function of $l$
-- $k(l) = 2^{l-1} + 1$.

With this in mind,
for a large $l$ we get:

$$
\lim_{l \rightarrow \infty}{\frac{approx(l)}{k(l)}}
\geq \lim_{l \rightarrow \infty}{\frac{2^l}{2^{l-1}+1}}
= \lim_{l \rightarrow \infty}{\frac{2^l}{2^{l-1}}}
%= \lim_{l \rightarrow \infty}{2}
= 2
$$

In other words,
for any $b < 2$
we can find a $k$-interval function
on which the interval decomposition algorithm
performs at least $bk$ times worse than the optimum.
\todo{For large $n$...}

For demonstration,
\autoref{tab:hardsmalll} shows the lower bounds
on error ratio computed using $f_l^n$
for some small values of $l$.

\begin{table}[h]
\centering
\begin{tabular}{l|lllll}
$l$
& $k$ % 2^{l-1}+1
& $\mathit{hard}_{\mathit{id}}(l,n)$ % 2^l(n-l)
& $\mathit{hard}_{\mathit{eff}}(l,n)$ % n+l-2
& $\mathit{approx}(l) \geq$ % 2^l
& $\frac{\mathit{approx}(l)}{k(l)} \geq$
% \frac{2^l}{2^{l-1}+1}
\\
\hline
$1$ & $2$ & $2n-4$ & $n-1$ & $2$ & $1$ \\
$2$ & $3$ & $4n-8$ & $n$ & $4$ & $\frac{4}{3}$ \\
$3$ & $5$ & $8n-24$ & $n+1$ & $8$ & $\frac{8}{5}$ \\
$4$ & $9$ & $16n-64$ & $n+2$ & $16$ & $\frac{16}{9}$
\end{tabular}
\caption{Lower bound of approximation ratio using $f_l^n$
for small $l$}
\label{tab:hardsmalll}
\end{table}

\todo{lim sup ~ horni odhad limity - pres k
nebo lim pres l + justification v uvodu dukazu (lepsi) - beru jen nektera k}
