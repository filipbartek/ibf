\chapter{Better approximation is hard}
\todomaybe[inline]{Leave the chapter out, since the result is not very interesting and the proof is quite complicated. Or give just a sketch of the proof.}
\todomaybe[inline]{Use a better chapter title.}
\todo[inline]{Finish the chapter.}

\citeauthor{Dubovsky2012} has shown a $2$-approximation algorithm
for spanning general $2$-interval functions
(including $3$-switch and $4$-switch).\citep[p.~33]{Dubovsky2012}
The algorithm simply spans each of the two intervals
separately optimally and then returns the union
of these partial spanning sets.
A natural generalization of this algorithm to $k$-interval
functions spans each of the $k$ intervals separately
optimally.
It's easy to see that this algorithm performs
at least as well as the $2k$-approximation algorithm
shown in chapter \ref{chap:2kapprox}.
We'll show, however,
that for a large $k$,
the approximation ratio of the enhanced algorithm
converges to $2k$.
\todo{Consider rewording.}

In order to do this,
we will construct a class of \quot{hard} multi-interval
Boolean functions.

\todo[inline]{Finish.}
