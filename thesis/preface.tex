\chapter*{Introduction}
\addcontentsline{toc}{chapter}{Introduction}

An $n$-ary Boolean function takes an $n$-tuple of Boolean values as input
and outputs a single Boolean value.
A Boolean function can be represented in various ways.
A common way to represent Boolean functions
is the truth table,
which explicitly lists the output value for each possible input $n$-tuple.
For example,
the simple Boolean function $f^2_{\wedge}$ that realizes the binary logical operator of conjunction
can be represented by the following table:
\begin{center}
\begin{tabular}{cc|c}
$x_1$ & $x_2$ & $f^2_{\wedge}(x_1, x_2)$ \\
\hline
$0$ & $0$ & $0$ \\
$0$ & $1$ & $0$ \\
$1$ & $0$ & $0$ \\
$1$ & $1$ & $1$ \\
\end{tabular}
\end{center}

The upper index of $f^2_{\wedge}$
signifies the arity of the function.
In this case it is $2$,
meaning it is a function of two Boolean variables,
or equivalently of pairs of Boolean values,
or binary vectors of length $2$.

Other representations of Boolean functions
include
propositional formulas,
circuits,
decision diagrams
\todo{Cite proper literature, for example Wegener: The complexity of Boolean functions}
and sets of intervals.
In this thesis,
we will deal with the interval representation of Boolean functions
and relate it to propositional formula representation of a special form called \acrfull{dnf}.

The \acrshort{dnf} representation of Boolean functions
has been applied in constraint-based software
\citep{DeMillo1991} % Section IV.A, p.~906
and hardware
\citep{Lewin1995} % Section 4, p.~47
testing systems.
In these systems,
\acrshort{dnf} formulas act as characteristic functions
of constrained variable domains.
Since some of the basic constraints
reduce variable domains to small (often singular) sets of intervals,
\todo{Cite some source, or simplify and cite Schieber.}
it is interesting to look for economic \acrshort{dnf}
representations of intervals.

Boolean functions defined by intervals were introduced
by \citet{Schieber2005154}.
A pair of $n$-bit integers $a, b$
defines a $1$-interval function
$\intervalfn{f}{n}{a}{b}$:
$$
\apply{\intervalfn{f}{n}{a}{b}}{x} = 1
\iff a \leq x \leq b
$$

Note that such function
is the characteristic function of the interval
$\interval{a}{b}$.

\citeauthor{Schieber2005154} showed an efficient method
to find a minimum \acrfull{dnf} representation
of any $1$-interval Boolean function
given by a pair of endpoints \citep{Schieber2005154}.

Since the problem of Boolean minimization is in general
$\Sigma_2^p$-complete \citep{Umans1998},
there is an interesting question of how difficult it is
to minimize \acrshort{dnf} representation of
a function defined by a set of intervals.
\citeauthor{Dubovsky2012} investigated the problem
in the case of $2$-interval functions
\citep{Dubovsky2012}.

In this thesis we review the existing results regarding
minimization of \acrshort{dnf} representations of
single- and multi-interval functions,
and then present four new results:

\begin{itemize}
\item a simplified algorithm
for minimizing \acrshort{dnf} representations
of $2$-switch
$2$-interval functions (\cref{sec:2int2switch}),

\item
an argument that shows
that the technique used for proving the optimality
of the algorithms that minimize
\acrshort{dnf} representation
of $1$- and $2$-switch
(that is all $1$-interval
and $2$-switch $2$-interval)
functions
can not be used in general for
functions with $3$ or more switches
(\cref{sec:3switch}),

\item
an algorithm
that,
given the intervals of a $k$-interval function,
finds the function's \acrshort{dnf} representation
that is at worst
$2k$ times larger
than an optimal \acrshort{dnf} representation
(\cref{chap:2kapprox}) and

\item
an improved approximation algorithm
for $k$-interval functions
and show that it does not perform significantly
better than $2k$-approximation for large $k$
(\cref{chap:betterapprox}).
\end{itemize}
