\chapter{Introduction}

\todo[inline]{Remove numbering of this chapter
to follow \texttt{dp\_uprava\_en.pdf}.}

\todo[inline]{Write a proper introduction.}

\section{Definitions}

The basic terminology used in this thesis
was introduced by Schieber et al. \cite{Schieber2005154},
\dubovsky{} \cite{Dubovsky2012}
and \husek{} \cite{Husek2014, Husek2015}.

\begin{definition}[Binary vector]
In this thesis,
we'll use the term \definiendum{binary vector}
to denote a vector over the Boolean domain.
That is, $x$ is a binary vector if and only if
$\ex{n \in \nat}{x \in \booldom^n}$.
We call such $x$ an \definiendum{$n$-bit binary vector}.
\end{definition}

Note that each binary vector represents a natural number.
% TODO: Add "in binary coding" or something similar.
More specifically,
$n$-bit binary vectors represent numbers
between $0$ and $2^n - 1$.
We will not differentiate between a number
and its binary representation.
Specifically,
we'll use a lexicographic linear ordering
on binary vectors --
for $n$-bit vectors $a$ and $b$,
$a \leq b$ if and only if $a = b$ or
the \acrlong{msb}
in which $a$ and $b$ differ
is $0$ in $a$ and $1$ in $b$.
Clearly,
this ordering corresponds to the natural ordering
on the represented numbers.

\begin{definition}
[Symbol extraction]
Let $v$ be a vector of length $n$ and $1 \leq i \leq n$.
Then $\bit{v}{i}$ is the $i$-th component of $v$.

Let $1 \leq a \leq b \leq n$.
Then $\bits{v}{a}{b}$ is the subvector of $v$
that starts at $a$-th position
and ends at $b$-th position
($\bits{v}{a}{b}
= (\bit{v}{a}, \bit{v}{a+1},
\ldots, \bit{v}{b-1}, \bit{v}{b})$).
\end{definition}

Notably,
$v = (\bit{v}{1}, \ldots, \bit{v}{n}) = \bits{v}{1}{n}$.

\begin{definition}
[Symbol repetition]
Let $x$ be a symbol (for example $0$ or $1$)
and $0 \leq n$.
Then $\rep{x}{n}$ is the vector of length $n$
each component of which is equal to $x$.
\end{definition}

\begin{definition}[Boolean function]
$f: \booldom^n \rightarrow \booldom$ is
an \definiendum{$n$-ary Boolean function}.
\end{definition}

Since we will not be dealing with any non-Boolean functions,
in the remainder of the text,
I will use the terms
\quot{function} and \quot{Boolean function}
interchangeably.

% http://en.wikipedia.org/wiki/Boolean_function
Every $n$-ary Boolean function $f$ can be expressed
as a propositional formula $\mathcal{F}$ on $n$ variables
$x_1, \ldots, x_n$.
There's a natural bijection
between binary vectors of length $n$
and valuations of $n$ variables
($1$-bits in the binary vector
correspond exactly to $1$-valued variables
in the valuation).

% TODO: Add example Boolean function with logical representation etc.

% http://en.wikipedia.org/wiki/Disjunctive_normal_form
\begin{definition}[\Acrlong{dnf}]
A propositional logic formula $\mathcal{F}$
is in \definiendum{\acrfull{dnf}} if and only if
$\mathcal{F}$ is a disjunction of terms,
where \definiendum{terms}
are elementary\footnote{In an elementary conjunction
of literals,
no variable appears in both polarities.
A trivial example of a non-elementary conjunction
is $x_1 \overline{x_1}$.}
conjunctions of literals.

We'll often view terms as sets of literals
and \acrshort{dnf} formulas as sets of terms.
\end{definition}

% http://en.wikipedia.org/wiki/Valuation_%28logic%29
A \acrshort{dnf} formula $\mathcal{F}$ is
a \definiendum{\acrshort{dnf} representation}
of a Boolean function $f$
if it expresses the same function,
that is
$\fora{x \in \booldom^n}{f(x) = 1 \iff
\valuation{\mathcal{F}}{x} = 1}$.
% TODO: Rewrite; should be more natural or use the bijection explicitly.

To make our work easier,
we'll define a vector representation of a term:

\begin{definition}
[Ternary vector]
A \definiendum{ternary vector}
is a vector over the set
$\{0, 1, \phi\}$.
\end{definition}

There's a natural bijection between ternary vectors
and terms (elementary conjunctions of literals).
Each of the $n$ components of a ternary vector $T$
corresponds to the single occurrence (or its absence)
of one of the $n$ variables of a term $C$:

% TODO: Consider wrapping in `table` environment
%       and adding a caption.
\begin{center}
\begin{tabular}{cc}
$\bit{T}{i}$ & $C \cap \curly{x_i, \overline{x_i}}$ \\
\hline
$\phi$ & $\emptyset$ \\
$1$ & $\curly{x_i}$ \\
$0$ & $\curly{\overline{x_i}}$
\end{tabular}
\end{center}

Note that since $C$ is elementary,
it can not contain both $x_i$ and $\overline{x_i}$.

Namely,
a \emph{binary} vector corresponds to a full term
(that is one that contains every variable),
and the ternary vector $\rep{\phi}{n}$
corresponds to the empty term (tautology).
% TODO: Consider defining "more qualified", "most qualified", "least qualified".

It's easy to see that sets of ternary vectors of length $n$
correspond to \acrshort{dnf} \emph{formulas}
on $n$ variables.

\begin{definition}[Spanning]
A ternary vector $T$ of length $n$ \definiendum{spans}
a binary vector $x$ of length $n$
if and only if
$\fora{i \in \curly{1, \ldots, n}}
{\bit{T}{i} = \phi \text{ or } \bit{T}{i} = \bit{x}{i}}$.
\end{definition}

The $i$-th symbol of a ternary vector
constrains the $i$-th symbol of the spanned binary vector
(or, equivalently, the valuation of the $i$-th variable).
If $\bit{T}{i}$ is a \quot{fixed bit}
($\bit{T}{i} \in \curly{0, 1}$),
the spanned binary vector $x$
must have the same value
in its $i$-th position,
that is $\bit{x}{i} = \bit{T}{i}$.
If $\bit{T}{i}$ is the \quot{don't care symbol} $\phi$,
% "don't" is used in Schieber2005
the spanned binary vector
may have any value in its $i$-th position.

\begin{definition}
[Spanned set]
Let $T$ be a ternary vector of length $n$.
Set of points spanned by $T$,
denoted $span(T)$,
is defined in the following way:

\begin{equation*}
span(T) = \curly{
x \in \booldom^n |
T\text{ spans }x
}
\end{equation*}
\end{definition}

Note that a ternary vector
with $m$ $\phi$-positions
spans $2^m$ binary vectors.

We'll also use a natural extension of spanning
to sets of ternary vectors --
if $\mathcal{T}$ is a set of ternary vectors, then

\begin{equation*}
span(\mathcal{T}) =
\bigcup_{T \in \mathcal{T}} span(T)
\end{equation*}

\begin{definition}[True point]
Let $f$ be an $n$-ary Boolean function.
An $n$-bit binary vector $x$ is a \definiendum{true point}
of $f$ if and only if $f(x)  = 1$.
\end{definition}

Equivalently,
we define a \definiendum{false point} of $f$
as any point $x$
such that $f(x) = 0$.

We'll denote the set of all true points of $f$ as $TP(f)$
and the set of all false points as $FP(f)$.

\begin{definition}
[Minimum \acrshort{dnf} representation]
A \acrshort{dnf} representation of a function is
\definiendum{minimum}
if and only if
there's no representation with fewer terms.
\end{definition}

The focus of this thesis is
minimizing the number of terms
of \acrshort{dnf} representations
of certain Boolean functions.

Note that a function may have more than one minimum
\acrshort{dnf} representation.

\begin{definition}
[Spanning set]
Let $f$ be an $n$-ary Boolean function.
Set $\mathcal{T}$ of $n$-bit ternary vectors
is a \definiendum{spanning set} of $f$
if and only if
$span(\mathcal{T}) = TP(f)$.
\end{definition}

In other words,
a spanning set of a function
spans all of its true points
and none of its false points.

Note that spanning sets of a function
correspond to \acrshort{dnf} representations
of the function.
The correspondence preserves size
(number of ternary vectors and terms, respectively),
so a minimum spanning set
corresponds to a minimum \acrshort{dnf} representation.

\begin{definition}
[Disjoint spanning set]
A spanning set $\mathcal{T}$ is \definiendum{disjoint}
if and only if the spanned sets of its vectors
do not overlap,
that is
\[
\fora{T_i, T_j \in \mathcal{T}}{T_i = T_j \text{ or }
span(T_i) \cap span(T_j) = \emptyset}
\]
\end{definition}

In terms of \acrshort{dnf},
disjoint spanning sets
correspond to \acrshort{dnf} formulas
such that every valuation satisfies at most one term.

We have introduced two equivalent representations
of Boolean functions --
one based on ternary vectors
and the other being the \acrshort{dnf} representation.
Table \ref{table:representations}
shows the corresponding terms side by side.
% Note the inconsistent use of "term" here and in the table
% caption.

\begin{table}[h]
\label{table:representations}
\centering
\begin{tabular}{ll}
ternary vector & term \\
ternary vector set & \acrshort{dnf} formula \\
binary vector & variable valuation \\
ternary vector \emph{spans} a binary vector &
valuation \emph{satisfies} a term \\
ternary vector set \emph{spans} a binary vector &
valuation \emph{satisfies} a \acrshort{dnf} formula
\end{tabular}
\caption{
Corresponding terms used
in Boolean function representations
}
\end{table}

\subsection{Interval Boolean functions}

% TODO: Define comparison operators on binary vectors.
% TODO: Clarify the usage of "number" for "binary vector".

\begin{definition}
[$k$-interval Boolean function]
\label{def:kibf}
Let $a_1, b_1, \ldots, a_k, b_k$ be $n$-bit binary vectors
such that $\rep{0}{n} \leq a_1 \leq b_1 < a_2
\leq \cdots < a_i \leq b_i < a_{i+1}
\leq \cdots < a_k \leq b_k \leq \rep{1}{n}$.
% TODO: Clarify the inequalities.
Then $f^n_{\interval{a_1}{b_1}, \ldots, \interval{a_k}{b_k}}: \booldom{}^n \rightarrow \booldom{}$ is a function defined as follows:
\[
f^n_{\interval{a_1}{b_1}, \ldots, \interval{a_k}{b_k}} (x) =
\begin{cases}
1 & \text{if } a_i \leq x \leq b_i \text{ for some } i \\
0 & \text{otherwise}
\end{cases}
\]

We call
$f^n_{\interval{a_1}{b_1}, \ldots, \interval{a_k}{b_k}}$
a \definiendum{$k$-interval Boolean function}
and the vectors $a_1, b_1, \ldots, a_k, b_k$ its
\definiendum{endpoints}.
\end{definition}

Note that the inequalities ensure
that the intervals $\interval{a_i}{b_i}$
are non-empty and don't intersect.

\begin{definition}
[Proper $k$-interval Boolean function]
\label{def:properkibf}
Let
$f^n_{\interval{a_1}{b_1}, \ldots, \interval{a_k}{b_k}}$
be a $k$-interval Boolean function.
If the endpoints satisfy the inequalities
$b_1 < a_2 - 1$, \ldots, $b_i < a_{i+1} - 1$, \ldots,
$b_{k-1} < a_k - 1$
(in other words, adjacent intervals
are separated by at least one false point),
we call
$f^n_{\interval{a_1}{b_1}, \ldots, \interval{a_k}{b_k}}$
a \definiendum{proper $k$-interval Boolean function}.
\end{definition}

%Note that if $f$ is proper $k$-interval,
%it is also $k$-interval,
%and that if $f$ is $k$-interval
%and $g$ is \emph{proper} $l$-interval
%for $l > k$,
%then $f$ and $g$ differ.

Especially,
$\intervalfn{f}{n}{a}{b}$ is a function
whose true points form the interval $\interval{a}{b}$.

% TODO: Fix reference.
Table \ref{table:exampleintfns} shows
examples of interval functions.

\begin{table}[H]
\label{table:exampleintfns}
\centering
\begin{tabular}{lll}
Interval function & Propositional formula & Description \\
\hline
$\intervalfn{f}{2}{11}{11}$ &
$x_1 \wedge x_2$ & Conjunction \\
$\intervalfn{f}{2}{01}{11}$ &
$x_1 \vee x_2$ & Disjunction \\
$\intervalfn{f}{n}{\rep{0}{n}}{\rep{1}{n}}$ &
$1$ & Tautology \\
$\intervalfn{f}{n}{\rep{0}{n-1} 1}{\rep{1}{n-1} 0}$ &
$\nott{x_1 \ldots x_n} \wedge
\nott{\nott{x_1} \ldots \nott{x_n}}$ &
Non-equivalence \\
$f^n_{\interval{\rep{0}{n}}{\rep{0}{n}},
\interval{\rep{1}{n}}{\rep{1}{n}}}$ &
$x_1 \ldots x_n \vee \nott{x_1} \ldots \nott{x_n}$ &
Equivalence
\end{tabular}
\caption{Examples of interval functions}
\end{table}
