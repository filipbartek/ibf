\chapter*{Introduction}
\addcontentsline{toc}{chapter}{Introduction}

Boolean functions defined by intervals were introduced
by \citet{Schieber2005154}.
A pair of $n$-bit numbers $a, b$
defines a $1$-interval function
$\intervalfn{f}{n}{a}{b}$:

\[
\apply{\intervalfn{f}{n}{a}{b}}{x} = 1
\iff a \leq x \leq b
\]

\citeauthor{Schieber2005154} showed an efficient method
to find minimum \acrshort{dnf} representation
of any $1$-interval Boolean function
given by a pair of endpoints.

Since the problem of Boolean minimization is in general
$\Sigma_2^p$-complete \citep{Umans1998},
\todo{Does Umans show this for term-wise minimization as well? Note that \citeauthor{Cepek2012} uses $\Sigma$ with CNF representations. Shouldn't one of the problems be $\Pi_2$-hard instead?}
there is an interesting question of how difficult it is
to minimize \acrshort{dnf} representation of
a function defined by a set of intervals.
\citeauthor{Dubovsky2012} investigated the problem
for $2$-interval functions.

In this thesis we review the existing results regarding
minimization of \acrshort{dnf} representations of
single- and multi-interval functions,
and then present 3 new results:

\begin{itemize}
\item a simplified algorithm
for minimizing \acrshort{dnf} representations
of $2$-switch
\todo{Define switch in Definitions.}
$2$-interval functions (\autoref{sec:2int2switch}),
\item
show that the technique used for proving the optimality
of the algorithms that minimize
\acrshort{dnf} representation
of $1$- and $2$-switch functions
can not be used in
functions with $3$ or more switches
(\autoref{sec:3switch}) and
\item
introduce an algorithm
for $k$-interval functions
which gives a solution which is at worst
$2k$ times larger than an optimal one
(\autoref{chap:2kapprox}).
\end{itemize}
\todomaybe{Mention the "Better approximation is hard" result.}
