\chapter*{Introduction}
\addcontentsline{toc}{chapter}{Introduction}

An $n$-ary Boolean function takes an $n$-tuple of Boolean values as input
and outputs a single Boolean value.
A Boolean function can be represented in various ways.
A common way to represent Boolean functions
is the truth table,
which explicitly lists the output value for each possible input $n$-tuple \citep[Definition 1.2]{Crama2011}.
For example,
the simple Boolean function $f^2_{\wedge}$ that realizes the binary logical operator of conjunction
can be represented by the following table:
\begin{center}
\begin{tabular}{cc|c}
$x_1$ & $x_2$ & $f^2_{\wedge}(x_1, x_2)$ \\
\hline
$0$ & $0$ & $0$ \\
$0$ & $1$ & $0$ \\
$1$ & $0$ & $0$ \\
$1$ & $1$ & $1$ \\
\end{tabular}
\end{center}

The upper index of $f^2_{\wedge}$
signifies the arity of the function.
In this case it is $2$,
meaning it is a function of two Boolean variables,
or equivalently of pairs of Boolean values,
or binary vectors of length $2$.

Other representations of Boolean functions
include
Boolean formulas \citep[Definition 1.4]{Crama2011},
circuits
\citep[Definition 3.1]{Wegener1987}
and binary decision diagrams \citep[Section 1.12.3]{Crama2011}.
Another possible representation is sets of intervals,
which is the one we study in this thesis.
We will relate it to the \acrfull{dnf}
Boolean formula representation.
More specifically,
we will look for efficient ways to transform an interval representation
to an equivalent \acrshort{dnf} representation.

The \acrshort{dnf} representation of Boolean functions
has been applied in constraint-based software
and hardware
testing systems
\citep{DeMillo1991,Lewin1995}.
% Section IV.A, p.~906
% Section 4, p.~47
In these systems,
\acrshort{dnf} formulas act as characteristic functions
of constrained variable domains.
It can be especially useful to represent the constrained
domains with disjoint spanning sets,
since these allow fast uniform polling
of admissible values.
\todo{Cite source.}

Since the constrained variables are typically numeric
and comparison is one of the basic constraints,
the variable domains are often constrained
to small sets of intervals.
\todo{Cite some source, e.g.~Handbook of Constraint Programming, or simplify and cite Schieber.}
Therefore it can be interesting to look
for compact \acrshort{dnf}
representations of intervals.

The notion of
Boolean functions defined by intervals was introduced
by \citet{Schieber2005154}.
A pair of $n$-bit integers $a, b$
defines the $1$-interval function
$\intervalfn{f}{n}{a}{b}$:
$$
\apply{\intervalfn{f}{n}{a}{b}}{x} = 1
\iff a \leq x \leq b
$$

Note that such function
is the characteristic function of the interval
$\interval{a}{b}$.

\citeauthor{Schieber2005154} showed an efficient method
to find a minimum \acrfull{dnf} representation
of any $1$-interval Boolean function
given by a pair of endpoints \citep{Schieber2005154}.

Since the problem of Boolean minimization is in general
$\Sigma_2^p$-complete \citep{Umans1998},
it is an interesting question how difficult it is
to minimize \acrshort{dnf} representation of
a $k$-interval function,
that is a function defined by a set of $k$ intervals.
\citeauthor{Dubovsky2012} investigated the problem
in the case of $2$-interval functions
\citep{Dubovsky2012}.
Another useful measure
is to consider the number of switches of the function,
which correspond to the inner interval endpoints
(we will give a precise definition
in \cref{sec:lswitch}).

In this thesis we review the existing results regarding
minimization of \acrshort{dnf} representations of
single- and multi-interval functions,
and then present four new results:

\begin{itemize}
\item a simplified algorithm
for minimizing \acrshort{dnf} representations
of $2$-switch
$2$-interval functions (\cref{sec:2int2switch}),

\item
an argument that shows
that the technique used for proving the optimality
of \citeauthor{Schieber2005154}'s
$1$-interval minimization algorithm
\citep{Schieber2005154}
and \citeauthor{Dubovsky2012}'s
$2$-switch $2$-interval minimization algorithm
\citep{Dubovsky2012}
can not be used in general for
functions with $3$ or more switches
(\cref{sec:3switch}),

\item
an simple approximation algorithm that,
given the endpoints of intervals
of a $k$-interval function $f$,
finds a \acrshort{dnf} representation of $f$
that is at worst
$2k$ times larger
than an optimal \acrshort{dnf} representation
(\cref{chap:2kapprox}) and

\item
an improved approximation algorithm
for $k$-interval functions
and show that it does not perform significantly
better than $2k$-approximation for large $k$
(\cref{chap:betterapprox}).
\end{itemize}
