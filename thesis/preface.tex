\chapter*{Introduction}
\addcontentsline{toc}{chapter}{Introduction}

\todocepek[inline]{Přidej:
Booleovské funkce obecně a jejich běžné reprezentace - formule, obvody, decision diagrams (bez definic); teprve potom reprezentace intervaly (mnoziny prir. cisel, zavedl Schieber); zobecneni na vice intervalu; prace Dubovskeho.}

Boolean functions defined by intervals were introduced
by \citet{Schieber2005154}.
A pair of $n$-bit numbers $a, b$
defines a $1$-interval function
$\intervalfn{f}{n}{a}{b}$:

$$
\apply{\intervalfn{f}{n}{a}{b}}{x} = 1
\iff a \leq x \leq b
$$

Note that such function is the indicator function
\todo{Is this term correct?}
of membership of $n$-bit numbers
in the interval $\interval{a}{b}$.

\citeauthor{Schieber2005154} showed an efficient method
to find a minimum \acrfull{dnf} representation
of any $1$-interval Boolean function
given by a pair of endpoints.\citep{Schieber2005154}

Since the problem of Boolean minimization is in general
$\Sigma_2^p$-complete \citep{Umans1998},
there is an interesting question of how difficult it is
to minimize \acrshort{dnf} representation of
a function defined by a set of intervals.
\citeauthor{Dubovsky2012} investigated the problem
in the case of $2$-interval functions.\citep{Dubovsky2012}

In this thesis we review the existing results regarding
minimization of \acrshort{dnf} representations of
single- and multi-interval functions,
and then present four new results:

\begin{itemize}
\item a simplified algorithm
for minimizing \acrshort{dnf} representations
of $2$-switch
$2$-interval functions (\cref{sec:2int2switch}),
\item
show that the technique used for proving the optimality
of the algorithms that minimize
\acrshort{dnf} representation
of $1$- and $2$-switch
(that is all $1$-interval
and $2$-switch $2$-interval)
functions
can not be used in general for
functions with $3$ or more switches
(\cref{sec:3switch}),
\item
introduce an algorithm
that,
given the intervals of a $k$-interval function,
finds the function's \acrshort{dnf} representation
that is at worst
$2k$ times larger
than an optimal \acrshort{dnf} representation
(\cref{chap:2kapprox}) and
\item
propose an improved approximation algorithm
for $k$-interval functions
and show that it does not perform significantly
better than $2k$-approximation for large $k$
(\cref{chap:betterapprox}).
\end{itemize}
