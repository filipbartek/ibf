\chapter{Single interval functions}

In this chapter,
I will introduce an efficient algorithm for optimal spanning
of single interval Boolean functions,
originally shown by Schieber et al.\cite{Schieber2005154}

In the chapters that follow,
I will use this algorithm as a procedure
in order to span multi-interval functions.

In the remainder of this chapter,
$a$ and $b$ ($a \leq b$) will denote the endpoints
of the spanned interval
and $n$ the number of input bits.
Thus,
we'll be spanning the function $\intervalfn{f}{n}{a}{b}$.

The algorithm is recursive.

\section{Trivial cases}

First we'll deal with the trivial interval functions.

\subsection{\texorpdfstring{$a = b$}{a = b}}

We span the interval $\interval{a}{a}$
with the single ternary vector $a$.

From now on,
let $a < b$.

\subsection{
\texorpdfstring{$a = \rep{0}{n}$}{a = 0...0}
and
\texorpdfstring{$b = \rep{1}{n}$}{b = 1...1}
}

We span the interval
$\interval{\rep{0}{n}}{\rep{1}{n}}$
with the single ternary vector $\rep{\phi}{n}$.

From now on,
let $a > \rep{0}{n}$ or $b < \rep{1}{n}$.

\section{Prefix and suffix case}
\label{sec:prefixsuffix}

Since we have dealt with the trivial cases,
we are now left with the situation
$\rep{0}{n} \leq a < b \leq \rep{1}{n}$
and
$a > \rep{0}{n}$ or $b < \rep{1}{n}$.

Let's consider the prefix case,
that is $\interval{\rep{0}{n}}{b}$ ($a = \rep{0}{n}$),
and the suffix case,
that is $\interval{a}{\rep{1}{n}}$ ($b = \rep{1}{n}$).

\begin{definition}[Complement]
The complement of a binary symbol $x$ ($x \in \booldom$),
denoted $\compl{x}$,
is $1 - x$.

We get the complement of a binary vector
by flipping all of its bits,
that is $\compl{x} = (\compl{x_1}, \ldots,
\compl{x_n})$.

The complement of the $\phi$ symbol is $\phi$
($\compl{\phi} = \phi$).

To obtain the complement of a ternary vector,
we flip all its fixed bits.
\end{definition}

Note that prefix and suffix cases are complementary --
we may transform a suffix instance
$\interval{a}{\rep{1}{n}}$
to a prefix instance $\interval{\rep{0}{n}}{\compl{a}}$.
Flipping the polarity of the resulting ternary vectors
yields a solution of the initial suffix instance
$\interval{a}{\rep{1}{n}}$.
This means it suffices to solve the prefix case,
which is what we'll do.

Let $a = \rep{0}{n}$ and $b < \rep{1}{n}$.
Let $c$ be the $n$-bit number $b + 1$.
Since $b < \rep{1}{n}$,
we don't need more than $n$ bits to encode $c$.

The algorithm produces one ternary vector
for each $1$-bit in $c$.
If $o$ is a position of a $1$-bit in $c$
($\bit{c}{o} = 1$),
then the corresponding ternary vector
is $\bits{c}{1}{o - 1} 0 \rep{\phi}{n - o}$.
Thus we get the following spanning set:

\begin{equation*}
\mathcal{T} =
\{\bits{c}{1}{o - 1} 0 \rep{\phi}{n - o} | \bit{c}{o} = 1\}
\end{equation*}

\begin{theorem}
$\mathcal{T}$ spans exactly the interval
$\interval{\rep{0}{n}}{b}$ (feasibility).
\end{theorem}

\begin{proof}
To see that every number spanned by $\mathcal{T}$
is in $\interval{\rep{0}{n}}{b}$,
note that all the numbers spanned by $\mathcal{T}$
are smaller than $c$,
since their \acrlong{msb} different from $c$
must be a $0$-bit
and they must differ from $c$.

On the other hand,
every number $x$ smaller than $c$ must differ from $c$,
and the leftmost different bit must be $0$ in $x$
and $1$ in $c$.
Let $o$ be its position.
Then
$\bits{c}{1}{o - 1} 0 \rep{\phi}{n - o} \in \mathcal{T}$
spans $x$.
\end{proof}

\begin{theorem}
$\mathcal{T}$ is minimal in size (optimality).
\end{theorem}

\begin{proof}
We'll construct a set $V$ of $|\mathcal{T}|$ true points
no pair of which can be spanned by a single ternary vector.
Dubovský\cite{Dubovsky2012} calls such sets ,,orthogonal''.
Doing so, we'll show a lower bound $|\mathcal{T}|$
on the size of feasible solutions,
proving optimality of $\mathcal{T}$.

Similarly to the spanning vectors,
the orthogonal true points correspond to $1$-bits of $c$:

\begin{equation}
V =
\{\bits{c}{1}{o - 1} 0 \bits{c}{o + 1}{n} |
\bit{c}{o} = 1\}
\end{equation}

Clearly $|V| = |\mathcal{T}|$.

It's easy to see that any ternary vector that spans two
% TODO: Not so easy to see for a fresh reader.
% Go into more detail.
different points in $V$ must also span the false point $c$,
so it can't be a part of the solution.
Also note that all points in $V$ are smaller than $c$,
so they are true points.

If there was a feasible spanning set
of size smaller than $|V|$,
at least one of its vectors would need to span at least
two points in $V$.
As we have shown, such ternary vector would necessarily
also span the false point $c$,
leading to contradiction with the set's feasibility.
\end{proof}

\section{General case}

Having solved the trivial and prefix and suffix cases,
we are left with the situation
$\rep{0}{n} < a < b < \rep{1}{n}$.

If $a$ and $b$ have the same \acrshort{msb},
we recursively span
$\interval{\bits{a}{2}{n}}{\bits{b}{2}{n}}$
and prepend the \acrshort{msb}
to the solution.

Let $\bit{a}{1} \neq \bit{b}{1}$.
Since $a \leq b$,
necessarily $\bit{a}{1} = 0$
and $\bit{b}{1} = 1$.
This restriction leaves us with
four possible combinations of pairs
of \acrshort{msb}s:

\begin{enumerate}
\item $\bits{a}{1}{2} = 01$, $\bits{b}{1}{2} = 10$
\item $\bits{a}{1}{2} = 00$, $\bits{b}{1}{2} = 10$
\item $\bits{a}{1}{2} = 01$, $\bits{b}{1}{2} = 11$
\item $\bits{a}{1}{2} = 00$, $\bits{b}{1}{2} = 11$
\end{enumerate}

Following Schieber et al.,\cite{Schieber2005154}
we'll deal with each of these cases separately.
We won't prove feasibility nor optimality in this text.
Please refer to the original article for the proofs.

\subsection{\texorpdfstring
{$\bits{a}{1}{2} = 01$ and $\bits{b}{1}{2} = 10$}
{a[1,2] = 01 and b[1,2] = 10}
}

In this case,
we'll span the two sub-intervals
$\interval{a}{0 \rep{1}{n-1}}$
and
$\interval{1 \rep{0}{n-1}}{b}$
separately.
Note that after leaving out the initial shared bit,
they are prefix and suffix interval, respectively,
so the algorithm from section \ref{sec:prefixsuffix}
is used to span each of the sub-intervals.

\subsection{\texorpdfstring
{$\bits{a}{1}{2} = 00$ and $\bits{b}{1}{2} = 10$}
{a[1,2] = 00 and b[1,2] = 10}
}
\label{sec:0010}

In this case,
we divide the interval in three sub-intervals:

\begin{itemize}
\item $\interval{a}{00 \rep{1}{n-2}}$
\item $\interval{01 \rep{0}{n-2}}{01 \rep{1}{n-2}}$
\item $\interval{10 \rep{0}{n-2}}{b}$
\end{itemize}

The subinterval
$\interval{01 \rep{0}{n-2}}{01 \rep{1}{n-2}}$
is spanned by the single ternary vector
$01 \rep{\phi}{n-2}$.

The subintervals
$\interval{a}{00 \rep{1}{n-2}}$
and
$\interval{10 \rep{0}{n-2}}{b}$
are spanned together as follows:

\begin{itemize}
\item Recursively solve the $(n-1)$-bit instance
$\interval{0 \bits{a}{3}{n}}{1 \bits{b}{3}{n}}
= \interval{\bit{a}{1} \bits{a}{3}{n}}{\bit{b}{1} \bits{b}{3}{n}}$
\item Insert a $0$-bit in the second position
of the vectors from the resulting spanning set
\end{itemize}

\subsection{\texorpdfstring
{$\bits{a}{1}{2} = 01$ and $\bits{b}{1}{2} = 11$}
{a[1,2] = 01 and b[1,2] = 11}
}

This case is complementary to case \ref{sec:0010}.
As in suffix case,
note that flipping the bits in the endpoints
transforms this case to case \ref{sec:0010}
and flipping the fixed bits in the resulting spanning set
preserves correctness.

\subsection{\texorpdfstring
{$\bits{a}{1}{2} = 00$ and $\bits{b}{1}{2} = 11$}
{a[1,2] = 00 and b[1,2] = 11}
}

% TODO: Solve the general case
% (without proving optimality
% and perhaps also feasibility).