\chapter{Single interval functions}

In this chapter,
I will introduce an efficient algorithm for optimal spanning
of single interval Boolean functions,
originally shown by Schieber et al.\cite{Schieber2005154}

In the chapters that follow,
I will use this algorithm as a procedure
in order to span multi-interval functions.

In the remainder of this chapter,
$a$ and $b$ ($a \leq b$) will denote the endpoints
of the spanned interval
and $n$ the number of input bits.
Thus,
we'll be spanning the function $\intervalfn{f}{n}{a}{b}$.

\section{Prefix and suffix case}

Let's first consider the prefix case,
that is $\interval{\rep{0}{n}}{b}$ ($a = \rep{0}{n}$),
and the suffix case,
that is $\interval{a}{\rep{1}{n}}$ ($b = \rep{1}{n}$).

If both $a = \rep{0}{n}$ and $b = \rep{1}{n}$,
the optimal spanning set trivially consists
of the single ternary vector $\rep{\phi}{n}$,
corresponding to the trivial formula $1$.

From now on,
let $a > \rep{0}{n}$ or $b < \rep{1}{n}$.

Note that prefix and suffix cases are complementary --
we may transform a suffix instance
$\interval{a}{\rep{1}{n}}$
to a prefix instance $\interval{\rep{0}{n}}{\compl{a}}$.
Flipping the polarity of the resulting ternary vectors
yields a solution for the initial suffix instance
$\interval{a}{\rep{1}{n}}$.
This means it's enough to solve the prefix case,
which is what we'll do.

Let $a = \rep{0}{n}$ and $b < \rep{1}{n}$.
Let $c$ be the $n$-bit number $b + 1$.
Since $b < \rep{1}{n}$,
we don't need more than $n$ bits to encode $c$.

The algorithm produces one ternary vector
for each $1$-bit in $c$.
If $o$ is a position of a $1$-bit in $c$
($\bit{c}{o} = 1$),
then the corresponding ternary vector
is $\bits{c}{1}{o - 1} 0 \rep{\phi}{n - o}$.
Thus we get the following spanning set:

\begin{equation*}
\mathcal{T} =
\{\bits{c}{1}{o - 1} 0 \rep{\phi}{n - o} | \bit{c}{o} = 1\}
\end{equation*}

\begin{theorem}
$\mathcal{T}$ spans exactly the interval
$\interval{\rep{0}{n}}{b}$ (feasibility).
\end{theorem}

\begin{proof}
To see that every number spanned by $\mathcal{T}$
is in $\interval{\rep{0}{n}}{b}$,
note that all the numbers spanned by $\mathcal{T}$
are smaller than $c$,
since their \acrlong{msb} different from $c$
must be a $0$-bit
and they must differ from $c$.

On the other hand,
every number $x$ smaller than $c$ must differ from $c$,
and the leftmost different bit must be $0$.
Let $o$ be its position.
Then
$\bits{c}{1}{o - 1} 0 \rep{\phi}{n - o} \in \mathcal{T}$
spans $x$.
\end{proof}

\begin{theorem}
$\mathcal{T}$ is minimal in size (optimality).
\end{theorem}

\begin{proof}
We'll construct a set of $|\mathcal{T}|$ true points
no pair of which can be spanned by a single ternary vector.
Doing so, we'll show a lower bound $|\mathcal{T}|$
on the size of feasible solutions,
proving optimality of $\mathcal{T}$.

Once more,
the orthogonal true points correspond to $1$-bits of $c$:

\begin{equation}
V =
\{\bits{c}{1}{o - 1} 0 \bits{c}{o + 1}{n} |
\bit{c}{o} = 1\}
\end{equation}

Clearly $|V| = |\mathcal{T}|$.

It's easy to see that any ternary vector that spans two
% TODO: Not so easy to see for a fresh reader.
% Go into more detail.
different points in $V$ must also span the false point $c$,
so it can't be a part of the solution.
Also note that all points in $V$ are smaller than $c$,
so they are true points.

If there was a feasible spanning set
of size smaller than $|V|$,
at least one of its vectors would need to span at least
two points in $V$.
As we have shown, such ternary vector would necessarily
also span the false point $c$,
leading to contradiction with its feasibility.
\end{proof}

% TODO: Solve the general case
% (without proving optimality
% and perhaps also feasibility).