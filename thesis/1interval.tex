\chapter{\texorpdfstring{$1$}{1}-interval functions}
\label{chap:1interval}

In this chapter,
we will show an efficient way to compute
a minimum spanning set of any $1$-interval Boolean function
defined by the interval endpoints.
The algorithm was originally shown by
\citet{Schieber2005154}.

In the chapters that follow,
we will use this algorithm as a procedure
in order to span multi-interval functions.

In the remainder of this chapter,
$a$ and $b$ ($a \leq b$) will denote the endpoints
of the interval in question
and $n$ the number of input bits.
We will be looking for a minimum representation
of the function $\fnab$.

\begin{description}
\item[Input] $n$-bit numbers $a, b$ such that $a \leq b$ for $n \geq 0$.
\item[Output] A spanning set of $\fnab$.
\end{description}

The algorithm differentiates various cases
of input values.
Solving the more difficult cases involves recursive calls,
while the simpler cases are solved iteratively.

\section{Trivial cases}

We will first deal with the trivial intervals.

\subsection{\texorpdfstring{$n=0$}{n=0}}

Since the algorithm is recursive,
we need to deal with the degenerate case
of $0$-bit endpoints.
We span such interval using a single ternary vector
of length $0$.
Note that such vector corresponds to the empty term,
that is tautology.

From now on, let $n \geq 1$.

\subsection{\texorpdfstring{$\bit{a}{1} = \bit{b}{1}$}{\textbit{a}{1} = \textbit{b}{1}}}
\label{sec:commonprefix}

If $a$ and $b$ share the leading bit,
let $c$ be the longest common prefix of $a$ and $b$
and let $j$ be its length.
Note that $j \geq 1$.
Let $a' = \bits{a}{j+1}{n}$ and $b' = \bits{b}{j+1}{n}$
($a = ca'$ and $b = cb'$).
Let $\mathcal{T}'$ be the spanning set of
the function $\intervalfn{f}{n-j}{a'}{b'}$.
Since $n-j < n$,
we can get such set by recursion.
We get the spanning set of $\fnab$
by prepending $c$
to each of the vectors in $\mathcal{T}'$
($\mathcal{T} = c \mathcal{T}'
= \curly{cT' | T' \in \mathcal{T}'}$).
It is easy to see that if $\mathcal{T}'$
spans exactly $\interval{a'}{b'}$,
then $c \mathcal{T}'$ spans exactly
$\interval{c a'}{c b'} = \interval{a}{b}$.

Note that in the special case $a=b$
(that is $c=a=b$ and $j=n$),
$\mathcal{T}'$ consists of the single vector of length $0$
and $\mathcal{T}$ correctly consists of the single vector $c=a=b$.

From now on let $\bit{a}{1} \neq \bit{b}{1}$.
Since $a \leq b$,
necessarily $\bit{a}{1} = 0$ and $\bit{b}{1} = 1$.
From now on we may especially assume that $a < b$.

\subsection{\texorpdfstring
{$a = \rep{0}{n}$}{a = \textrep{0}{n}}
and
\texorpdfstring{$b = \rep{1}{n}$}{b = \textrep{1}{n}}
}

We span the full interval
$\interval{\rep{0}{n}}{\rep{1}{n}}$
with the single ternary vector $\rep{\phi}{n}$,
which corresponds to an empty term (tautology).

From now on,
let $a > \rep{0}{n}$ or $b < \rep{1}{n}$.

\section{Prefix and suffix case}
\label{sec:prefixsuffix}

\todo[inline]{Mention that the algorithm is linear time.}

Since we have dealt with the trivial cases,
we are now left with the situation
$\rep{0}{n} \leq a < b \leq \rep{1}{n}$
and
$a > \rep{0}{n}$ or $b < \rep{1}{n}$.

In this section we shall consider the prefix case,
that is $\interval{\rep{0}{n}}{b}$ ($a = \rep{0}{n}$),
and the suffix case,
that is $\interval{a}{\rep{1}{n}}$ ($b = \rep{1}{n}$).

\subsection{Suffix case}

\todo[inline]{Add an example with a picture.}

The prefix and suffix cases are complementary.
There is a size-preserving
1-1
\todomaybe{Typeset \quot{1-1} properly (correct dash etc.).}
correspondence between
the spanning sets of suffix intervals
and the spanning sets
of prefix intervals.
The correspondence complements each of the ternary vectors in the spanning set
-- the spanning set $\mathcal{T}$
of a suffix interval $\interval{a}{\rep{1}{n}}$
corresponds to the spanning set
$\compl{\mathcal{T}}
= \curly{\compl{T} | T \in \mathcal{T}}$
of the prefix interval
$\interval{\rep{0}{n}}{\compl{a}}$.
$$
\mathcal{T} \text{ spans } \interval{a}{\rep{1}{n}}
\iff
\curly{\compl{T} | T \in \mathcal{T}}
\text{ spans } \interval{\rep{0}{n}}{\compl{a}}
$$
\todomaybe{Prove correctness.}

Since the correspondence preserves
size of the spanning sets,
it also preserves optimality.
Thus we can use the correspondence to optimally span
suffix intervals
using a procedure that spans prefix intervals.
We will show such procedure in the following section.

\subsection{Prefix case}
\label{sec:prefix}

Let's proceed to span a prefix interval.
Let $a = \rep{0}{n}$ and $b < \rep{1}{n}$.

Let $c$ be the $n$-bit number $b + 1$.
Since $b < \rep{1}{n}$,
we do not need more than $n$ bits to encode $c$.

The algorithm produces one ternary vector
for each bit that is set to $1$ in $c$.
If $o$ is a position of a $1$ in $c$
($\bit{c}{o} = 1$),
then the corresponding ternary vector
is $\bits{c}{1}{o - 1} 0 \rep{\phi}{n - o}$.
Thus we get the following spanning set:

\begin{equation*}
\mathcal{T} =
\{\bits{c}{1}{o - 1} 0 \rep{\phi}{n - o} | \bit{c}{o} = 1\}
\end{equation*}

\begin{theorem}[Feasibility]
\label{theorem:prefixfeasible}
$\mathcal{T}$ spans exactly the interval
$\interval{\rep{0}{n}}{b}$.
\end{theorem}

\begin{proof}
To see that every number spanned by $\mathcal{T}$
is in $\interval{\rep{0}{n}}{b}$,
note that given an index $o$ of a bit which is set to $1$
in $c$,
the biggest number spanned by
$\bits{c}{1}{o-1} 0 \rep{\phi}{n-o}$ is
$\bits{c}{1}{o-1} 0 \rep{1}{n-o}$
which is still strictly smaller than $c$.

On the other hand,
consider a number $x$ smaller than $c$
and let $o$ be the most significant bit in which $x$ and $c$ differ. It follows that
$\bits{x}{1}{o-1} = \bits{c}{1}{o-1}$
and $0 = \bit{x}{o} < \bit{c}{o} = 1$.
Then
$\bits{c}{1}{o - 1} 0 \rep{\phi}{n - o} \in \mathcal{T}$
spans $x$.
\end{proof}

\begin{theorem}[Optimality]
\label{proof:prefixoptimal}
\todo{Relabel to "theorem:prefixoptimal".}
$\mathcal{T}$ is the minimum spanning set of
$\interval{\rep{0}{n}}{b}$.
\end{theorem}

\begin{proof}
We will construct an orthogonal set $V$
of size $\size{\mathcal{T}}$.
With the aid of \autoref{theorem:orthodnf},
this proves that $\mathcal{T}$ is a minimum spanning set.

Similarly to the spanning vectors,
the orthogonal true points correspond to $1$-bits of $c$:

\[
V =
\curly{\bits{c}{1}{o - 1} 0 \bits{c}{o + 1}{n}
| \bit{c}{o} = 1}
\]

Clearly $|V| = |\mathcal{T}|$.
Also note that all points in $V$ are smaller than $c$,
so they are true points.

We need to prove that $V$ is orthogonal.
Let $x, y \in V$, $x \neq y$.
Let $o_x$ and $o_y$ be the positions of the symbol $1$
in $c$
that were used to construct $x$ and $y$
($x = \bits{c}{1}{o_x - 1} 0 \bits{c}{o_x + 1}{n}$,
$y = \bits{c}{1}{o_y - 1} 0 \bits{c}{o_y + 1}{n}$).
Since $x \neq y$, necessarily $o_x \neq o_y$.
%Without loss of generality
%let $o_x < o_y$.

Let $T$ be any ternary vector that spans both $x$ and $y$.
To see that $T$ must also span the false point $c$,
note that every component $i$ of $c$
matches the respective component in at least one of $x,y$
(for every $i$,
$\bit{c}{i} = \bit{x}{i}$ or $\bit{c}{i} = \bit{y}{i}$).

\Cref{tab:xyc}
% \Cref used instead of \autoref for correct capitalization. Make sure the element names match in the packages cleveref and hyperref.
shows a more detailed comparison
of the corresponding subvectors of $x$, $y$ and $c$
in case $o_x < o_y$.

\begin{table}[h]
\centering
\begin{tabular}{c|ccccc}
& $\interval{1}{o_x - 1}$ & $[o_x]$
& $\interval{o_x + 1}{o_y - 1}$ & $[o_y]$
& $\interval{o_y + 1}{n}$ \\
\hline
$x$
& \underline{$\bits{c}{1}{o_x - 1}$}
& $0$
& \underline{$\bits{c}{o_x + 1}{o_y - 1}$}
& \underline{$\bit{c}{o_y}$}
& \underline{$\bits{c}{o_y + 1}{n}$} \\
$y$
& \underline{$\bits{c}{1}{o_x - 1}$}
& \underline{$\bit{c}{o_x}$}
& \underline{$\bits{c}{o_x + 1}{o_y - 1}$}
& $0$
& \underline{$\bits{c}{o_y + 1}{n}$} \\
\hline
$c$
& $\bits{c}{1}{o_x - 1}$
& $\bit{c}{o_x}$
& $\bits{c}{o_x + 1}{o_y - 1}$
& $\bit{c}{o_y}$
& $\bits{c}{o_y + 1}{n}$
\end{tabular}
\caption[Subvectors of $x$, $y$ and $c$
in case $o_x < o_y$]
{Subvectors of $x$, $y$ and $c$
in case $o_x < o_y$.
The underlined subvectors of $x$ and $y$
match their counterparts in $c$.}
\label{tab:xyc}
\end{table}
\todomaybe{Use a different emphasis for the matching subvectors in the table. Underline may get mistaken for complement.}

%This example demonstrates a basic technique
%of proving orthogonality of vectors.

We conclude that $V$ is orthogonal
and since $\size{\mathcal{T}} = \size{V}$,
$\mathcal{T}$ is a minimum spanning set
by \autoref{theorem:orthodnf}.
\end{proof}

\begin{corollary}
\label{corollary:prefixdependence}
For every prefix function $\prefixfn{f}{n}{b}$,
there is an orthogonal set $V$
of size $dnf(\prefixfn{f}{n}{b})$
such that
every pair of vectors $x,y \in V$, $x \neq y$,
collides on the false point $b+1$.
\end{corollary}

It is easy to see that the spanning set $\mathcal{T}$
produced by the algorithm is disjoint.
Since it is minimum in general,
it is a minimum disjoint spanning set too.

\section{General case}

Having solved the trivial and prefix and suffix cases,
we are left with the situation
$\rep{0}{n} < a < b < \rep{1}{n}$.

Since we have handled the case $\bit{a}{1} = \bit{b}{1}$
in \cref{sec:commonprefix},
we may assume that $\bit{a}{1} = 0$ and $\bit{b}{1} = 1$.
This restriction leaves us with
four possible combinations of pairs
of leading bits of $a$ and $b$:

\begin{enumerate}
\item $\bits{a}{1}{2} = 01$, $\bits{b}{1}{2} = 10$
\item $\bits{a}{1}{2} = 00$, $\bits{b}{1}{2} = 10$
\item $\bits{a}{1}{2} = 01$, $\bits{b}{1}{2} = 11$
\item $\bits{a}{1}{2} = 00$, $\bits{b}{1}{2} = 11$
\end{enumerate}

Following \citet{Schieber2005154},
we will deal with each of these cases separately.
\citeauthor{Schieber2005154}'s proofs of feasibility
and optimality of the following
algorithm are rather technical and out of scope
of this text.
Note, however, that the proof of optimality is implicitly
based on building an orthogonal set of the same size
as the computed spanning set.

In the sections that follow,
we will denote $\bits{a}{3}{n}$ with $\hat{a}$
($a = \bits{a}{1}{2} \hat{a}$)
and $\bits{b}{3}{n}$ with $\hat{b}$
($b = \bits{b}{1}{2} \hat{b}$).

\subsection{\texorpdfstring
{$\bits{a}{1}{2} = 01$ and $\bits{b}{1}{2} = 10$}
{\textbits{a}{1}{2} = 01 and \textbits{b}{1}{2} = 10}
}
\label{sec:1interval0110}

In this case
we span the two sub-intervals
$\interval{01 \hat{a}}{01 \rep{1}{n-2}}$
and
$\interval{10 \rep{0}{n-2}}{10 \hat{b}}$
separately.

The endpoints of the sub-interval
$\interval{01 \hat{a}}{01 \rep{1}{n-2}}$
share the leading bit $0$.
This means that this sub-interval is immediately
reduced to the suffix instance
$\interval{1 \hat{a}}{1 \rep{1}{n-2}}$,
which can be spanned using the suffix algorithm
show in \cref{sec:prefixsuffix}.

The remaining sub-interval
$\interval{10 \rep{0}{n-2}}{10 \hat{b}}$
is reduced to a prefix interval
$\interval{0 \rep{0}{n-2}}{0 \hat{b}}$
in a similar fashion.

\subsection{\texorpdfstring
{$\bits{a}{1}{2} = 00$ and $\bits{b}{1}{2} = 10$}
{\textbits{a}{1}{2} = 00 and \textbits{b}{1}{2} = 10}
}
\label{sec:0010}

In this case,
we divide the interval into three sub-intervals:

\begin{itemize}
\item $\interval{00 \hat{a}}{00 \rep{1}{n-2}}$
\item $\interval{01 \rep{0}{n-2}}{01 \rep{1}{n-2}}$
\item $\interval{10 \rep{0}{n-2}}{10 \hat{b}}$
\end{itemize}

The sub-interval
$\interval{01 \rep{0}{n-2}}{01 \rep{1}{n-2}}$
is spanned by the single ternary vector
$01 \rep{\phi}{n-2}$.

The sub-intervals
$\interval{00 \hat{a}}{00 \rep{1}{n-2}}$
and
$\interval{10 \rep{0}{n-2}}{10 \hat{b}}$
are spanned together as follows:

\begin{enumerate}
\item Recursively solve the $(n-1)$-bit instance
$\interval{0 \hat{a}}{1 \hat{b}}
= \interval{\bit{a}{1} \bits{a}{3}{n}}{\bit{b}{1} \bits{b}{3}{n}}$
\item Insert a $0$-bit in the second position
of the vectors from the resulting spanning set
\end{enumerate}

Note that $\bit{a}{2} = \bit{b}{2} = 0$,
so the $n$-bit vectors produced this way
exactly span the union of the intervals
$\interval{00 \hat{a}}{00 \rep{1}{n-2}}$
and $\interval{10 \rep{0}{n-2}}{10 \hat{b}}$.

\subsection{\texorpdfstring
{$\bits{a}{1}{2} = 01$ and $\bits{b}{1}{2} = 11$}
{\textbits{a}{1}{2} = 01 and \textbits{b}{1}{2} = 11}
}

This case is complementary to case \ref{sec:0010}.
As in suffix case,
note that flipping all the bits in the endpoints
transforms this case to case \ref{sec:0010}
and flipping the fixed bits in the resulting spanning set
preserves correctness.
\todo{Does it?}

\subsection{\texorpdfstring
{$\bits{a}{1}{2} = 00$ and $\bits{b}{1}{2} = 11$}
{\textbits{a}{1}{2} = 00 and \textbits{b}{1}{2} = 11}
}
\label{sec:1interval0011}

Let $j$ be maximal such that
$\bits{a}{1}{j} = \rep{0}{j}$ and
$\bits{b}{1}{j} = \rep{1}{j}$.
Since $a > \rep{0}{n}$ (or $b < \rep{1}{n}$),
necessarily $j < n$.

The number $j$ gives us three sub-intervals to span:

\begin{itemize}
\item $\interval{a}{\rep{0}{j} \rep{1}{n-j}}$
\item $\interval{\rep{0}{j-1} 1 \rep{0}{n-j}}
{\rep{1}{j-1} 0 \rep{1}{n-j}}$
\item $\interval{\rep{1}{j} \rep{0}{n-j}}{b}$
\end{itemize}

Note that the second sub-interval contains exactly
the numbers that don't start with either $j$ zeros
or $j$ ones,
and as such can be spanned by $j$ ternary vectors.
We construct $T_1, \ldots, T_j$ that span
the second sub-interval
by appending $\rep{\phi}{n-j}$
to each of the $j$ cyclic shifts of $01 \rep{\phi}{j-2}$.
Thus,
$T_1 = 01 \rep{\phi}{j-2} \rep{\phi}{n-j}$,
$\ldots$,
$T_i = \rep{\phi}{i-1} 01 \rep{\phi}{j-1-i} \rep{\phi}{n-j}$
(for $i \in \curly{1, \ldots, j-1}$),
$\ldots$,
$T_{j-1} = \rep{\phi}{j-2} 01 \rep{\phi}{n-j}$,
$T_j = 1 \rep{\phi}{j-2} 0 \rep{\phi}{n-j}$.

The other two sub-intervals are spanned recursively.
Let $a'' = \bits{a}{j}{n}$ and $b'' = \bits{b}{j}{n}$.
Note that $\bit{a''}{1} = 0$ and $\bit{b''}{1} = 1$.
Note that $|a''| = |b''| = n - j + 1$.
Since $j \geq 2$, $n-j+1 < n$.
Let $\mathcal{T}''$ be the spanning set
of $\interval{a''}{b''}$ computed recursively.

The spanning set of $\interval{a}{b}$ will be computed
from the vectors $T_1, \ldots, T_j$ and $\mathcal{T}''$
based on the relation between $(n-j)$-bit suffixes of $a$
and $b$.
Let $a' = \bits{a}{j+1}{n}$ and $b' = \bits{b}{j+1}{n}$.

\subsubsection{\texorpdfstring
{$b' < a' - 1$}
{b' < a' - 1}
}

In this case,
the spanning set of $\interval{a}{b}$ consists
of the vectors $T_1, \ldots, T_j$ and vectors obtained
by prepending each vector from $\mathcal{T}''$ with
$\rep{\phi}{j-1}$.

\subsubsection{\texorpdfstring
{$b' \geq a' - 1$}
{b' >= a' - 1}
}
% Case 4.2

In this case,
the spanning set of $\interval{a}{b}$ consists
of the vectors $T_1, \ldots, T_{j-1}$ (omitting $T_j$)
and a set $\mathcal{T}'$ which is derived
from $\mathcal{T}''$
in the following way:

\begin{align*}
\mathcal{T}' &= \curly{
\rep{\phi}{j-1} T | T \in \mathcal{T}'' \text{ and }
\bit{T}{1} \in \curly{0, \phi}
} \\
&\cup \curly{
1 \rep{\phi}{j-1} \bits{T}{2}{n-j+1} | T \in \mathcal{T}''
\text{ and } \bit{T}{1} = 1
}
\end{align*}

% TODO: Go into more detail.
