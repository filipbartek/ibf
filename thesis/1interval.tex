\chapter{\texorpdfstring{$1$}{1}-interval functions}
\label{chap:1interval}

In this chapter,
we will show and efficient way to compute
a minimum spanning set of any $1$-interval Boolean function
defined by the interval endpoints.
The algorithm was originally shown by Schieber et al.
\cite{Schieber2005154}

In the chapters that follow,
we will use this algorithm as a procedure
in order to span multi-interval functions.

In the remainder of this chapter,
$a$ and $b$ ($a \leq b$) will denote the endpoints
of the interval in question
and $n$ the number of input bits;
we'll be looking for a minimum representation of
function $\intervalfn{f}{n}{a}{b}$.

\begin{description}
\item[Input] $n$-bit numbers $a, b$ such that $a \leq b$
\item[Output] A set of ternary vectors
\end{description}

The algorithm is recursive in input length.
\todokucera{Buď rozvést nebo odstranit.}

\section{Trivial cases}

We will first deal with the trivial intervals.

\subsection{\texorpdfstring{$a = b$}{a = b}}

We span the interval $\interval{a}{a}$
with the single ternary vector $a$.

From now on, let $a < b$.

\subsection{\texorpdfstring{$a = \rep{0}{n}$}{a = 0...0}
and
\texorpdfstring{$b = \rep{1}{n}$}{b = 1...1}
}

We span the interval
$\interval{\rep{0}{n}}{\rep{1}{n}}$
with the single ternary vector $\rep{\phi}{n}$,
which corresponds to an empty term.

From now on,
let $a > \rep{0}{n}$ or $b < \rep{1}{n}$.

\section{Prefix and suffix case}
\label{sec:prefixsuffix}

Since we have dealt with the trivial cases,
we are now left with the situation
$\rep{0}{n} \leq a < b \leq \rep{1}{n}$
and
$a > \rep{0}{n}$ or $b < \rep{1}{n}$.

In this section we shall consider the prefix case,
that is $\interval{\rep{0}{n}}{b}$ ($a = \rep{0}{n}$),
and the suffix case,
that is $\interval{a}{\rep{1}{n}}$ ($b = \rep{1}{n}$).

Note that prefix and suffix cases are complementary.
We may transform a suffix instance
$\interval{a}{\rep{1}{n}}$
to a prefix instance $\interval{\rep{0}{n}}{\compl{a}}$
(where $\compl{a}$ is the complementary vector of $a$)
and flip the polarity of the resulting ternary vectors
in all their components.
Since the instances $\interval{a}{\rep{1}{n}}$
and $\interval{\rep{0}{n}}{\compl{a}}$
are symmetric in general,
using an optimal algorithm to span the prefix instance
$\interval{\rep{0}{n}}{\compl{a}}$
ensures we get an optimal spanning set
of the original suffix instance as well.

Let's proceed to span a prefix instance.
Let $a = \rep{0}{n}$ and $b < \rep{1}{n}$.
Let $c$ be the $n$-bit number $b + 1$.
Since $b < \rep{1}{n}$,
we do not need more than $n$ bits to encode $c$.

The algorithm produces one ternary vector
for each bit that is set to $1$ in $c$.
If $o$ is a position of a $1$ in $c$
($\bit{c}{o} = 1$),
then the corresponding ternary vector
is $\bits{c}{1}{o - 1} 0 \rep{\phi}{n - o}$.
Thus we get the following spanning set:

\begin{equation*}
\mathcal{T} =
\{\bits{c}{1}{o - 1} 0 \rep{\phi}{n - o} | \bit{c}{o} = 1\}
\end{equation*}

\begin{theorem}[Feasibility]
\label{theorem:prefixfeasible}
$\mathcal{T}$ spans exactly the interval
$\interval{\rep{0}{n}}{b}$.
\end{theorem}

\begin{proof}
To see that every number spanned by $\mathcal{T}$
is in $\interval{\rep{0}{n}}{b}$,
note that given an index $o$ of a bit which is set to $1$
in $c$,
the biggest number spanned by
$\bits{c}{1}{o-1} 0 \rep{\phi}{n-o}$ is
$\bits{c}{1}{o-1} 0 \rep{1}{n-o}$
which is still strictly smaller than $c$.

On the other hand,
consider a number $x$ smaller than $c$
and let $o$ be the most significant bit in which $x$ and $c$ differ. It follows that
$\bits{x}{1}{o-1} = \bits{c}{1}{o-1}$
and $0 = \bit{x}{o} < \bit{c}{o} = 1$.
Then
$\bits{c}{1}{o - 1} 0 \rep{\phi}{n - o} \in \mathcal{T}$
spans $x$.
\end{proof}

\begin{theorem}[Optimality]
$\mathcal{T}$ is the minimum spanning set of
$\intervalfn{f}{n}{\rep{0}{n}}{b}$.
\end{theorem}

\begin{proof}
We'll construct a set $V$ of $|\mathcal{T}|$ true points
no pair of which can be spanned by a single ternary vector.
\dubovsky{}\cite{Dubovsky2012} calls such sets
\quot{orthogonal}.
\todo{Define orthogonality in chapter Definitions.}
\todo{Relate to essential sets.}
Doing so, we'll show a lower bound $|\mathcal{T}|$
\todo{Refer to a statement that demonstrates properties of orthogonal sets.}
on the size of feasible solutions,
proving optimality of $\mathcal{T}$.

Similarly to the spanning vectors,
the orthogonal true points correspond to $1$-bits of $c$:

\begin{equation}
V =
\{\bits{c}{1}{o - 1} 0 \bits{c}{o + 1}{n} |
\bit{c}{o} = 1\}
\end{equation}

Clearly $|V| = |\mathcal{T}|$.
Also note that all points in $V$ are smaller than $c$,
so they are true points.

It's easy to see that any ternary vector that spans two
different points in $V$ must also span the false point $c$,
so it can't be a part of the solution.
\todo{Explain in detail.}

If there was a feasible spanning set
of size smaller than $|V|$,
at least one of its vectors would need to span at least
two points in $V$.
As we have shown, such ternary vector would necessarily
also span the false point $c$,
leading to contradiction with the set's feasibility
(Theorem \ref{theorem:prefixfeasible}).
\end{proof}

\section{General case}

Having solved the trivial and prefix and suffix cases,
we are left with the situation
$\rep{0}{n} < a < b < \rep{1}{n}$.

If $a$ and $b$ have the same \acrshort{msb},
we recursively span
$\interval{\bits{a}{2}{n}}{\bits{b}{2}{n}}$
and prepend the \acrshort{msb}
to the solution.

Let us now consider the case when
$\bit{a}{1} \neq \bit{b}{1}$.
Since $a \leq b$,
necessarily $\bit{a}{1} = 0$
and $\bit{b}{1} = 1$.
This restriction leaves us with
four possible combinations of pairs
of \acrshort{msb}s:
\todo{Avoid using MSBs.}

\begin{enumerate}
\item $\bits{a}{1}{2} = 01$, $\bits{b}{1}{2} = 10$
\item $\bits{a}{1}{2} = 00$, $\bits{b}{1}{2} = 10$
\item $\bits{a}{1}{2} = 01$, $\bits{b}{1}{2} = 11$
\item $\bits{a}{1}{2} = 00$, $\bits{b}{1}{2} = 11$
\end{enumerate}

Following Schieber et al.,\cite{Schieber2005154}
\todokucera{pokud chcete mít v citacích jméno, tak použijte odpovídající BiBTeXový styl (autoři rok)}
we'll deal with each of these cases separately.
The proofs of feasibility and optimality of the following
algorithm are rather technical and they are out of scope
of this text.
Please refer to the original article for the proofs.
Note that the proof of optimality is implicitly based
on showing orthogonal sets.

\todo[inline]{Consider spelling the leading bits of $a$ and $b$ explicitly, for example $00 a'$ in place of $a$.}

\subsection{\texorpdfstring
{$\bits{a}{1}{2} = 01$ and $\bits{b}{1}{2} = 10$}
{a[1,2] = 01 and b[1,2] = 10}
}

In this case,
we'll span the two sub-intervals
$\interval{a}{0 \rep{1}{n-1}}$
and
$\interval{1 \rep{0}{n-1}}{b}$
separately.
Note that after leaving out the initial shared bit,
they are prefix and suffix interval, respectively,
so the algorithm from section \ref{sec:prefixsuffix}
can be used to span each of the sub-intervals.

\subsection{\texorpdfstring
{$\bits{a}{1}{2} = 00$ and $\bits{b}{1}{2} = 10$}
{a[1,2] = 00 and b[1,2] = 10}
}
\label{sec:0010}

In this case,
we divide the interval into three sub-intervals:

\begin{itemize}
\item $\interval{a}{00 \rep{1}{n-2}}$
\item $\interval{01 \rep{0}{n-2}}{01 \rep{1}{n-2}}$
\item $\interval{10 \rep{0}{n-2}}{b}$
\end{itemize}

The subinterval
$\interval{01 \rep{0}{n-2}}{01 \rep{1}{n-2}}$
is spanned by the single ternary vector
$01 \rep{\phi}{n-2}$.

The subintervals
$\interval{a}{00 \rep{1}{n-2}}$
and
$\interval{10 \rep{0}{n-2}}{b}$
are spanned together as follows:

\begin{itemize}
\item Recursively solve the $(n-1)$-bit instance
$\interval{0 \bits{a}{3}{n}}{1 \bits{b}{3}{n}}
= \interval{\bit{a}{1} \bits{a}{3}{n}}{\bit{b}{1} \bits{b}{3}{n}}$
\item Insert a $0$-bit in the second position
of the vectors from the resulting spanning set
\end{itemize}

\subsection{\texorpdfstring
{$\bits{a}{1}{2} = 01$ and $\bits{b}{1}{2} = 11$}
{a[1,2] = 01 and b[1,2] = 11}
}

This case is complementary to case \ref{sec:0010}.
As in suffix case,
note that flipping the bits in the endpoints
transforms this case to case \ref{sec:0010}
and flipping the fixed bits in the resulting spanning set
preserves correctness.

\subsection{\texorpdfstring
{$\bits{a}{1}{2} = 00$ and $\bits{b}{1}{2} = 11$}
{a[1,2] = 00 and b[1,2] = 11}
}

Let $j$ be maximal such that
$\bits{a}{1}{j} = \rep{0}{j}$ and
$\bits{b}{1}{j} = \rep{1}{j}$.
Since $a > \rep{0}{n}$ (and $b < \rep{1}{n}$),
necessarily $j < n$.

$j$ gives us three sub-intervals to span:

\begin{itemize}
\item $\interval{a}{\rep{0}{j} \rep{1}{n-j}}$
\item $\interval{\rep{0}{j-1} 1 \rep{0}{n-j}}
{\rep{1}{j-1} 0 \rep{1}{n-j}}$
\item $\interval{\rep{1}{j} \rep{0}{n-j}}{b}$
\end{itemize}

Note that the second sub-interval contains exactly
the numbers that don't start with either $j$ zeros
or $j$ ones,
and as such can be spanned by $j$ ternary vectors.
We construct $T_1, \ldots, T_j$ that span
the second sub-interval
by appending $\rep{\phi}{n-j}$
to each of the $j$ cyclic shifts of $01 \rep{\phi}{j-2}$.
Thus,
$T_1 = 01 \rep{\phi}{j-2} \rep{\phi}{n-j}$,
$\ldots$,
$T_i = \rep{\phi}{i-1} 01 \rep{\phi}{j-1-i} \rep{\phi}{n-j}$
(for $i \in \curly{1, \ldots, j-1}$),
$\ldots$,
$T_{j-1} = \rep{\phi}{j-2} 01 \rep{\phi}{n-j}$,
$T_j = 1 \rep{\phi}{j-2} 0 \rep{\phi}{n-j}$.

The other two sub-intervals are spanned recursively.
Let $a'' = \bits{a}{j}{n}$ and $b'' = \bits{b}{j}{n}$.
Note that $\bit{a''}{1} = 0$ and $\bit{b''}{1} = 1$.
Note that $|a''| = |b''| = n - j + 1$.
Since $j \geq 2$, $n-j+1 < n$.
Let $\mathcal{T}''$ be the spanning set
of $\interval{a''}{b''}$ computed recursively.

The spanning set of $\interval{a}{b}$ will be computed
from the vectors $T_1, \ldots, T_j$ and $\mathcal{T}''$
based on the relation between $(n-j)$-bit suffixes of $a$
and $b$.
Let $a' = \bits{a}{j+1}{n}$ and $b' = \bits{b}{j+1}{n}$.

\subsubsection{\texorpdfstring
{$b' < a' - 1$}
{b' < a' - 1}
}

In this case,
the spanning set of $\interval{a}{b}$ consists
of the vectors $T_1, \ldots, T_j$ and vectors obtained
by prepending each vector from $\mathcal{T}''$ with
$\rep{\phi}{j-1}$.

\subsubsection{\texorpdfstring
{$b' \leq a' - 1$}
{b' >= a' - 1}
}

In this case,
the spanning set of $\interval{a}{b}$ consists
of the vectors $T_1, \ldots, T_{j-1}$ (omitting $T_j$)
and a set $\mathcal{T}'$ which is derived
from $\mathcal{T}''$
in the following way:

\begin{align*}
\mathcal{T}' &= \curly{
\rep{\phi}{j-1} T | T \in \mathcal{T}'' \text{ and }
\bit{T}{1} \in \curly{0, \phi}
} \\
&\cup \curly{
1 \rep{\phi}{j-1} \bits{T}{2}{n-j+1} | T \in \mathcal{T}''
\text{ and } \bit{T}{1} = 1
}
\end{align*}

% TODO: Go into more detail.
